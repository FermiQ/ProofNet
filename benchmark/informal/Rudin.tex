\documentclass{article}

\title{\textbf{
Exercises from \\
\textit{Principles of Mathematical Analysis} \\
by Walter Rudin
}}

\date{}

\usepackage{amsmath}
\usepackage{amssymb}

\begin{document}
\maketitle

\paragraph{Exercise 1.1a} If $r$ is rational $(r \neq 0)$ and $x$ is irrational, prove that $r+x$ is irrational.

\paragraph{Exercise 1.1b} If $r$ is rational $(r \neq 0)$ and $x$ is irrational, prove that $rx$ is irrational.

\paragraph{Exercise 1.2} Prove that there is no rational number whose square is $12$.

\paragraph{Exercise 1.4} Let $E$ be a nonempty subset of an ordered set; suppose $\alpha$ is a lower bound of $E$ and $\beta$ is an upper bound of $E$. Prove that $\alpha \leq \beta$.

\paragraph{Exercise 1.5} Let $A$ be a nonempty set of real numbers which is bounded below. Let $-A$ be the set of all numbers $-x$, where $x \in A$. Prove that $\inf A=-\sup (-A)$.

\paragraph{Exercise 1.8} Prove that no order can be defined in the complex field that turns it into an ordered field.

\paragraph{Exercise 1.11a} If $z$ is a complex number, prove that there exists an $r\geq 0$ and a complex number $w$ with $| w | = 1$ such that $z = rw$.

\paragraph{Exercise 1.12} If $z_1, \ldots, z_n$ are complex, prove that $|z_1 + z_2 + \ldots + z_n| \leq |z_1| + |z_2| + \cdots + |z_n|$.

\paragraph{Exercise 1.13} If $x, y$ are complex, prove that $||x|-|y|| \leq |x-y|$.

\paragraph{Exercise 1.14} If $z$ is a complex number such that $|z|=1$, that is, such that $z \bar{z}=1$, compute $|1+z|^{2}+|1-z|^{2}$.

\paragraph{Exercise 1.16a} Suppose $k \geq 3, x, y \in \mathbb{R}^k, |x - y| = d > 0$, and $r > 0$. Prove that if $2r > d$, there are infinitely many $z \in \mathbb{R}^k$ such that $|z-x|=|z-y|=r$.

\paragraph{Exercise 1.17} Prove that $|\mathbf{x}+\mathbf{y}|^{2}+|\mathbf{x}-\mathbf{y}|^{2}=2|\mathbf{x}|^{2}+2|\mathbf{y}|^{2}$ if $\mathbf{x} \in R^{k}$ and $\mathbf{y} \in R^{k}$.

\paragraph{Exercise 1.18a} If $k \geq 2$ and $\mathbf{x} \in R^{k}$, prove that there exists $\mathbf{y} \in R^{k}$ such that $\mathbf{y} \neq 0$ but $\mathbf{x} \cdot \mathbf{y}=0$

\paragraph{Exercise 1.18b} If $k = 1$ and $\mathbf{x} \in R^{k}$, prove that there does not exist $\mathbf{y} \in R^{k}$ such that $\mathbf{y} \neq 0$ but $\mathbf{x} \cdot \mathbf{y}=0$

\paragraph{Exercise 1.19} Suppose $a, b \in R^k$. Find $c \in R^k$ and $r > 0$ such that $|x-a|=2|x-b|$ if and only if $| x - c | = r$. Prove that $3c = 4b - a$ and $3r = 2 |b - a|$.

\paragraph{Exercise 2.19a} If $A$ and $B$ are disjoint closed sets in some metric space $X$, prove that they are separated.

\paragraph{Exercise 2.24} Let $X$ be a metric space in which every infinite subset has a limit point. Prove that $X$ is separable. Hint: Fix $\delta>0$, and pick $x_{1} \in X$. Having chosen $x_{1}, \ldots, x_{J} \in X$,

\paragraph{Exercise 2.25} Prove that every compact metric space $K$ has a countable base.

\paragraph{Exercise 2.27a} Suppose $E\subset\mathbb{R}^k$ is uncountable, and let $P$ be the set of condensation points of $E$. Prove that $P$ is perfect.

\paragraph{Exercise 2.27b} Suppose $E\subset\mathbb{R}^k$ is uncountable, and let $P$ be the set of condensation points of $E$. Prove that at most countably many point of $E$ are not in $P$.

\paragraph{Exercise 2.28} Prove that every closed set in a separable metric space is the union of a (possibly empty) perfect set and a set which is at most countable.

\paragraph{Exercise 2.29} Prove that every open set in $\mathbb{R}$ is the union of an at most countable collection of disjoint segments.

\paragraph{Exercise 3.1a} Prove that convergence of $\left\{s_{n}\right\}$ implies convergence of $\left\{\left|s_{n}\right|\right\}$.

\paragraph{Exercise 3.2a} Prove that $\lim_{n \rightarrow \infty}\sqrt{n^2 + n} -n = 1/2$.

\paragraph{Exercise 3.3} If $s_{1}=\sqrt{2}$, and $s_{n+1}=\sqrt{2+\sqrt{s_{n}}} \quad(n=1,2,3, \ldots),$ prove that $\left\{s_{n}\right\}$ converges, and that $s_{n}<2$ for $n=1,2,3, \ldots$.

\paragraph{Exercise 3.5} For any two real sequences $\left\{a_{n}\right\},\left\{b_{n}\right\}$, prove that $\limsup _{n \rightarrow \infty}\left(a_{n}+b_{n}\right) \leq \limsup _{n \rightarrow \infty} a_{n}+\limsup _{n \rightarrow \infty} b_{n},$ provided the sum on the right is not of the form $\infty-\infty$.

\paragraph{Exercise 3.6a} Prove that $\lim_{n \rightarrow \infty} \sum_{i<n} a_i = \infty$, where $a_i = \sqrt{i + 1} -\sqrt{i}$.

\paragraph{Exercise 3.7} Prove that the convergence of $\Sigma a_{n}$ implies the convergence of $\sum \frac{\sqrt{a_{n}}}{n}$ if $a_n\geq 0$.

\paragraph{Exercise 3.8} If $\Sigma a_{n}$ converges, and if $\left\{b_{n}\right\}$ is monotonic and bounded, prove that $\Sigma a_{n} b_{n}$ converges.

\paragraph{Exercise 3.13} Prove that the Cauchy product of two absolutely convergent series converges absolutely.

\paragraph{Exercise 3.20} Suppose $\left\{p_{n}\right\}$ is a Cauchy sequence in a metric space $X$, and some sequence $\left\{p_{n l}\right\}$ converges to a point $p \in X$. Prove that the full sequence $\left\{p_{n}\right\}$ converges to $p$.

\paragraph{Exercise 3.21} If $\left\{E_{n}\right\}$ is a sequence of closed nonempty and bounded sets in a complete metric space $X$, if $E_{n} \supset E_{n+1}$, and if $\lim _{n \rightarrow \infty} \operatorname{diam} E_{n}=0,$ then $\bigcap_{1}^{\infty} E_{n}$ consists of exactly one point.

\paragraph{Exercise 3.22} Suppose $X$ is a nonempty complete metric space, and $\left\{G_{n}\right\}$ is a sequence of dense open sets of $X$. Prove Baire's theorem, namely, that $\bigcap_{1}^{\infty} G_{n}$ is not empty.

\paragraph{Exercise 4.1a} Suppose $f$ is a real function defined on $\mathbb{R}$ which satisfies $\lim_{h \rightarrow 0} f(x + h) - f(x - h) = 0$ for every $x \in \mathbb{R}$. Show that $f$ does not need to be continuous.

\paragraph{Exercise 4.2a} If $f$ is a continuous mapping of a metric space $X$ into a metric space $Y$, prove that $f(\overline{E}) \subset \overline{f(E)}$ for every set $E \subset X$. ($\overline{E}$ denotes the closure of $E$).

\paragraph{Exercise 4.3} Let $f$ be a continuous real function on a metric space $X$. Let $Z(f)$ (the zero set of $f$ ) be the set of all $p \in X$ at which $f(p)=0$. Prove that $Z(f)$ is closed.

\paragraph{Exercise 4.4a} Let $f$ and $g$ be continuous mappings of a metric space $X$ into a metric space $Y$, and let $E$ be a dense subset of $X$. Prove that $f(E)$ is dense in $f(X)$.

\paragraph{Exercise 4.4b} Let $f$ and $g$ be continuous mappings of a metric space $X$ into a metric space $Y$, and let $E$ be a dense subset of $X$. Prove that if $g(p) = f(p)$ for all $p \in P$ then $g(p) = f(p)$ for all $p \in X$.

\paragraph{Exercise 4.5a} If $f$ is a real continuous function defined on a closed set $E \subset \mathbb{R}$, prove that there exist continuous real functions $g$ on $\mathbb{R}$ such that $g(x)=f(x)$ for all $x \in E$.

\paragraph{Exercise 4.5b} Show that there exist a set $E \subset \mathbb{R}$ and a real continuous function $f$ defined on $E$, such that there does not exist a continuous real function $g$ on $\mathbb{R}$ such that $g(x)=f(x)$ for all $x \in E$.

\paragraph{Exercise 4.6} If $f$ is defined on $E$, the graph of $f$ is the set of points $(x, f(x))$, for $x \in E$. In particular, if $E$ is a set of real numbers, and $f$ is real-valued, the graph of $f$ is a subset of the plane. Suppose $E$ is compact, and prove that $f$ is continuous on $E$ if and only if its graph is compact.

\paragraph{Exercise 4.8a} Let $f$ be a real uniformly continuous function on the bounded set $E$ in $R^{1}$. Prove that $f$ is bounded on $E$.

\paragraph{Exercise 4.11a} Suppose $f$ is a uniformly continuous mapping of a metric space $X$ into a metric space $Y$ and prove that $\left\{f\left(x_{n}\right)\right\}$ is a Cauchy sequence in

\paragraph{Exercise 4.12} A uniformly continuous function of a uniformly continuous function is uniformly continuous.

\paragraph{Exercise 4.14} Let $I=[0,1]$ be the closed unit interval. Suppose $f$ is a continuous mapping of $I$ into $I$. Prove that $f(x)=x$ for at least one $x \in I$.

\paragraph{Exercise 4.15} Prove that every continuous open mapping of $R^{1}$ into $R^{1}$ is monotonic.

\paragraph{Exercise 4.19} Suppose $f$ is a real function with domain $R^{1}$ which has the intermediate value property. If $f(a)<c<f(b)$, then $f(x)=c$ for some $x$ between $a$ and $b$. Suppose also, for every rational $r$, that the set of all $x$ with $f(x)=r$ is closed. Prove that $f$ is continuous.

\paragraph{Exercise 4.21a} Suppose $K$ and $F$ are disjoint sets in a metric space $X, K$ is compact, $F$ is closed. Prove that there exists $\delta>0$ such that $d(p, q)>\delta$ if $p \in K, q \in F$.

\paragraph{Exercise 4.24} Assume that $f$ is a continuous real function defined in $(a, b)$ such that $f\left(\frac{x+y}{2}\right) \leq \frac{f(x)+f(y)}{2}$ for all $x, y \in(a, b)$. Prove that $f$ is convex.

\paragraph{Exercise 4.26a} Suppose $X, Y, Z$ are metric spaces, and $Y$ is compact. Let $f$ map $X$ into $Y$, let $g$ be a continuous one-to-one mapping of $Y$ into $Z$, and put $h(x)=g(f(x))$ for $x \in X$. Prove that $f$ is uniformly continuous if $h$ is uniformly continuous.

\paragraph{Exercise 5.1} Let $f$ be defined for all real $x$, and suppose that $|f(x)-f(y)| \leq(x-y)^{2}$for all real $x$ and $y$. Prove that $f$ is constant.

\paragraph{Exercise 5.2} Suppose $f^{\prime}(x)>0$ in $(a, b)$. Prove that $f$ is strictly increasing in $(a, b)$, and let $g$ be its inverse function. Prove that $g$ is differentiable, and that$g^{\prime}(f(x))=\frac{1}{f^{\prime}(x)} \quad(a<x<b)$

\paragraph{Exercise 5.3} Suppose $g$ is a real function on $R^{1}$, with bounded derivative (say $\left|g^{\prime}\right| \leq M$ ). Fix $\varepsilon>0$, and define $f(x)=x+\varepsilon g(x)$. Prove that $f$ is one-to-one if $\varepsilon$ is small enough.

\paragraph{Exercise 5.4} If $C_{0}+\frac{C_{1}}{2}+\cdots+\frac{C_{n-1}}{n}+\frac{C_{n}}{n+1}=0,$ where $C_{0}, \ldots, C_{n}$ are real constants, prove that the equation $C_{0}+C_{1} x+\cdots+C_{n-1} x^{n-1}+C_{n} x^{n}=0$ has at least one real root between 0 and 1 .

\paragraph{Exercise 5.5} Suppose $f$ is defined and differentiable for every $x>0$, and $f^{\prime}(x) \rightarrow 0$ as $x \rightarrow+\infty$. Put $g(x)=f(x+1)-f(x)$. Prove that $g(x) \rightarrow 0$ as $x \rightarrow+\infty$.

\paragraph{Exercise 5.6} Suppose (a) $f$ is continuous for $x \geq 0$, (b) $f^{\prime}(x)$ exists for $x>0$, (c) $f(0)=0$, (d) $f^{\prime}$ is monotonically increasing. Put $g(x)=\frac{f(x)}{x} \quad(x>0)$ and prove that $g$ is monotonically increasing.

\paragraph{Exercise 5.7} Suppose $f^{\prime}(x), g^{\prime}(x)$ exist, $g^{\prime}(x) \neq 0$, and $f(x)=g(x)=0$. Prove that $\lim _{t \rightarrow x} \frac{f(t)}{g(t)}=\frac{f^{\prime}(x)}{g^{\prime}(x)}.$

\paragraph{Exercise 5.15} Suppose $a \in R^{1}, f$ is a twice-differentiable real function on $(a, \infty)$, and $M_{0}, M_{1}, M_{2}$ are the least upper bounds of $|f(x)|,\left|f^{\prime}(x)\right|,\left|f^{\prime \prime}(x)\right|$, respectively, on $(a, \infty)$. Prove that $M_{1}^{2} \leq 4 M_{0} M_{2} .$

\paragraph{Exercise 5.17} Suppose $f$ is a real, three times differentiable function on $[-1,1]$, such that $f(-1)=0, \quad f(0)=0, \quad f(1)=1, \quad f^{\prime}(0)=0 .$ Prove that $f^{(3)}(x) \geq 3$ for some $x \in(-1,1)$.

\paragraph{Exercise 6.1} Suppose $\alpha$ increases on $[a, b], a \leq x_{0} \leq b, \alpha$ is continuous at $x_{0}, f\left(x_{0}\right)=1$, and $f(x)=0$ if $x \neq x_{0}$. Prove that $f \in \mathcal{R}(\alpha)$ and that $\int f d \alpha=0$.

\paragraph{Exercise 6.2} Suppose $f \geq 0, f$ is continuous on $[a, b]$, and $\int_{a}^{b} f(x) d x=0$. Prove that $f(x)=0$ for all $x \in[a, b]$.

\paragraph{Exercise 6.4} If $f(x)=0$ for all irrational $x, f(x)=1$ for all rational $x$, prove that $f \notin \mathcal{R}$ on $[a, b]$ for any $a<b$.

\paragraph{Exercise 6.6} Let $P$ be the Cantor set. Let $f$ be a bounded real function on $[0,1]$ which is continuous at every point outside $P$. Prove that $f \in \mathcal{R}$ on $[0,1]$.

\end{document}
