
\documentclass{article}

\title{\textbf{
Exercises from \\
\textit{Topology} \\
by James Munkres
}}

\date{}

\usepackage{amsmath}
\usepackage{amssymb}

\begin{document}
\maketitle

\paragraph{Exercise 13.1} Let $X$ be a topological space; let $A$ be a subset of $X$. Suppose that for each $x \in A$ there is an open set $U$ containing $x$ such that $U \subset A$. Show that $A$ is open in $X$.

\paragraph{Exercise 13.3a} Let $X$ be a set, let $\mathcal{T}_c$ be the collection of all subsets $U$ of $X$ such that $X - U$ either is countable or is all of $X$. Show that $\mathcal{T}_c$ is a topology on the set $X$.

\paragraph{Exercise 13.3b} Show that the collection $$\mathcal{T}_\infty = \{U | X - U \text{ is infinite or empty or all of X}\}$$ is does not need to be a topology on the set $X$.

\paragraph{Exercise 13.4a1} If $\mathcal{T}_\alpha$ is a family of topologies on $X$, show that $\bigcap \mathcal{T}_\alpha$ is a topology on $X$.

\paragraph{Exercise 13.4a2} If $\mathcal{T}_\alpha$ is a family of topologies on $X$, show that $\bigcup \mathcal{T}_\alpha$ does not need to be a topology on $X$.

\paragraph{Exercise 13.4b1} Let $\mathcal{T}_\alpha$ be a family of topologies on $X$. Show that there is a unique smallest topology on $X$ containing all the collections $\mathcal{T}_\alpha$.

\paragraph{Exercise 13.4b2} Let $\mathcal{T}_\alpha$ be a family of topologies on $X$. Show that there is a unique largest topology on $X$ contained in all the collections $\mathcal{T}_\alpha$.

\paragraph{Exercise 13.5a} Show that if $\mathcal{A}$ is a basis for a topology on $X$, then the topology generated by $\mathcal{A}$ equals the intersection of all topologies on $X$ that contain $\mathcal{A}$.

\paragraph{Exercise 13.5b} Show that if $\mathcal{A}$ is a subbasis for a topology on $X$, then the topology generated by $\mathcal{A}$ equals the intersection of all topologies on $X$ that contain $\mathcal{A}$.

\paragraph{Exercise 13.6} Show that the lower limit topology $\mathbb{R}_l$ and $K$-topology $\mathbb{R}_K$ are not comparable.

\paragraph{Exercise 13.8a} Show that the collection $\{(a,b) \mid a < b, a \text{ and } b \text{ rational}\}$ is a basis that generates the standard topology on $\mathbb{R}$.

\paragraph{Exercise 13.8b} Show that the collection $\{(a,b) \mid a < b, a \text{ and } b \text{ rational}\}$ is a basis that generates a topology different from the lower limit topology on $\mathbb{R}$.

\paragraph{Exercise 16.1} Show that if $Y$ is a subspace of $X$, and $A$ is a subset of $Y$, then the topology $A$ inherits as a subspace of $Y$ is the same as the topology it inherits as a subspace of $X$.

\paragraph{Exercise 16.4} A map $f: X \rightarrow Y$ is said to be an open map if for every open set $U$ of $X$, the set $f(U)$ is open in $Y$. Show that $\pi_{1}: X \times Y \rightarrow X$ and $\pi_{2}: X \times Y \rightarrow Y$ are open maps.

\paragraph{Exercise 16.6} Show that the countable collection \[\{(a, b) \times (c, d) \mid a < b \text{ and } c < d, \text{ and } a, b, c, d \text{ are rational}\}\] is a basis for $\mathbb{R}^2$.

\paragraph{Exercise 16.9} Show that the dictionary order topology on the set $\mathbb{R} \times \mathbb{R}$ is the same as the product topology $\mathbb{R}_d \times \mathbb{R}$, where $\mathbb{R}_d$ denotes $\mathbb{R}$ in the discrete topology.

\paragraph{Exercise 17.2} Show that if $A$ is closed in $Y$ and $Y$ is closed in $X$, then $A$ is closed in $X$.

\paragraph{Exercise 17.3} Show that if $A$ is closed in $X$ and $B$ is closed in $Y$, then $A \times B$ is closed in $X \times Y$.

\paragraph{Exercise 17.4} Show that if $U$ is open in $X$ and $A$ is closed in $X$, then $U-A$ is open in $X$, and $A-U$ is closed in $X$.

\paragraph{Exercise 18.8a} Let $Y$ be an ordered set in the order topology. Let $f, g: X \rightarrow Y$ be continuous. Show that the set $\{x \mid f(x) \leq g(x)\}$ is closed in $X$.

\paragraph{Exercise 18.8b} Let $Y$ be an ordered set in the order topology. Let $f, g: X \rightarrow Y$ be continuous. Let $h: X \rightarrow Y$ be the function $h(x)=\min \{f(x), g(x)\}.$ Show that $h$ is continuous.

\paragraph{Exercise 18.13} Let $A \subset X$; let $f: A \rightarrow Y$ be continuous; let $Y$ be Hausdorff. Show that if $f$ may be extended to a continuous function $g: \bar{A} \rightarrow Y$, then $g$ is uniquely determined by $f$.

\paragraph{Exercise 19.4} Show that $(X_1 \times  \cdots \times X_{n-1}) \times X_n$ is homeomorphic with $X_1 \times  \cdots \times X_n$.

\paragraph{Exercise 19.6a} Let $\mathbf{x}_1, \mathbf{x}_2, \ldots$ be a sequence of the points of the product space $\prod X_\alpha$.  Show that this sequence converges to the point $\mathbf{x}$ if and only if the sequence $\pi_\alpha(\mathbf{x}_i)$ converges to $\pi_\alpha(\mathbf{x})$ for each $\alpha$.

\paragraph{Exercise 19.9} Show that the choice axiom is equivalent to the statement that for any indexed family of nonempty sets, $\{A_\alpha\}_{\alpha \in J}$ with $J \neq 0$, the cartesian product \[\prod_{\alpha \in J} A_\alpha\] is not empty.

\paragraph{Exercise 20.2} Show that $\mathbb{R} \times \mathbb{R}$ in the dictionary order topology is metrizable.

\paragraph{Exercise 20.5} Let $\mathbb{R}^\infty$ be the subset of $\mathbb{R}^\omega$ consisting of all sequences that are eventually zero.  What is the closure of $\mathbb{R}^\infty$ in $\mathbb{R}^\omega$ in the uniform topology? Justify your answer.

\paragraph{Exercise 21.6a} Define $f_{n}:[0,1] \rightarrow \mathbb{R}$ by the equation $f_{n}(x)=x^{n}$. Show that the sequence $\left(f_{n}(x)\right)$ converges for each $x \in[0,1]$.

\paragraph{Exercise 21.6b} Define $f_{n}:[0,1] \rightarrow \mathbb{R}$ by the equation $f_{n}(x)=x^{n}$. Show that the sequence $\left(f_{n}\right)$ does not converge uniformly.

\paragraph{Exercise 21.8} Let $X$ be a topological space and let $Y$ be a metric space. Let $f_{n}: X \rightarrow Y$ be a sequence of continuous functions. Let $x_{n}$ be a sequence of points of $X$ converging to $x$. Show that if the sequence $\left(f_{n}\right)$ converges uniformly to $f$, then $\left(f_{n}\left(x_{n}\right)\right)$ converges to $f(x)$.

\paragraph{Exercise 22.2a} Let $p: X \rightarrow Y$ be a continuous map. Show that if there is a continuous map $f: Y \rightarrow X$ such that $p \circ f$ equals the identity map of $Y$, then $p$ is a quotient map.

\paragraph{Exercise 22.2b} If $A \subset X$, a retraction of $X$ onto $A$ is a continuous map $r: X \rightarrow A$ such that $r(a)=a$ for each $a \in A$. Show that a retraction is a quotient map.

\paragraph{Exercise 22.5} Let $p \colon X \rightarrow Y$ be an open map. Show that if $A$ is open in $X$, then the map $q \colon A \rightarrow p(A)$ obtained by restricting $p$ is an open map.

\paragraph{Exercise 23.2} Let $\left\{A_{n}\right\}$ be a sequence of connected subspaces of $X$, such that $A_{n} \cap A_{n+1} \neq \varnothing$ for all $n$. Show that $\bigcup A_{n}$ is connected.

\paragraph{Exercise 23.3} Let $\left\{A_{\alpha}\right\}$ be a collection of connected subspaces of $X$; let $A$ be a connected subset of $X$. Show that if $A \cap A_{\alpha} \neq \varnothing$ for all $\alpha$, then $A \cup\left(\bigcup A_{\alpha}\right)$ is connected.

\paragraph{Exercise 23.4} Show that if $X$ is an infinite set, it is connected in the finite complement topology.

\paragraph{Exercise 23.6} Let $A \subset X$. Show that if $C$ is a connected subspace of $X$ that intersects both $A$ and $X-A$, then $C$ intersects $\operatorname{Bd} A$.

\paragraph{Exercise 23.9} Let $A$ be a proper subset of $X$, and let $B$ be a proper subset of $Y$. If $X$ and $Y$ are connected, show that $(X \times Y)-(A \times B)$ is connected.

\paragraph{Exercise 23.11} Let $p: X \rightarrow Y$ be a quotient map. Show that if each set $p^{-1}(\{y\})$ is connected, and if $Y$ is connected, then $X$ is connected.

\paragraph{Exercise 23.12} Let $Y \subset X$; let $X$ and $Y$ be connected. Show that if $A$ and $B$ form a separation of $X-Y$, then $Y \cup A$ and $Y \cup B$ are connected.

\paragraph{Exercise 24.2} Let $f: S^{1} \rightarrow \mathbb{R}$ be a continuous map. Show there exists a point $x$ of $S^{1}$ such that $f(x)=f(-x)$.

\paragraph{Exercise 24.3a} Let $f \colon X \rightarrow X$ be continuous. Show that if $X = [0, 1]$, there is a point $x$ such that $f(x) = x$. (The point $x$ is called a fixed point of $f$.)

\paragraph{Exercise 24.4} Let $X$ be an ordered set in the order topology. Show that if $X$ is connected, then $X$ is a linear continuum.

\paragraph{Exercise 24.6} Show that if $X$ is a well-ordered set, then $X \times [0, 1)$ in the dictionary order is a linear continuum.

\paragraph{Exercise 25.4} Let $X$ be locally path connected. Show that every connected open set in $X$ is path connected.

\paragraph{Exercise 25.9} Let $G$ be a topological group; let $C$ be the component of $G$ containing the identity element $e$. Show that $C$ is a normal subgroup of $G$.

\paragraph{Exercise 26.9} Let $A$ and $B$ be subspaces of $X$ and $Y$, respectively; let $N$ be an open set in $X \times Y$ containing $A \times B$. If $A$ and $B$ are compact, then there exist open sets $U$ and $V$ in $X$ and $Y$, respectively, such that $A \times B \subset U \times V \subset N .$

\paragraph{Exercise 26.11} Let $X$ be a compact Hausdorff space. Let $\mathcal{A}$ be a collection of closed connected subsets of $X$ that is simply ordered by proper inclusion. Then $Y=\bigcap_{A \in \mathcal{A}} A$ is connected.

\paragraph{Exercise 26.12} Let $p: X \rightarrow Y$ be a closed continuous surjective map such that $p^{-1}(\{y\})$ is compact, for each $y \in Y$. (Such a map is called a perfect map.) Show that if $Y$ is compact, then $X$ is compact.

\paragraph{Exercise 27.1} Prove that if $X$ is an ordered set in which every closed interval is compact, then $X$ has the least upper bound property.

\paragraph{Exercise 27.4} Show that a connected metric space having more than one point is uncountable.

\paragraph{Exercise 28.4} A space $X$ is said to be countably compact if every countable open covering of $X$ contains a finite subcollection that covers $X$. Show that for a $T_1$ space $X$, countable compactness is equivalent to limit point compactness.

\paragraph{Exercise 28.5} Show that X is countably compact if and only if every nested sequence $C_1 \supset C_2 \supset \cdots$ of closed nonempty sets of X has a nonempty intersection.

\paragraph{Exercise 28.6} Let $(X, d)$ be a metric space. If $f: X \rightarrow X$ satisfies the condition $d(f(x), f(y))=d(x, y)$ for all $x, y \in X$, then $f$ is called an isometry of $X$. Show that if $f$ is an isometry and $X$ is compact, then $f$ is bijective and hence a homeomorphism.

\paragraph{Exercise 29.1} Show that the rationals $\mathbb{Q}$ are not locally compact.

\paragraph{Exercise 29.4} Show that $[0, 1]^\omega$ is not locally compact in the uniform topology.

\paragraph{Exercise 29.5} If $f \colon X_1 \rightarrow X_2$ is a homeomorphism of locally compact Hausdorff spaces, show that $f$ extends to a homeomorphism of their one-point compactifications.

\paragraph{Exercise 29.6} Show that the one-point compactification of $\mathbb{R}$ is homeomorphic with the circle $S^1$.

\paragraph{Exercise 29.10} Show that if $X$ is a Hausdorff space that is locally compact at the point $x$, then for each neighborhood $U$ of $x$, there is a neighborhood $V$ of $x$ such that $\bar{V}$ is compact and $\bar{V} \subset U$.

\paragraph{Exercise 30.10} Show that if $X$ is a countable product of spaces having countable dense subsets, then $X$ has a countable dense subset.

\paragraph{Exercise 30.13} Show that if $X$ has a countable dense subset, every collection of disjoint open sets in $X$ is countable.

\paragraph{Exercise 31.1} Show that if $X$ is regular, every pair of points of $X$ have neighborhoods whose closures are disjoint.

\paragraph{Exercise 31.2} Show that if $X$ is normal, every pair of disjoint closed sets have neighborhoods whose closures are disjoint.

\paragraph{Exercise 31.3} Show that every order topology is regular.

\paragraph{Exercise 32.1} Show that a closed subspace of a normal space is normal.

\paragraph{Exercise 32.2a} Show that if $\prod X_\alpha$ is Hausdorff, then so is $X_\alpha$. Assume that each $X_\alpha$ is nonempty.

\paragraph{Exercise 32.2b} Show that if $\prod X_\alpha$ is regular, then so is $X_\alpha$. Assume that each $X_\alpha$ is nonempty.

\paragraph{Exercise 32.2c} Show that if $\prod X_\alpha$ is normal, then so is $X_\alpha$. Assume that each $X_\alpha$ is nonempty.

\paragraph{Exercise 32.3} Show that every locally compact Hausdorff space is regular.

\paragraph{Exercise 33.7} Show that every locally compact Hausdorff space is completely regular.

\paragraph{Exercise 33.8} Let $X$ be completely regular, let $A$ and $B$ be disjoint closed subsets of $X$. Show that if $A$ is compact, there is a continuous function $f \colon X \rightarrow [0, 1]$ such that $f(A) = \{0\}$ and $f(B) = \{1\}$.

\paragraph{Exercise 34.9} Let $X$ be a compact Hausdorff space that is the union of the closed subspaces $X_1$ and $X_2$. If $X_1$ and $X_2$ are metrizable, show that $X$ is metrizable.

\paragraph{Exercise 37.2} A collection $\mathcal{A}$ of subsets of $X$ has the countable intersection property if every countable intersection of elements of $\mathcal{A}$ is nonempty. Show that $X$ is a Lindelöf space if and only if for every collection $\mathcal{A}$ of subsets of $X$ having the countable intersection property, $\bigcap_{A \in \mathcal{A}} \bar{A}$ is nonempty.

\paragraph{Exercise 38.4} Let $Y$ be an arbitrary compactification of $X$; let $\beta(X)$ be the Stone-Čech compactification. Show there is a continuous surjective closed map $g \colon \beta(X)\rightarrow Y$ that equals the identity on $X$.

\paragraph{Exercise 38.6} Let $X$ be completely regular. Show that $X$ is connected if and only if the Stone-Čech compactification of $X$ is connected.

\paragraph{Exercise 39.5} Show that if $X$ has a countable basis, a collection $\mathcal{A}$ of subsets of $X$ is countably locally finite if and only if it is countable.

\paragraph{Exercise 43.2} Let $(X, d_X)$ and $(Y, d_Y)$ be metric spaces; let $Y$ be complete. Let $A \subset X$. Show that if $f \colon A \rightarrow Y$ is uniformly continuous, then $f$ can be uniquely extended to a continuous function $g \colon \bar{A} \rightarrow Y$, and $g$ is uniformly continuous.

\paragraph{Exercise 43.7} Show that the set of all sequences $(x_1, x_2, \ldots)$ such that $\sum x_i^2$ converges is complete in $l^2$-metric.

\end{document}
