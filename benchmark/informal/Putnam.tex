
\documentclass{article}

\title{\textbf{
Exercises from \\
\textit{Putnam Competition} \\
}}

\date{}

\usepackage{amsmath}
\usepackage{amssymb}

\begin{document}
\maketitle

\paragraph{Exercise 2021.b4} Let $F_{0}, F_{1}, \ldots$ be the sequence of Fibonacci numbers, with $F_{0}=0, F_{1}=1$, and $F_{n}=F_{n-1}+F_{n-2}$ for $n \geq 2$. For $m>2$, let $R_{m}$ be the remainder when the product $\prod_{k=1}^{F_{m}-1} k^{k}$ is divided by $F_{m}$. Prove that $R_{m}$ is also a Fibonacci number.

\paragraph{Exercise 2020.b5} For $j \in\{1,2,3,4\}$, let $z_{j}$ be a complex number with $\left|z_{j}\right|=1$ and $z_{j} \neq 1$. Prove that $3-z_{1}-z_{2}-z_{3}-z_{4}+z_{1} z_{2} z_{3} z_{4} \neq 0 .$

\paragraph{Exercise 2018.a5} Let $f: \mathbb{R} \rightarrow \mathbb{R}$ be an infinitely differentiable function satisfying $f(0)=0, f(1)=1$, and $f(x) \geq 0$ for all $x \in$ $\mathbb{R}$. Show that there exist a positive integer $n$ and a real number $x$ such that $f^{(n)}(x)<0$.

\paragraph{Exercise 2018.b2} Let $n$ be a positive integer, and let $f_{n}(z)=n+(n-1) z+$ $(n-2) z^{2}+\cdots+z^{n-1}$. Prove that $f_{n}$ has no roots in the closed unit disk $\{z \in \mathbb{C}:|z| \leq 1\}$.

\paragraph{Exercise 2018.b4} Given a real number $a$, we define a sequence by $x_{0}=1$, $x_{1}=x_{2}=a$, and $x_{n+1}=2 x_{n} x_{n-1}-x_{n-2}$ for $n \geq 2$. Prove that if $x_{n}=0$ for some $n$, then the sequence is periodic.

\paragraph{Exercise 2018.b6} Let $S$ be the set of sequences of length 2018 whose terms are in the set $\{1,2,3,4,5,6,10\}$ and sum to 3860 . Prove that the cardinality of $S$ is at most $2^{3860} \cdot\left(\frac{2018}{2048}\right)^{2018} .$

\paragraph{Exercise 2017.b3} Suppose that $f(x)=\sum_{i=0}^{\infty} c_{i} x^{i}$ is a power series for which each coefficient $c_{i}$ is 0 or 1 . Show that if $f(2 / 3)=3 / 2$, then $f(1 / 2)$ must be irrational.

\paragraph{Exercise 2016.a6} Suppose that $G$ is a finite group generated by the two elements $g$ and $h$, where the order of $g$ is odd. Show that every element of $G$ can be written in the form $g^{m_1} h^{n_1} g^{m_2} h^{n_2} \cdots g^{m_r} h^{n_r}$ with $1 \leq r \leq|G|$ and $m_1, n_1, m_2, n_2, \ldots, m_r, n_r \in$ $\{-1,1\} .$ (Here $|G|$ is the number of elements of $G$.)

\paragraph{Exercise 2015.b1} Let $f$ be a three times differentiable function (defined on $\mathbb{R}$ and real-valued) such that $f$ has at least five distinct real zeros. Prove that $f+6 f^{\prime}+12 f^{\prime \prime}+8 f^{\prime \prime \prime}$ has at least two distinct real zeros.

\end{document}
