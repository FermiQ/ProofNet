
\documentclass{article}

\title{\textbf{
Exercises from \\
\textit{A Classical Introduction to Modern Number Theory} \\
by Kenneth Ireland and Michael Rosen
}}

\date{}

\usepackage{amsmath}
\usepackage{amssymb}
\usepackage{amsthm}

\begin{document}
\maketitle


\paragraph{Exercise 1.27} For all odd $n$ show that $8 \mid n^{2}-1$.
\begin{proof}
    We have $n^2-1=(n+1)(n-1)$. Since $n$ is odd, both $n+1, n-1$ are even, and moreso, one of these must be divisible by 4 , as one of the two consecutive odd numbers is divisible by 4 . Thus, their product is divisible by 8 . Similarly, if 3 does not divide $n$, it must divide one of $n-1, n+1$, otherwise it wouldn't divide three consecutive integers, which is impossible. As $n$ is odd, $n+1$ is even, so $(n+1)(n-1)$ is divisible by both 2 and 3 , so it is divisible by 6 .
\end{proof}



\paragraph{Exercise 1.30} Prove that $\frac{1}{2}+\frac{1}{3}+\cdots+\frac{1}{n}$ is not an integer.
\begin{proof}
Let $2^s$ be the largest power of 2 occuring as a denominator in $H_n$, say $2^s=k \leqslant n$. Write $H_n=$ $\frac{1}{2^s}+\left(1+1 / 2+\ldots+1 /(k-1)+1 /(k+1)+\ldots+1 / n\right.$. The sum in parentheses can be written as $1 / 2^{s-1}$ times sum of fractions with odd denominators, so the denominator of the sum in parentheses will not be divisible by $2^s$, but it must equal $2^s$ by Ex $1.29$.
\end{proof}



\paragraph{Exercise 1.31} Show that 2 is divisible by $(1+i)^{2}$ in $\mathbb{Z}[i]$.
\begin{proof}
We have $(1+i)^2=1+2 i-1=2 i$, so $2=-i(1+i)^2$.
\end{proof}



\paragraph{Exercise 2.4} If $a$ is a nonzero integer, then for $n>m$ show that $\left(a^{2^{n}}+1, a^{2^{m}}+1\right)=1$ or 2 depending on whether $a$ is odd or even.
\begin{proof}    
\begin{align*}
\operatorname{ord}_p\, n!  &= \sum_{k\geq 1} \left \lfloor \frac{n}{p^{k}}\right \rfloor \leq  \sum_{k\geq 1}  \frac{n}{p^{k}} = \frac{n}{p} \frac{1}{1 - \frac{1}{p}} = \frac{n}{p-1}
\end{align*}

The decomposition of $n!$ in prime factors is

$n! = p_1^{\alpha_1}p_2^{\alpha_2}\cdots p_k^{\alpha_k}$ 
where $\alpha_i = \operatorname{ord}_{p_i}\, n! \leq \frac{n}{p_i-1}$, and $p_i \leq n, \ i=1,2,\cdots,k$.

Then
\begin{align*}
n! &\leq p_1^{\frac{n}{p_1-1}}p_2^{\frac{n}{p_2-1}}\cdots p_k^{\frac{n}{p_n-1}}\\
\sqrt[n]{n!} &\leq p_1^{\frac{1}{p_1-1}}p_2^{\frac{1}{p_2-1}}\cdots p_k^{\frac{1}{p_n-1}}\\
&\leq \prod_{p\leq n} p^{\frac{1}{p-1}}
\end{align*}
(the values of $p$ in this product describe all prime numbers $p\leq n$.)
\end{proof}



\paragraph{Exercise 2.21} Define $\wedge(n)=\log p$ if $n$ is a power of $p$ and zero otherwise. Prove that $\sum_{A \mid n} \mu(n / d) \log d$ $=\wedge(n)$.
\begin{proof}    
$$
\left\{
\begin{array}{cccl}
    \land(n)& =  & \log p & \mathrm{if}\  n =p^\alpha,\ \alpha \in \mathbb{N}^*  \\
  &  = &   0 & \mathrm{otherwise }.
\end{array}
\right.
$$
Let $n = p_1^{\alpha_1}\cdots p_t^{\alpha_t}$ the decomposition of $n$ in prime factors. As $\land(d) = 0$ for all divisors of $n$, except for $d = p_j^i, i>0, j=1,\ldots t$,
\begin{align*}
\sum_{d \mid n} \land(d)&= \sum_{i=1}^{\alpha_1} \land(p_1^{i}) + \cdots+ \sum_{i=1}^{\alpha_t} \land(p_t^{i})\\ 
&= \alpha_1 \log p_1+\cdots + \alpha_t \log p_t\\
&= \log n
\end{align*}
By Mobius Inversion Theorem,
$$\land(n) = \sum_{d \mid n} \mu\left (\frac{n}{d}\right ) \log d.$$
\end{proof}



\paragraph{Exercise 2.27a} Show that $\sum^{\prime} 1 / n$, the sum being over square free integers, diverges.
\begin{proof}
    
Let $S \subset \mathbb{N}^*$ the set of square free integers.

Let $N \in \mathbb{N}^*$. Every integer $n, \, 1\leq n \leq N$ can be written as $n = a b^2$, where $a,b$ are integers and $a$ is square free. Then $1\leq a \leq N$, and $1\leq b \leq \sqrt{N}$, so
$$\sum_{n\leq N} \frac{1}{n} \leq \sum_{a \in S, a\leq N}\  \sum_{1\leq b \leq \sqrt{N}} \frac{1}{ab^2} \leq  \sum_{a \in S, a\leq N}\ \frac{1}{a} \, \sum_{b=1}^\infty  \frac{1}{b^2} = \frac{\pi^2}{6} \sum_{a \in S, a\leq N}\ \frac{1}{a}.$$
So $$\sum_{a \in S, a\leq N} \frac{1}{a}  \geq \frac{6}{\pi^2} \sum_{n\leq N} \frac{1}{n}.$$
As $\sum_{n=1}^\infty \frac{1}{n}$ diverges, $\lim\limits_{N \to \infty} \sum\limits_{a \in S, a\leq N} \frac{1}{a} = +\infty$, so the family $\left(\frac{1}{a}\right)_{a\in S}$ of the inverse of square free integers is not summable.

Let $S_N = \prod_{p<N}(1+1/p)$ , and $p_1,p_2,\ldots, p_l\ (l = l(N))$ all prime integers less than $N$. Then
\begin{align*}
S_N &= \left(1+\frac{1}{p_1}\right) \cdots \left(1+\frac{1}{p_l}\right)\\
&=\sum_{(\varepsilon_1,\cdots,\varepsilon_l) \in \{0,1\}^l } \frac{1}{p_1^{\varepsilon_1} \cdots p_l^{\varepsilon_l}}
\end{align*}
We prove this last formula  by induction. This is true for $l=1$ : $\sum_{\varepsilon \in \{0,1\}} 1/p_1^\varepsilon = 1 + 1/p_1$.

If it is true for the integer $l$, then 
\begin{align*}
\left(1+\frac{1}{p_1}\right) \cdots \left(1+\frac{1}{p_l}\right)\left(1+\frac{1}{p_{l+1}}\right) &= \sum_{(\varepsilon_1,\ldots,\varepsilon_l) \in \{0,1\}^l } \frac{1}{p_1^{\varepsilon_1} \cdots p_l^{\varepsilon_l}} \left(1+\frac{1}{p_{l+1}}\right)\\
&=\sum_{(\varepsilon_1,\ldots,\varepsilon_l) \in \{0,1\}^l } \frac{1}{p_1^{\varepsilon_1} \cdots p_l^{\varepsilon_l}} + \sum_{(\varepsilon_1,\ldots,\varepsilon_l) \in \{0,1\}^l } \frac{1}{p_1^{\varepsilon_1} \cdots p_l^{\varepsilon_l}p_{l+1}}\\
&=\sum_{(\varepsilon_1,\ldots,\varepsilon_l,\varepsilon_{l+1}) \in \{0,1\}^{l+1} } \frac{1}{p_1^{\varepsilon_1} \cdots p_l^{\varepsilon_l}p_{l+1}^{\varepsilon_{l+1}}} 
\end{align*}
So it is true for all $l$. 

Thus $S_N = \sum_{n\in \Delta} \frac{1}{n}$, where $\Delta$ is the set of square free integers whose prime factors are less than $N$.

As $\sum 1/n$, the sum being over square free integers, diverges, $\lim\limits_{N\to \infty} S_N = + \infty$ :
$$\lim_{N \to \infty} \prod_{p<N} \left(1+\frac{1}{p}\right) = +\infty.$$
 $e^x \geq 1+x, x \geq \log (1+x)$ for $x>0$, so
$$\log S_N = \sum_{k=1}^{l(N)} \log\left(1+\frac{1}{p_k}\right) \leq \sum_{k=1}^{l(N)} \frac{1}{p_k}.$$
$\lim\limits_{N\to \infty} \log S_N = +\infty$ and $\lim\limits_{N\to \infty} l(N) = +\infty$, so
$$\lim_{N\to \infty} \sum_{p<N} \frac{1}{p} = +\infty.$$
\end{proof}



\paragraph{Exercise 3.1} Show that there are infinitely many primes congruent to $-1$ modulo 6 .
\begin{proof}    
Let $n$ any integer such that $n\geq 3$, and $N = n! -1 =   2 \times 3 \times\cdots\times n - 1 >1$. 

Then $N \equiv -1 \pmod 6$. As $6k +2, 6k +3, 6k +4$ are composite for all integers $k$, every prime factor of $N$ is congruent to $1$ or $-1$ modulo $6$.  If every prime factor of $N$ was congruent to 1, then $N \equiv 1 \pmod 6$ : this is a contradiction because $-1 \not \equiv 1 \pmod 6$.  So there exists a prime factor $p$ of $N$ such that $p\equiv -1 \pmod 6$.

If $p\leq n$, then $p \mid n!$, and $p \mid N = n!-1$, so $p \mid 1$. As $p$ is prime, this is a contradiction, so $p>n$. 

Conclusion :

 for any integer $n$, there exists a prime $p >n$ such that $p \equiv -1 \pmod 6$ : there are infinitely many primes congruent to $-1$ modulo $6$.
\end{proof}



\paragraph{Exercise 3.4} Show that the equation $3 x^{2}+2=y^{2}$ has no solution in integers.
\begin{proof}    
If $3x^2+2 = y^2$, then  $\overline{y}^2 = \overline{2}$ in $\mathbb{Z}/3\mathbb{Z}$.


As $\{-1,0,1\}$ is a complete set of residues modulo $3$, the squares in $\mathbb{Z}/3\mathbb{Z}$ are $\overline{0} = \overline{0}^2$ and  $\overline{1} = \overline{1}^2 = (\overline{-1})^2$, so $\overline{2}$ is not a square in $\mathbb{Z}/3\mathbb{Z}$ : $\overline{y}^2 = \overline{2}$ is impossible in $\mathbb{Z}/3\mathbb{Z}$.

Thus $3x^2+2 = y^2$ has no solution in integers.
\end{proof}



\paragraph{Exercise 3.5} Show that the equation $7 x^{3}+2=y^{3}$ has no solution in integers.
\begin{proof}
    If $7x^2 + 2 = y^3,\ x,y \in \mathbb{Z}$, then $y^3 \equiv 2 \pmod 7$ (so $y \not \equiv 0 \pmod 7$)

From Fermat's Little Theorem, $y^6 \equiv 1 \pmod 7$, so $2^2 \equiv y^6 \equiv 1 \pmod 7$, which implies $7 \mid 2^2-1 = 3$ : this is a contradiction. Thus the equation $7x^2 + 2 = y^3$ has no solution in integers.
\end{proof}



\paragraph{Exercise 3.10} If $n$ is not a prime, show that $(n-1) ! \equiv 0(n)$, except when $n=4$.
\begin{proof}    
Suppose that $n >1$ is not a prime. Then $n = uv$, where $2 \leq u \leq v \leq n-1$.

$\bullet$ If $u \neq v$, then $n = uv \mid (n-1)! = 1\times 2 \times\cdots \times u \times\cdots \times v \times \cdots \times (n-1)$ (even if $u\wedge v \neq 1$ !).

$\bullet$ If $u=v$, $n = u^2$ is a square.

If $u$ is not prime, $u =st,\ 2\leq s \leq t \leq u-1 \leq n-1$, and $n = u' v'$, where $u' =s,v' =st^2$ verify  $2 \leq u' < v' \leq n-1$. As in the first case, $n = u'v' \mid (n-1)!$.  

If $u = p$ is a prime, then $n =p^2$.

In the case $p = 2$, $n = 4$ and $n=4  \nmid (n-1)! = 6$. In the other case $p >2$, and $(n-1)! = (p^2 - 1)!$ contains the factors $p < 2p < p^2$, so $p^2 \mid (p^2-1)!, n \mid (n-1)!$.

Conclusion : if $n$ is not a prime, $(n - 1)! \equiv 0 \pmod n$, except when $n=4$.
\end{proof}



\paragraph{Exercise 3.14} Let $p$ and $q$ be distinct odd primes such that $p-1$ divides $q-1$. If $(n, p q)=1$, show that $n^{q-1} \equiv 1(p q)$.
\begin{proof}    
As $n \wedge pq = 1, n\wedge p=1, n \wedge q = 1$, so from Fermat's Little Theorem
$$n^{q-1} \equiv 1 \pmod q,\qquad n^{p-1} \equiv 1 \pmod p.$$
$p-1 \mid q-1$, so there exists $k \in \mathbb{Z}$ such that $q-1 = k(p-1)$.
Thus
$$n^{q-1} = (n^{p-1})^k \equiv 1 \pmod p.$$
$p \mid n^{q-1} - 1, q \mid n^{q-1} - 1$, and $p\wedge q = 1$, so $pq \mid n^{q-1} - 1$ :
$$n^{q-1} \equiv 1 \pmod{pq}.$$
\end{proof}

\paragraph{Exercise 4.4} Consider a prime $p$ of the form $4 t+1$. Show that $a$ is a primitive root modulo $p$ iff $-a$ is a primitive root modulo $p$.
\begin{proof}
    Suppose that $a$ is a primitive root modulo $p$. As $p-1$ is even, $(-a)^{p-1}=a^{p-1} \equiv 1$ $(\bmod p)$
If $(-a)^n \equiv 1(\bmod p)$, with $n \in \mathbb{N}$, then $a^n \equiv(-1)^n(\bmod p)$.
Therefore $a^{2 n} \equiv 1(\bmod p)$. As $a$ is a primitive root modulo $p, p-1|2 n, 2 t| n$, so $n$ is even.

Hence $a^n \equiv 1(\bmod p)$, and $p-1 \mid n$. So the least $n \in \mathbb{N}^*$ such that $(-a)^n \equiv 1$ $(\bmod p)$ is $p-1:$ the order of $-a$ modulo $p$ is $p-1,-a$ is a primitive root modulo $p$. Conversely, if $-a$ is a primitive root modulo $p$, we apply the previous result at $-a$ to to obtain that $-(-a)=a$ is a primitive root.
\end{proof}



\paragraph{Exercise 4.5} Consider a prime $p$ of the form $4 t+3$. Show that $a$ is a primitive root modulo $p$ iff $-a$ has order $(p-1) / 2$.
\begin{proof}
    Let $a$ a primitive root modulo $p$.
As $a^{p-1} \equiv 1(\bmod p), p \mid\left(a^{(p-1) / 2}-1\right)\left(a^{(p-1) / 2}+1\right)$, so $p \mid a^{(p-1) / 2}-1$ or $p \mid$ $a^{(p-1) / 2}+1$. As $a$ is a primitive root modulo $p, a^{(p-1) / 2} \not \equiv 1(\bmod p)$, so
$$
a^{(p-1) / 2} \equiv-1 \quad(\bmod p) .
$$
Hence $(-a)^{(p-1) / 2}=(-1)^{2 t+1} a^{(p-1) / 2} \equiv(-1) \times(-1)=1(\bmod p)$.
Suppose that $(-a)^n \equiv 1(\bmod p)$, with $n \in \mathbb{N}$.
Then $a^{2 n}=(-a)^{2 n} \equiv 1(\bmod p)$, so $p-1\left|2 n, \frac{p-1}{2}\right| n$.
So $-a$ has order $(p-1) / 2$ modulo $p$.
Conversely, suppose that $-a$ has order $(p-1) / 2=2 t+1$ modulo $p$. Let $2, p_1, \ldots p_k$ the prime factors of $p-1$, where $p_i$ are odd.
$a^{(p-1) / 2}=a^{2 t+1}=-(-a)^{2 t+1}=-(-a)^{(p-1) / 2} \equiv-1$, so $a^{(p-1) / 2} \not \equiv 1(\bmod 2)$.
As $p-1$ is even, $(p-1) / p_i$ is even, so $a^{(p-1) / p_i}=(-a)^{(p-1) / p_i} \not \equiv 1(\bmod p)($ since $-a$ has order $p-1)$.
So the order of $a$ is $p-1$ (see Ex. 4.8) : $a$ is a primitive root modulo $p$.
\end{proof}



\paragraph{Exercise 4.6} If $p=2^{n}+1$ is a Fermat prime, show that 3 is a primitive root modulo $p$.
\begin{proof}
 \newcommand{\legendre}[2]{\genfrac{(}{)}{}{}{#1}{#2}}
Write $p = 2^k + 1$, with $k = 2^n$.

We suppose that $n>0$, so $k\geq 2, p \geq 5$. As $p$ is prime, $3^{p-1} \equiv 1 \pmod p$. 

In other words, $3^{2^k} \equiv 1 \pmod p$ : the order of $3$ is a divisor of $2^k$, a power of $2$.

$3$ has order $2^k$ modulo $p$ iff $3^{2^{k-1}} \not \equiv 1 \pmod p$. As $\left (3^{2^{k-1}} \right)^2 \equiv 1 \pmod p$, where $p$ is prime, this is equivalent to $3^{2^{k-1}}  \equiv -1 \pmod p$, which remains to prove.

$3^{2^{k-1}} = 3^{(p-1)/2} \equiv \legendre{3}{p} \pmod p$.

As the result is true for $p=5$, we can suppose $n\geq 2$.
From the law of quadratic reciprocity :
$$\legendre{3}{p} \legendre{p}{3} = (-1)^{(p-1)/2} = (-1)^{2^{k-1}} = 1.$$
So $\legendre{3}{p} = \legendre{p}{3}$
 
\begin{align*}
p = 2^{2^n}+1 &\equiv (-1)^{2^n} + 1 \pmod 3\\
&\equiv 2 \equiv -1 \pmod 3,
\end{align*}
so $\legendre{3}{p} = \legendre {p}{3} = -1$, that is to say
$$3^{2^{k-1}}  \equiv -1 \pmod p.$$
The order of $3$ modulo $p = 2^{2^n} + 1$ is $p-1 = 2^{2^n}$ : $3$ is a primitive root modulo $p$.

(On the other hand, if $3$ is of order $p-1$ modulo $p$, then $p$ is prime, so
$$ F_n = 2^{2^n} + 1 \ \mathrm{is}\ \mathrm{prime}\ \iff 3^{(F_n-1)/2} = 3^{2^{2^n - 1}} \equiv -1 \pmod {F_n}.)$$
\end{proof}



\paragraph{Exercise 4.8} Let $p$ be an odd prime. Show that $a$ is a primitive root modulo $p$ iff $a^{(p-1) / q} \not \equiv 1(p)$ for all prime divisors $q$ of $p-1$.
\begin{proof}    
$\bullet$ If $a$ is a primitive root, then $a^k \not \equiv 1$ for all $k, 1\leq k < p-1$, so $a^{(p-1)/q} \not \equiv 1 \pmod p$ for all prime divisors $q$ of $p - 1$.

$\bullet$ In the other direction, suppose $a^{(p-1)/q} \not \equiv 1 \pmod p$ for all prime divisors $q$ of $p - 1$.

Let $\delta$ the order of $a$, and $p-1 = q_1^{a_1}q_2^{a_2}\cdots q_k^{a_k}$ the decomposition of $p-1$ in prime factors. As $\delta \mid p-1, \delta = q_1^{b_1}p_2^{b_2}\cdots q_k^{b_k}$, with $b_i \leq a_i, i=1,2,\ldots,k$. If $b_i < a_i$ for some index $i$, then $\delta \mid (p-1)/q_i$, so $a^{(p-1)/q_i} \equiv 1 \pmod p$, which is in contradiction with the hypothesis. Thus $b_i = a_i$ for all $i$, and $\delta = q-1$ : $a$ is a primitive root modulo $p$.
\end{proof}



\paragraph{Exercise 4.11} Prove that $1^{k}+2^{k}+\cdots+(p-1)^{k} \equiv 0(p)$ if $p-1 \nmid k$ and $-1(p)$ if $p-1 \mid k$.
\begin{proof}    
Let $S_k = 1^k+2^k+\cdots+(p-1)^k$.

Let $g$ a primitive root modulo $p$ : $\overline{g}$ a generator of $\mathbb{F}_p^*$.

As $(\overline{1},\overline{g},\overline{g}^{2}, \ldots, \overline{g}^{p-2}) $ is a permutation of $ (\overline{1},\overline{2}, \ldots,\overline{p-1})$,
\begin{align*}
\overline{S_k} &= \overline{1}^k + \overline{2}^k+\cdots+ \overline{p-1}^k\\
&= \sum_{i=0}^{p-2} \overline{g}^{ki} =
\left\{
\begin{array}{ccc}
\overline{ p-1} = -\overline{1} &  \mathrm{if} &  p-1 \mid k  \\
 \frac{ \overline{g}^{(p-1)k} -1}{ \overline{g}^k -1} = \overline{0}&  \mathrm{if}  &   p-1 \nmid k
\end{array}
\right.
\end{align*}
since $p-1 \mid k \iff \overline{g}^k = \overline{1}$.

Conclusion :
\begin{align*}
1^k+2^k+\cdots+(p-1)^k&\equiv 0 \pmod p\ \mathrm{if} \ p-1 \nmid k\\
1^k+2^k+\cdots+(p-1)^k&\equiv -1 \pmod p\ \mathrm{if} \ p-1 \mid k\\
\end{align*}
\end{proof}



\paragraph{Exercise 5.13} Show that any prime divisor of $x^{4}-x^{2}+1$ is congruent to 1 modulo 12 .
\begin{proof}    
\newcommand{\legendre}[2]{\genfrac{(}{)}{}{}{#1}{#2}}
$\bullet$ As $a^6 +1 = (a^2+1)(a^4-a^2+1)$, $p\mid a^4 - a^2+1$ implies $p \mid a^6 + 1$, so $\legendre{-1}{p} = 1$ and $p\equiv 1 \pmod 4$.

$\bullet$ $p \mid 4a^4 - 4 a^2 +4 = (2a-1)^2 + 3$, so $\legendre{-3}{p} = 1$.

As $-3 \equiv 1 \pmod 4$, $\legendre{-3}{p} = \legendre{p}{3}$, so $\legendre{p}{3} = 1$, thus $p \equiv 1 \pmod 3$.

$4 \mid p-1$ and $3 \mid p-1$, thus $12 \mid p-1$ : $$p \equiv 1 \pmod {12}.$$
\end{proof}



\paragraph{Exercise 5.28} Show that $x^{4} \equiv 2(p)$ has a solution for $p \equiv 1(4)$ iff $p$ is of the form $A^{2}+64 B^{2}$.
\begin{proof}    
If  $p\equiv 1\ [4]$ and if there exists $x \in \mathbb{Z}$ such that $x^4 \equiv 2\ [p]$, then
$$2^{\frac{p-1}{4} }\equiv  x^{p-1} \equiv 1 \ [p].$$ 

From Ex. 5.27, where $p = a^2 +b^2, a$ odd,  we know that $$f^{\frac{ab}{2}} \equiv 2^{\frac{p-1}{4} } \equiv 1 \ [p].$$

Since $f^2 \equiv -1\ [p]$, the order of $f$ modulo $p$ is 4, thus $4 \mid \frac{ab}{2}$, so $8\mid ab$.

As $a$ is odd, $8 | b$, then $p = A^2 + 64 B^2$ (with $A = a, B = b/8$).

\bigskip

Conversely, if $p=A^2+64 B^2$, then $p\equiv A^2 \equiv 1 \ [4]$.

Let $a=A,b=8B$. Then $$2^{\frac{p-1}{4} } \equiv f^{\frac{ab}{2}} \equiv f^{4AB} \equiv (-1)^{2AB} \equiv 1 \ [p].$$

As $2^{\frac{p-1}{4} } \equiv 1 \ [p]$, $x^4 \equiv 2 \ [p]$ has a solution in $\mathbb{Z}$ (Prop. 4.2.1) : $2$ is a biquadratic residue modulo $p$.

Conclusion : 

$$\exists A \in \mathbb{Z}, \exists B \in \mathbb{Z}\,, p = A^2+64 B^2 \iff( p\equiv 1 \ [4] \ \mathrm{and} \ \exists x \in \mathbb{Z}, \, x^4 \equiv 2 \ [p]).$$
\end{proof}



\paragraph{Exercise 5.37} Show that if $a$ is negative then $p \equiv q(4 a) together with p\not | a$ imply $(a / p)=(a / q)$.
\begin{proof}    
\newcommand{\legendre}[2]{\genfrac{(}{)}{}{}{#1}{#2}}
Write $a = -A, A>0$. As $p \equiv q \pmod {4a}$, we know from Prop. 5.3.3. (b) that $(A/p) = (A/q)$.

Moreover,
\begin{align*}
\legendre{a}{p}&= \legendre{-A}{p} = (-1)^{(p-1)/2} \legendre{A}{p}\\
\legendre{a}{q}&= \legendre{-A}{q} = (-1^{(q-1)/2} \legendre{A}{q}
\end{align*}
As  $p \equiv q \pmod {4a}$, $ p = q + 4ak, k\in \mathbb{Z}$, so
$$(-1)^{(p-1)/2} = (-1)^{(q+4ak-1)/2} = (-1)^{(q-1)/2},$$
so $(a/p) = (a/q)$.
\end{proof}



\paragraph{Exercise 12.12} Show that $\sin (\pi / 12)$ is an algebraic number.
\begin{proof}
$$
\begin{aligned}
    \sin \pi/12=\sin \left(\pi/4-\pi/6\right) & =\sin \pi/4 \cos \pi/6-\cos \pi/4 \sin \pi/6 \\
& =\frac{\sqrt{3}}{2 \sqrt{2}}-\frac{1}{2 \sqrt{2}} \\
& =\frac{\sqrt{3}-1}{2 \sqrt{2}}
\end{aligned}
$$
\end{proof}

\paragraph{Exercise 18.4} Show that 1729 is the smallest positive integer expressible as the sum of two different integral cubes in two ways.
\begin{proof}
    Let $n=a^3+b^3$, and suppose that $\operatorname{gcd}(a, b)=1$. If a prime $p \mid a^3+b^3$, then
$$
\left(a b^{-1}\right)^3 \equiv_p-1
$$
Thus $3 \mid \frac{p-1}{2}$, that is, $p \equiv_6 1$.
If we have $n=a^3+b^3=c^3+d^3$, then we can factor $n$ as
$$
\begin{aligned}
& n=(a+b)\left(a^2-a b+b^2\right) \\
& n=(c+d)\left(c^2-c d+d^2\right)
\end{aligned}
$$
Thus we need $n$ to have atleast 3 disctinct prime factors, and so the smallest taxicab number is on the form
$$
n=(6 k+1)(12 k+1)(18 k+1)
$$
\end{proof}
\end{document}
