
\documentclass{article}

\title{\textbf{
Exercises from \\
\textit{A Classical Introduction to Modern Number Theory} \\
by Kenneth Ireland and Michael Rosen
}}

\date{}

\usepackage{amsmath}
\usepackage{amssymb}

\begin{document}
\maketitle


\paragraph{Exercise 1.27} For all odd $n$ show that $8 \mid n^{2}-1$.


\paragraph{Exercise 1.30} Prove that $\frac{1}{2}+\frac{1}{3}+\cdots+\frac{1}{n}$ is not an integer.


\paragraph{Exercise 1.31} Show that 2 is divisible by $(1+i)^{2}$ in $\mathbb{Z}[i]$.


\paragraph{Exercise 2.4} If $a$ is a nonzero integer, then for $n>m$ show that $\left(a^{2^{n}}+1, a^{2^{m}}+1\right)=1$ or 2 depending on whether $a$ is odd or even.


\paragraph{Exercise 2.21} Define $\wedge(n)=\log p$ if $n$ is a power of $p$ and zero otherwise. Prove that $\sum_{A \mid n} \mu(n / d) \log d$ $=\wedge(n)$.


\paragraph{Exercise 2.27a} Show that $\sum^{\prime} 1 / n$, the sum being over square free integers, diverges.


\paragraph{Exercise 3.1} Show that there are infinitely many primes congruent to $-1$ modulo 6 .


\paragraph{Exercise 3.4} Show that the equation $3 x^{2}+2=y^{2}$ has no solution in integers.


\paragraph{Exercise 3.5} Show that the equation $7 x^{3}+2=y^{3}$ has no solution in integers.


\paragraph{Exercise 3.10} If $n$ is not a prime, show that $(n-1) ! \equiv 0(n)$, except when $n=4$.


\paragraph{Exercise 3.14} Let $p$ and $q$ be distinct odd primes such that $p-1$ divides $q-1$. If $(n, p q)=1$, show that $n^{q-1} \equiv 1(p q)$.


\paragraph{Exercise 3.18} Let $N$ be the number of solutions to $f(x) \equiv 0(n)$ and $N_{i}$ be the number of solutions to $f(x) \equiv 0\left(p_{i}^{a_{i}}\right)$. Prove that $N=N_{1} N_{2} \cdots N_{i}$.


\paragraph{Exercise 4.4} Consider a prime $p$ of the form $4 t+1$. Show that $a$ is a primitive root modulo $p$ iff $-a$ is a primitive root modulo $p$.


\paragraph{Exercise 4.5} Consider a prime $p$ of the form $4 t+3$. Show that $a$ is a primitive root modulo $p$ iff $-a$ has order $(p-1) / 2$.


\paragraph{Exercise 4.6} If $p=2^{n}+1$ is a Fermat prime, show that 3 is a primitive root modulo $p$.


\paragraph{Exercise 4.8} Let $p$ be an odd prime. Show that $a$ is a primitive root modulo $p$ iff $a^{(p-1) / q} \not \equiv 1(p)$ for all prime divisors $q$ of $p-1$.


\paragraph{Exercise 4.11} Prove that $1^{k}+2^{k}+\cdots+(p-1)^{k} \equiv 0(p)$ if $p-1 \nmid k$ and $-1(p)$ if $p-1 \mid k$.


\paragraph{Exercise 5.13} Show that any prime divisor of $x^{4}-x^{2}+1$ is congruent to 1 modulo 12 .


\paragraph{Exercise 5.28} Show that $x^{4} \equiv 2(p)$ has a solution for $p \equiv 1(4)$ iff $p$ is of the form $A^{2}+64 B^{2}$.


\paragraph{Exercise 5.37} Show that if $a$ is negative then $p \equiv q(4 a) together with p\not | a$ imply $(a / p)=(a / q)$.


\paragraph{Exercise 12.12} Show that $\sin (\pi / 12)$ is an algebraic number.



\paragraph{Exercise 18.1} Show that $165 x^{2}-21 y^{2}=19$ has no integral solution.


\paragraph{Exercise 18.4} Show that 1729 is the smallest positive integer expressible as the sum of two different integral cubes in two ways.


\end{document}
