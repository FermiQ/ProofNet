
\documentclass{article}

\title{\textbf{
Exercises from \\
\textit{Putnam Competition} \\
}}

\date{}

\usepackage{amsmath}
\usepackage{amssymb}

\begin{document}
\maketitle


\paragraph{Exercise 2020.b5} For $j \in\{1,2,3,4\}$, let $z_{j}$ be a complex number with $\left|z_{j}\right|=1$ and $z_{j} \neq 1$. Prove that $3-z_{1}-z_{2}-z_{3}-z_{4}+z_{1} z_{2} z_{3} z_{4} \neq 0 .$


\paragraph{Exercise 2018.a5} Let $f: \mathbb{R} \rightarrow \mathbb{R}$ be an infinitely differentiable function satisfying $f(0)=0, f(1)=1$, and $f(x) \geq 0$ for all $x \in$ $\mathbb{R}$. Show that there exist a positive integer $n$ and a real number $x$ such that $f^{(n)}(x)<0$.


\paragraph{Exercise 2018.b2} Let $n$ be a positive integer, and let $f_{n}(z)=n+(n-1) z+$ $(n-2) z^{2}+\cdots+z^{n-1}$. Prove that $f_{n}$ has no roots in the closed unit disk $\{z \in \mathbb{C}:|z| \leq 1\}$.


\paragraph{Exercise 2018.b4} Given a real number $a$, we define a sequence by $x_{0}=1$, $x_{1}=x_{2}=a$, and $x_{n+1}=2 x_{n} x_{n-1}-x_{n-2}$ for $n \geq 2$. Prove that if $x_{n}=0$ for some $n$, then the sequence is periodic.


\paragraph{Exercise 2017.b3} Suppose that $f(x)=\sum_{i=0}^{\infty} c_{i} x^{i}$ is a power series for which each coefficient $c_{i}$ is 0 or 1 . Show that if $f(2 / 3)=3 / 2$, then $f(1 / 2)$ must be irrational.


\paragraph{Exercise 2014.a5} Let
$P_n(x)=1+2 x+3 x^2+\cdots+n x^{n-1} .$ Prove that the polynomials $P_j(x)$ and $P_k(x)$ are relatively prime for all positive integers $j$ and $k$ with $j \neq k$.


\paragraph{Exercise 2010.a4} Prove that for each positive integer $n$, the number $10^{10^{10^n}}+10^{10^n}+10^n-1$ is not prime.


\paragraph{Exercise 2001.a5} Prove that there are unique positive integers $a, n$ such that $a^{n+1}-(a+1)^n=2001$.


\paragraph{Exercise 2000.a2} Prove that there exist infinitely many integers $n$ such that $n, n+1, n+2$ are each the sum of the squares of two integers. 


\paragraph{Exercise 1999.b4} Let $f$ be a real function with a continuous third derivative such that $f(x), f^{\prime}(x), f^{\prime \prime}(x), f^{\prime \prime \prime}(x)$ are positive for all $x$. Suppose that $f^{\prime \prime \prime}(x) \leq f(x)$ for all $x$. Show that $f^{\prime}(x)<2 f(x)$ for all $x$.


\paragraph{Exercise 1998.a3} Let $f$ be a real function on the real line with continuous third derivative. Prove that there exists a point $a$ such that
$f(a) \cdot f^{\prime}(a) \cdot f^{\prime \prime}(a) \cdot f^{\prime \prime \prime}(a) \geq 0$. 


\paragraph{Exercise 1998.b6} Prove that, for any integers $a, b, c$, there exists a positive integer $n$ such that $\sqrt{n^3+a n^2+b n+c}$ is not an integer.



\end{document}
