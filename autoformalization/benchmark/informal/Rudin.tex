\documentclass{article}

\title{\textbf{
Exercises from \\
\textit{Principles of Mathematical Analysis} \\
by Walter Rudin
}}

\date{}

\usepackage{amsmath}
\usepackage{amssymb}

\begin{document}
\maketitle

\paragraph{Exercise 1.1} If $r$ is rational $(r \neq 0)$ and $x$ is irrational, prove that $r+x$ and $r x$ are irrational.

\paragraph{Exercise 1.2} Prove that there is no rational number whose square is $12$.

\paragraph{Exercise 1.5} Let $A$ be a nonempty set of real numbers which is bounded below. Let $-A$ be the set of all numbers $-x$, where $x \in A$. Prove that $\inf A=-\sup (-A)$.

\paragraph{Exercise 1.14} If $z$ is a complex number such that $|z|=1$, that is, such that $z \bar{z}=1$, compute $|1+z|^{2}+|1-z|^{2}$.

\paragraph{Exercise 1.18a} If $k \geq 2$ and $\mathbf{x} \in R^{k}$, prove that there exists $\mathbf{y} \in R^{k}$ such that $\mathbf{y} \neq 0$ but $\mathbf{x} \cdot \mathbf{y}=0$

\paragraph{Exercise 3.1a} Prove that convergence of $\left\{s_{n}\right\}$ implies convergence of $\left\{\left|s_{n}\right|\right\}$.

\paragraph{Exercise 3.3} If $s_{1}=\sqrt{2}$, and $s_{n+1}=\sqrt{2+\sqrt{s_{n}}} \quad(n=1,2,3, \ldots),$ prove that $\left\{s_{n}\right\}$ converges, and that $s_{n}<2$ for $n=1,2,3, \ldots$.

\paragraph{Exercise 3.5} For any two real sequences $\left\{a_{n}\right\},\left\{b_{n}\right\}$, prove that $\limsup _{n \rightarrow \infty}\left(a_{n}+b_{n}\right) \leq \limsup _{n \rightarrow \infty} a_{n}+\limsup _{n \rightarrow \infty} b_{n},$ provided the sum on the right is not of the form $\infty-\infty$.

\paragraph{Exercise 3.7} Prove that the convergence of $\Sigma a_{n}$ implies the convergence of $\sum \frac{\sqrt{a_{n}}}{n}$ if $a_n\geq 0$.

\paragraph{Exercise 3.8} If $\Sigma a_{n}$ converges, and if $\left\{b_{n}\right\}$ is monotonic and bounded, prove that $\Sigma a_{n} b_{n}$ converges.

\paragraph{Exercise 3.13} Prove that the Cauchy product of two absolutely convergent series converges absolutely.

\paragraph{Exercise 3.20} Suppose $\left\{p_{n}\right\}$ is a Cauchy sequence in a metric space $X$, and some sequence $\left\{p_{n l}\right\}$ converges to a point $p \in X$. Prove that the full sequence $\left\{p_{n}\right\}$ converges to $p$.

\paragraph{Exercise 3.21} If $\left\{E_{n}\right\}$ is a sequence of closed nonempty and bounded sets in a complete metric space $X$, if $E_{n} \supset E_{n+1}$, and if $\lim _{n \rightarrow \infty} \operatorname{diam} E_{n}=0,$ then $\bigcap_{1}^{\infty} E_{n}$ consists of exactly one point.

\paragraph{Exercise 3.22} Suppose $X$ is a nonempty complete metric space, and
$\left\{G_{n}\right\}$ is a sequence of dense open sets of $X$. Prove Baire's
theorem, namely, that $\bigcap_{1}^{\infty} G_{n}$ is not empty. Hint: Find a
shrinking sequence of neighborhoods $E_{n}$ such that $\overline{E_{n}} \subset G_{n}$.

\end{document}
