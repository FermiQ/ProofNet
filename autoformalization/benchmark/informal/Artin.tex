\documentclass{article}

\title{\textbf{
Exercises from \\
\textit{Algebra} \\
by Michael Artin
}}

\date{}

\usepackage{amsmath}
\usepackage{amssymb}

\begin{document}
\maketitle

\paragraph{Exercise 2.2.9} Let $H$ be the subgroup generated by two elements $a, b$ of a group $G$. Prove that if $a b=b a$, then $H$ is an abelian group.

\paragraph{Exercise 2.3.1} Prove that the additive group $\mathbb{R}^{+}$ of real numbers is isomorphic to the multiplicative group $P$ of positive reals.

\paragraph{Exercise 2.3.2} Prove that the products $a b$ and $b a$ are conjugate elements in a group.

\paragraph{Exercise 2.4.19} Prove that if a group contains exactly one element of order 2 , then that element is in the center of the group.

\paragraph{Exercise 2.8.6} Prove that the center of the product of two groups is the product of their centers.

\paragraph{Exercise 2.10.11} Prove that the groups $\mathbb{R}^{+} / \mathbb{Z}^{+}$and $\mathbb{R}^{+} / 2 \pi \mathbb{Z}^{+}$are isomorphic.

\paragraph{Exercise 2.11.3} Prove that a group of even order contains an element of order $2 .$

\paragraph{Exercise 3.2.7} Prove that every homomorphism of fields is injective. 

\paragraph{Exercise 3.5.6} Let $V$ be a vector space which is spanned by a countably infinite set. Prove that every linearly independent subset of $V$ is finite or countably infinite.

\paragraph{Exercise 3.7.2} Let $V$ be a vector space over an infinite field $F$. Prove that $V$ is not the union of finitely many proper subspaces.

\paragraph{Exercise 6.1.14} Let $Z$ be the center of a group $G$. Prove that if $G / Z$ is a cyclic group, then $G$ is abelian and hence $G=Z$. 

\paragraph{Exercise 6.4.2} Prove that no group of order $p q$, where $p$ and $q$ are prime, is simple.

\paragraph{Exercise 6.4.3} Prove that no group of order $p^2 q$, where $p$ and $q$ are prime, is simple.

\paragraph{Exercise 6.4.12} Prove that no group of order 224 is simple.

\paragraph{Exercise 6.8.1} Prove that two elements $a, b$ of a group generate the same subgroup as $b a b^2, b a b^3$.

\paragraph{Exercise 6.8.4} Prove that the group generated by $x, y, z$ with the single relation $y x y z^{-2}=1$ is actually a free group.

\paragraph{Exercise 6.8.6} Let $G$ be a group with a normal subgroup $N$. Assume that $G$ and $G / N$ are both cyclic groups. Prove that $G$ can be generated by two elements.

\paragraph{Exercise 10.1.13} An element $x$ of a ring $R$ is called nilpotent if some power of $x$ is zero. Prove that if $x$ is nilpotent, then $1+x$ is a unit in $R$.

\paragraph{Exercise 10.2.4} Prove that in the ring $\mathbb{Z}[x],(2) \cap(x)=(2 x)$.

\paragraph{Exercise 10.6.7} Prove that every nonzero ideal in the ring of Gauss integers contains a nonzero integer.

\paragraph{Exercise 10.6.16} Prove that a polynomial $f(x)=\sum a_i x^i$ can be expanded in powers of $x-a$ : $f(x)=\Sigma c_i(x-a)^i$, and that the coefficients $c_i$ are polynomials in the coefficients $a_i$, with integer coefficients.

\paragraph{Exercise 10.3.24a} Let $I, J$ be ideals of a ring $R$. Show by example that $I \cup J$ need not be an ideal. 

\paragraph{Exercise 10.4.6} Let $I, J$ be ideals in a ring $R$. Prove that the residue of any element of $I \cap J$ in $R / I J$ is nilpotent.

\paragraph{Exercise 10.4.7a} Let $I, J$ be ideals of a ring $R$ such that $I+J=R$. Prove that $I J=I \cap J$.

\paragraph{Exercise 10.5.16} Let $F$ be a field. Prove that the rings $F[x] /\left(x^2\right)$ and $F[x] /\left(x^2-1\right)$ are isomorphic if and only if $F$ has characteristic $2 .$

\paragraph{Exercise 10.7.6} Prove that the ring $\mathbb{F}_5[x] /\left(x^2+x+1\right)$ is a field.

\paragraph{Exercise 10.7.10} Let $R$ be a ring, with $M$ an ideal of $R$. Suppose that every element of $R$ which is not in $M$ is a unit of $R$. Prove that $M$ is a maximal ideal and that moreover it is the only maximal ideal of $R$.

\paragraph{Exercise 11.2.13} If $a, b$ are integers and if $a$ divides $b$ in the ring of Gauss integers, then $a$ divides $b$ in $\mathbb{Z}$.

\paragraph{Exercise 11.3.1} Let $a, b$ be elements of a field $F$, with $a \neq 0$. Prove that a polynomial $f(x) \in F[x]$ is irreducible if and only if $f(a x+b)$ is irreducible.

\paragraph{Exercise 11.3.2} Let $F=\mathbb{C}(x)$, and let $f, g \in \mathbb{C}[x, y]$. Prove that if $f$ and $g$ have a common factor in $F[y]$, then they also have a common factor in $\mathbb{C}[x, y]$.

\paragraph{Exercise 11.3.4} Prove that two integer polynomials are relatively prime in $\mathbb{Q}[x]$ if and only if the ideal they generate in $\mathbb{Z}[x]$ contains an integer.

\paragraph{Exercise 11.4.1a} Prove that $x^2 + 27x + 213$ is irreducible in $\mathbb{Q}$. 

\paragraph{Exercise 11.4.1b} Prove that $x^3 + 6x + 12$ is irreducible in $\mathbb{Q}$. 

\paragraph{Exercise 11.4.6a} Prove that $x^2+x+1$ is irreducible in the field $\mathbb{F}_2$. 

\paragraph{Exercise 11.4.6b} Prove that $x^2+1$ is irreducible in $\mathbb{F}_7$

\paragraph{Exercise 11.4.6c} Prove that $x^3 - 9$ is irreducible in $\mathbb{F}_31$. 

\paragraph{Exercise 11.4.8} Let $p$ be a prime integer. Prove that the polynomial $x^n-p$ is irreducible in $\mathbb{Q}[x]$.

\paragraph{Exercise 11.4.10} Let $p$ be a prime integer, and let $f \in \mathbb{Z}[x]$ be a polynomial of degree $2 n+1$, say $f(x)=a_{2 n+1} x^{2 n+1}+\cdots+a_1 x+a_0$. Suppose that $a_{2 n+1} \neq 0$ (modulo $p$ ), $a_0, a_1, \ldots, a_n \equiv 0$ (modulo $p^2$ ), $a_{n+1}, \ldots, a_{2 n} \equiv 0$ (modulo $p$ ), $a_0 \not\equiv 0$ (modulo $p^3$ ). Prove that $f$ is irreducible in $\mathbb{Q}[x]$. 

\paragraph{Exercise 11.9.4} Let $p$ be a prime which splits in $R$, say $(p)=P \bar{P}$, and let $\alpha \in P$ be any element which is not divisible by $p$. Prove that $P$ is generated as an ideal by $(p, \alpha)$.

\paragraph{Exercise 11.12.3} Prove that if $x^2 \equiv-5$ (modulo $p$ ) has a solution, then there is an integer point on one of the two ellipses $x^2+5 y^2=p$ or $2 x^2+2 x y+3 y^2=p$. 

\paragraph{Exercise 11.13.3} Prove that there are infinitely many primes congruent to $-1$ (modulo 4 ).

\paragraph{Exercise 13.1.3} Let $R$ be an integral domain containing a field $F$ as subring and which is finite-dimensional when viewed as vector space over $F$. Prove that $R$ is a field.

\paragraph{Exercise 13.3.1} Let $F$ be a field, and let $\alpha$ be an element which generates a field extension of $F$ of degree 5. Prove that $\alpha^2$ generates the same extension.

\paragraph{Exercise 13.3.8} Let $K$ be a field generated over $F$ by two elements $\alpha, \beta$ of relatively prime degrees $m, n$ respectively. Prove that $[K: F]=m n$.

\paragraph{Exercise 13.4.10} Prove that if a prime integer $p$ has the form $2^r+1$, then it actually has the form $2^{2^k}+1$. 

\paragraph{Exercise 13.6.10} Let $K$ be a finite field. Prove that the product of the nonzero elements of $K$ is $-1$.
\end{document}
