
\documentclass{article}

\title{\textbf{
Exercises from \\
\textit{Topology} \\
by James Munkres
}}

\date{}

\usepackage{amsmath}
\usepackage{amssymb}

\begin{document}
\maketitle

\paragraph{exercise\_13\_1}
Let $X$ be a topological space; let $A$ be a subset of $X$. Suppose that for
each $x \in A$
there is an open set $U$ containing $x$ such that $U \subset A$. Show that $A$
is open in $X$.

\paragraph{exercise\_13\_5a} Show that if $\mathcal{A}$ is a basis for a topology on $X$, then the topology generated by $\mathcal{A}$ equals the intersection of all topologies on $X$ that contain $\mathcal{A}$.

\paragraph{exercise\_16\_4} A map $f: X \rightarrow Y$ is said to be an open map if for every open set $U$ of $X$, the set $f(U)$ is open in $Y$. Show that $\pi_{1}: X \times Y \rightarrow X$ and $\pi_{2}: X \times Y \rightarrow Y$ are open maps.

\paragraph{exercise\_17\_2} Show that if $A$ is closed in $Y$ and $Y$ is closed in $X$, then $A$ is closed in $X$.

\paragraph{exercise\_17\_3} Show that if $A$ is closed in $X$ and $B$ is closed in $Y$, then $A \times B$ is closed in $X \times Y$.

\paragraph{exercise\_17\_4} Show that if $U$ is open in $X$ and $A$ is closed in $X$, then $U-A$ is open in $X$, and $A-U$ is closed in $X$.

\paragraph{exercise\_18\_8a} Let $Y$ be an ordered set in the order topology. Let $f, g: X \rightarrow Y$ be continuous. Show that the set $\{x \mid f(x) \leq g(x)\}$ is closed in $X$.

\paragraph{exercise\_18\_8b} Let $Y$ be an ordered set in the order topology. Let $f, g: X \rightarrow Y$ be continuous. Let $h: X \rightarrow Y$ be the function $h(x)=\min \{f(x), g(x)\}.$ Show that $h$ is continuous. [Hint: Use the pasting lemma.]

\paragraph{exercise\_18\_13} Let $A \subset X$; let $f: A \rightarrow Y$ be continuous; let $Y$ be Hausdorff. Show that if $f$ may be extended to a continuous function $g: \bar{A} \rightarrow Y$, then $g$ is uniquely determined by $f$.

\paragraph{exercise\_21\_6a} Define $f_{n}:[0,1] \rightarrow \mathbb{R}$ by the equation $f_{n}(x)=x^{n}$. Show that the sequence $\left(f_{n}(x)\right)$ converges for each $x \in[0,1]$.

\paragraph{exercise\_21\_6b} Define $f_{n}:[0,1] \rightarrow \mathbb{R}$ by the equation $f_{n}(x)=x^{n}$. Show that the sequence $\left(f_{n}\right)$ does not converge uniformly.

\paragraph{exercise\_21\_8} Let $X$ be a topological space and let $Y$ be a metric space. Let $f_{n}: X \rightarrow Y$ be a sequence of continuous functions. Let $x_{n}$ be a sequence of points of $X$ converging to $x$. Show that if the sequence $\left(f_{n}\right)$ converges uniformly to $f$, then $\left(f_{n}\left(x_{n}\right)\right)$ converges to $f(x)$.

\paragraph{exercise\_22\_1} Let $p: X \rightarrow Y$ be a continuous map. Show that if there is a continuous map $f: Y \rightarrow X$ such that $p \circ f$ equals the identity map of $Y$, then $p$ is a quotient map.

\paragraph{exercise\_22\_2a} Let $p: X \rightarrow Y$ be a continuous map. Show that if there is a continuous map $f: Y \rightarrow X$ such that $p \circ f$ equals the identity map of $Y$, then $p$ is a quotient map.

\paragraph{exercise\_22\_2b} If $A \subset X$, a retraction of $X$ onto $A$ is a continuous map $r: X \rightarrow A$ such that $r(a)=a$ for each $a \in A$. Show that a retraction is a quotient map.

\paragraph{exercise\_22\_3} Let $H$ be a subspace of $G$. Show that if $H$ is also a subgroup of $G$, then both $H$ and $\bar{H}$ are topological groups.

\paragraph{exercise\_23\_2} Let $\left\{A_{n}\right\}$ be a sequence of connected subspaces of $X$, such that $A_{n} \cap A_{n+1} \neq \varnothing$ for all $n$. Show that $\bigcup A_{n}$ is connected.

\paragraph{exercise\_23\_3} Let $\left\{A_{\alpha}\right\}$ be a collection of connected subspaces of $X$; let $A$ be a connectea eubsen of $X$ Show that if $A \cap A_{\alpha} \neq \varnothing$ for all $\alpha$, then $A \cup\left(\bigcup \mid A_{\alpha}\right)$ is connected.

\paragraph{exercise\_23\_4} Show that if $X$ is an infinite set, it is connected in the finite complement topology.

\paragraph{exercise\_23\_6} Let $A \subset X$. Show that if $C$ is a connected subspace of $X$ that intersects both $A$ and $X-A$, then $C$ intersects $\mathrm{Bd} A$.

\paragraph{exercise\_23\_9} Let $A$ be a proper subset of $X$, and let $B$ be a proper subset of $Y$. If $X$ and $Y$ are connected, show that $(X \times Y)-(A \times B)$ is connected.

\paragraph{exercise\_23\_11} Let $p: X \rightarrow Y$ be a quotient map. Show that if each set $p^{-1}(\{y\})$ is connected, and if $Y$ is connected, then $X$ is connected.

\paragraph{exercise\_23\_12} Let $Y \subset X$; let $X$ and $Y$ be connected. Show that if $A$ and $B$ form a separation of $X-Y$, then $Y \cup A$ and $Y \cup B$ are connected.

\paragraph{exercise\_24\_2} Let $f: S^{1} \rightarrow \mathbb{R}$ be a continuous map. Show there exists a point $x$ of $S^{1}$ such that $f(x)=f(-x)$.

\paragraph{exercise\_25\_9} Let $G$ be a topological group; let $C$ be the component of $G$ containing the identity element $e$. Show that $C$ is a normal subgroup of $G$.

\paragraph{exercise\_26\_9} Theorem. Let $A$ and $B$ be subspaces of $X$ and $Y$, respectively; let $N$ be an open set in $X \times Y$ containing $A \times B$. If $A$ and $B$ are compact, then there exist open sets $U$ and $V$ in $X$ and $Y$, respectively, such that $A \times B \subset U \times V \subset N .$

\paragraph{exercise\_26\_11} Theorem. Let $X$ be a compact Hausdorff space. Let A be a collection of closed connected subsets of $X$ that is simply ordered by proper inclusion. Then $Y=\bigcap_{A \in \mathcal{A}} A$ is connected.

\paragraph{exercise\_26\_12} Let $p: X \rightarrow Y$ be a closed continuous surjective map such that $p^{-1}(\{y\})$ is compact, for each $y \in Y$. (Such a map is called a perfect map.) Show that if $Y$ is compact, then $X$ is compact.

\paragraph{exercise\_27\_4} Show that a connected metric space having more than one point is uncountable.

\paragraph{exercise\_28\_6} Let $(X, d)$ be a metric space. If $f: X \rightarrow X$ satisfies the condition $d(f(x), f(y))=d(x, y)$ for all $x, y \in X$, then $f$ is called an isometry of $X$. Show that if $f$ is an isometry and $X$ is compact, then $f$ is bijective and hence a homeomorphism.

\paragraph{exercise\_29\_10} Show that if $X$ is a Hausdorff space that is locally compact at the point $x$, then for each neighborhood $U$ of $x$, there is a neighborhood $V$ of $x$ such that $\bar{V}$ is compact and $\bar{V} \subset U$.

\end{document}
