
\documentclass{article}

\title{\textbf{
Exercises from \\
\textit{A Classical Introduction to Modern Number Theory} \\
by Kenneth Ireland and Michael Rosen
}}

\date{}

\usepackage{amsmath}
\usepackage{amssymb}

\begin{document}
\maketitle

\paragraph{exercise\_1\_27} For all odd $n$ show that $8 \mid n^{2}-1$.

\paragraph{exercise\_1\_30} Prove that $\frac{1}{2}+\frac{1}{3}+\cdots+\frac{1}{n}$ is not an integer.

\paragraph{exercise\_1\_31} Show that 2 is divisible by $(1+i)^{2}$ in $\mathbb{Z}[i]$.

\paragraph{exercise\_2\_4} If $a$ is a nonzero integer, then for $n>m$ show that $\left(a^{2^{n}}+1, a^{2^{m}}+1\right)=1$ or 2 depending on whether $a$ is odd or even.

\paragraph{exercise\_2\_21} Define $\wedge(n)=\log p$ if $n$ is a power of $p$ and zero otherwise. Prove that $\sum_{A \mid n} \mu(n / d) \log d$ $=\wedge(n)$.

\paragraph{exercise\_2\_27a} Show that $\sum^{\prime} 1 / n$, the sum being over square free integers, diverges.

\paragraph{exercise\_3\_1} Show that there are infinitely many primes congruent to $-1$ modulo 6 .

\paragraph{exercise\_3\_4} Show that the equation $3 x^{2}+2=y^{2}$ has no solution in integers.

\paragraph{exercise\_3\_5} Show that the equation $7 x^{3}+2=y^{3}$ has no solution in integers.

\paragraph{exercise\_3\_10} If $n$ is not a prime, show that $(n-1) ! \equiv 0(n)$, except when $n=4$.

\paragraph{exercise\_3\_14} Let $p$ and $q$ be distinct odd primes such that $p-1$ divides $q-1$. If $(n, p q)=1$, show that $n^{q-1} \equiv 1(p q)$.

\paragraph{exercise\_3\_18} Let $N$ be the number of solutions to $f(x) \equiv 0(n)$ and $N_{i}$ be the number of solutions to $f(x) \equiv 0\left(p_{i}^{a_{i}}\right)$. Prove that $N=N_{1} N_{2} \cdots N_{i}$.

\paragraph{exercise\_3\_20} Show that $x^{2} \equiv 1\left(2^{b}\right)$ has one solution if $b=1$, two solutions if $b=2$, and four solutions if $b \geq 3$.

\paragraph{exercise\_4\_4} Consider a prime $p$ of the form $4 t+1$. Show that $a$ is a primitive root modulo $p$ iff - $a$ is a primitive root modulo $p$.

\paragraph{exercise\_4\_5} Consider a prime $p$ of the form $4 t+3$. Show that $a$ is a primitive root modulo $p$ iff $-a$ has order $(p-1) / 2$.

\paragraph{exercise\_4\_6} If $p=2^{n}+1$ is a Fermat prime, show that 3 is a primitive root modulo $p$.

\paragraph{exercise\_4\_8} Let $p$ be an odd prime. Show that $a$ is a primitive root module $p$ iff $a^{(p-1) / q} \not \equiv 1(p)$ for all prime divisors $q$ of $p-1$.

\paragraph{exercise\_4\_9} Show that the product of all the primitive roots modulo $p$ is congruent to $(-1)^{\phi(p-1)}$ modulo $p$.

\paragraph{exercise\_4\_10} Show that the sum of all the primitive roots modulo $p$ is congruent to $\mu(p-1)$ modulo $p$.

\paragraph{exercise\_4\_11} Prove that $1^{k}+2^{k}+\cdots+(p-1)^{k} \equiv 0(p)$ if $p-1 \nmid k$ and $-1(p)$ if $p-1 \mid k$.

\paragraph{exercise\_4\_22} If $a$ has order 3 modulo $p$, show that $1+a$ has order 6 .

\paragraph{exercise\_4\_24} Show that $a x^{m}+b y^{n} \equiv c(p)$ has the same number of solutions as $a x^{m^{\prime}}+b y^{n^{\prime}} \equiv c(p)$, where $m^{\prime}=(m, p-1)$ and $n^{\prime}=(n, p-1)$.

\paragraph{exercise\_5\_2} Show that the number of solutions to $x^{2} \equiv a(p)$ is given by $1+(a / p)$.

\paragraph{exercise\_5\_3} Suppose that $p \nmid a$. Show that the number of solutions to $a x^{2}+b x+c \equiv 0(p)$ is given by $1+\left(\left(b^{2}-4 a c\right) / p\right)$.

\paragraph{exercise\_5\_4} Prove that $\sum_{a=1}^{p-1}(a / p)=0$.

\paragraph{exercise\_5\_5} Prove that $\sum_{\substack{p-1 \\ x=0}}((a x+b) / p)=0$ provided that $p \nmid a .$

\paragraph{exercise\_5\_6} Show that the number of solutions to $x^{2}-y^{2} \equiv a(p)$ is given by $\sum_{y=0}^{p-1}\left(1+\left(\left(y^{2}+a\right) / p\right)\right) .$

\paragraph{exercise\_5\_7} By calculating directly show that the number of solutions to $x^{2}-y^{2} \equiv a(p)$ is $p-1$ if $p \nmid a$ and $2 p-1$ if $p \mid a$.

\paragraph{exercise\_5\_13} Show that any prime divisor of $x^{4}-x^{2}+1$ is congruent to 1 modulo 12 .

\paragraph{exercise\_5\_27} Suppose that $f$ is such that $b \equiv a f(p)$. Show that $f^{2} \equiv-1(p)$ and that $2^{(p-1) / 4} \equiv$ $f^{a b / 2}(p)$

\paragraph{exercise\_5\_28} Show that $x^{4} \equiv 2(p)$ has a solution for $p \equiv 1(4)$ iff $p$ is of the form $A^{2}+64 B^{2}$.

\paragraph{exercise\_5\_37} Show that if $a$ is negative then $p \equiv q(4 a), p \times a$ implies $(a / p)=(a / q)$.

\paragraph{exercise\_6\_18} Show that there exist algebraic numbers of arbitrarily high degree.

\paragraph{exercise\_7\_6} Let $K \supset F$ be finite fields with $[K: F]=3$. Show that if $\alpha \in F$ is not a square in $F$, it is not a square in $K$.

\paragraph{exercise\_7\_24} Suppose that $f(x) \in \mathbb{Z} / p \mathbb{Z}[x]$ has the property that $f(x+y)=f(x)+$ $f(y) \in \mathbb{Z} / p \mathbb{Z}[x, y]$. Show that $f(x)$ must be of the form $a_{0} x+a_{1} x^{p}+a_{2} x^{p^{2}}+$ $\cdots+a_{m} x^{p^{m}}$.

\paragraph{exercise\_12\_12} Show that $\sin (\pi / 12)$ is an algebraic number.

\paragraph{exercise\_12\_19} Show that a finite integral domain is a field.

\paragraph{exercise\_12\_22} Let $F \subset E$ be algebraic number fields. Show that any isomorphism of $F$ into $\mathbb{C}$ extends in exactly $[E: F]$ ways to an isomorphism of $E$ into $\mathbb{C}$.

\paragraph{exercise\_12\_30} Let $p$ be an odd prime and consider $\mathbb{Q}(\sqrt{p})$. If $q \neq p$ is prime show that $\sigma_{q}(\sqrt{p})=$ $(p / q) \sqrt{p}$ where $\sigma_{q}$ is the Frobenius automorphism at a prime ideal in $\mathbb{Q}(\sqrt{p})$ lying above $q$.

\paragraph{exercise\_18\_1} Show that $165 x^{2}-21 y^{2}=19$ has no integral solution.

\paragraph{exercise\_18\_4} Show that 1729 is the smallest positive integer expressible as the sum of two different integral cubes in two ways.

\paragraph{exercise\_18\_32} Let $d$ be a square-free integer $d \equiv 1$ or 2 modulo 4 . Show that if $x$ and $y$ are integers such that $y^{2}=x^{3}-d$ then $(x, 2 d)=1$.

\end{document}
