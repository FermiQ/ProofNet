\documentclass{article}

\title{\textbf{
Exercises from \\
\textit{Abstract Algebra} \\
by David Dummit and Richard Foote
}}

\date{}

\usepackage{amsmath}
\usepackage{amssymb}

\begin{document}
\maketitle

\paragraph{exercise\_1\_1\_2a} Prove the the operation $\star$ on $\mathbb{Z}$ defined by $a\star b=a-b$ is not commutative.

\paragraph{exercise\_1\_1\_3} Prove that the addition of residue classes $\mathbb{Z}/n\mathbb{Z}$ is associative.

\paragraph{exercise\_1\_1\_4} Prove that the multiplication of residue class $\mathbb{Z}/n\mathbb{Z}$ is associative.

\paragraph{exercise\_1\_1\_5} Prove that for all $n>1$ that $\mathbb{Z}/n\mathbb{Z}$ is not a group under multiplication of residue classes.

\paragraph{exercise\_1\_1\_15} Prove that $(a_1a_2\dots a_n)^{-1} = a_n^{-1}a_{n-1}^{-1}\dots a_1^{-1}$ for all $a_1, a_2, \dots, a_n\in G$.

\paragraph{exercise\_1\_1\_16} Let $x$ be an element of $G$. Prove that $x^2=1$ if and only if $|x|$ is either $1$ or $2$.

\paragraph{exercise\_1\_1\_17} Let $x$ be an element of $G$. Prove that if $|x|=n$ for some positive integer $n$ then $x^{-1}=x^{n-1}$.

\paragraph{exercise\_1\_1\_18} Let $x$ and $y$ be elements of $G$. Prove that $xy=yx$ if and only if $y^{-1}xy=x$ if and only if $x^{-1}y^{-1}xy=1$.

\paragraph{exercise\_1\_1\_20} For $x$ an element in $G$ show that $x$ and $x^{-1}$ have the same order.

\paragraph{exercise\_1\_1\_22a} If $x$ and $g$ are elements of the group $G$, prove that $|x|=\left|g^{-1} x g\right|$.

\paragraph{exercise\_1\_1\_22b} Deduce that $|a b|=|b a|$ for all $a, b \in G$.

\paragraph{exercise\_1\_1\_25} Prove that if $x^{2}=1$ for all $x \in G$ then $G$ is abelian.

\paragraph{exercise\_1\_1\_29} Prove that $A \times B$ is an abelian group if and only if both $A$ and $B$ are abelian.

\paragraph{exercise\_1\_1\_34} If $x$ is an element of infinite order in $G$, prove that the elements $x^{n}, n \in \mathbb{Z}$ are all distinct.

\paragraph{exercise\_1\_3\_8} Prove that if $\Omega=\{1,2,3, \ldots\}$ then $S_{\Omega}$ is an infinite group

\paragraph{exercise\_1\_6\_4} Prove that the multiplicative groups $\mathbb{R}-\{0\}$ and $\mathbb{C}-\{0\}$ are not isomorphic.

\paragraph{exercise\_1\_6\_11} Let $A$ and $B$ be groups. Prove that $A \times B \cong B \times A$.

\paragraph{exercise\_1\_6\_17} Let $G$ be any group. Prove that the map from $G$ to itself defined by $g \mapsto g^{-1}$ is a homomorphism if and only if $G$ is abelian.

\paragraph{exercise\_1\_6\_23} Let $G$ be a finite group which possesses an automorphism $\sigma$ such that $\sigma(g)=g$ if and only if $g=1$. If $\sigma^{2}$ is the identity map from $G$ to $G$, prove that $G$ is abelian.

\paragraph{exercise\_1\_7\_5} Prove that the kernel of an action of the group $G$ on a set $A$ is the same as the kernel of the corresponding permutation representation $G\to S_A$.

\paragraph{exercise\_1\_7\_6} Prove that a group $G$ acts faithfully on a set $A$ if and only if the kernel of the action is the set consisting only of the identity.

\paragraph{exercise\_2\_1\_5} Prove that $G$ cannot have a subgroup $H$ with $|H|=n-1$, where $n=|G|>2$.

\paragraph{exercise\_2\_1\_13} Let $H$ be a subgroup of the additive group of rational numbers with the property that $1 / x \in H$ for every nonzero element $x$ of $H$. Prove that $H=0$ or $\mathbb{Q}$.

\paragraph{exercise\_2\_4\_4} Prove that if $H$ is a subgroup of $G$ then $H$ is generated by the set $H-\{1\}$.

\paragraph{exercise\_2\_4\_13} Prove that the multiplicative group of positive rational numbers is generated by the set $\left\{\frac{1}{p} \mid \text{$p$ is a prime} \right\}$.

\paragraph{exercise\_2\_4\_16a} A subgroup $M$ of a group $G$ is called a maximal subgroup if $M \neq G$ and the only subgroups of $G$ which contain $M$ are $M$ and $G$. Prove that if $H$ is a proper subgroup of the finite group $G$ then there is a maximal subgroup of $G$ containing $H$.

\paragraph{exercise\_2\_4\_16b} Show that the subgroup of all rotations in a dihedral group is a maximal subgroup.

\paragraph{exercise\_2\_4\_16c} Show that if $G=\langle x\rangle$ is a cyclic group of order $n \geq 1$ then a subgroup $H$ is maximal if and only $H=\left\langle x^{p}\right\rangle$ for some prime $p$ dividing $n$.

\paragraph{exercise\_3\_1\_3a} Let $A$ be an abelian group and let $B$ be a subgroup of $A$. Prove that $A / B$ is abelian.

\paragraph{exercise\_3\_1\_22a} Prove that if $H$ and $K$ are normal subgroups of a group $G$ then their intersection $H \cap K$ is also a normal subgroup of $G$.

\paragraph{exercise\_3\_1\_22b} Prove that the intersection of an arbitrary nonempty collection of normal subgroups of a group is a normal subgroup (do not assume the collection is countable).

\paragraph{exercise\_3\_2\_8} Prove that if $H$ and $K$ are finite subgroups of $G$ whose orders are relatively prime then $H \cap K=1$.

\paragraph{exercise\_3\_2\_11} Let $H \leq K \leq G$. Prove that $|G: H|=|G: K| \cdot|K: H|$ (do not assume $G$ is finite).

\paragraph{exercise\_3\_2\_16} Use Lagrange's Theorem in the multiplicative group $(\mathbb{Z} / p \mathbb{Z})^{\times}$to prove Fermat's Little Theorem: if $p$ is a prime then $a^{p} \equiv a(\bmod p)$ for all $a \in \mathbb{Z}$.

\paragraph{exercise\_3\_2\_21a} Prove that $\mathbb{Q}$ has no proper subgroups of finite index.

\paragraph{exercise\_3\_3\_3} Prove that if $H$ is a normal subgroup of $G$ of prime index $p$ then for all $K \leq G$ either $K \leq H$, or $G=H K$ and $|K: K \cap H|=p$.

\paragraph{exercise\_3\_4\_1} Prove that if $G$ is an abelian simple group then $G \cong Z_{p}$ for some prime $p$ (do not assume $G$ is a finite group).

\paragraph{exercise\_3\_4\_4} Use Cauchy's Theorem and induction to show that a finite abelian group has a subgroup of order $n$ for each positive divisor $n$ of its order.

\paragraph{exercise\_3\_4\_5a} Prove that subgroups of a solvable group are solvable.

\paragraph{exercise\_3\_4\_5b} Prove that quotient groups of a solvable group are solvable.

\paragraph{exercise\_3\_4\_11} Prove that if $H$ is a nontrivial normal subgroup of the solvable group $G$ then there is a nontrivial subgroup $A$ of $H$ with $A \unlhd G$ and $A$ abelian.

\paragraph{exercise\_4\_2\_8} Prove that if $H$ has finite index $n$ then there is a normal subgroup $K$ of $G$ with $K \leq H$ and $|G: K| \leq n!$.

\paragraph{exercise\_4\_2\_9a} Prove that if $p$ is a prime and $G$ is a group of order $p^{\alpha}$ for some $\alpha \in \mathbb{Z}^{+}$, then every subgroup of index $p$ is normal in $G$.

\paragraph{exercise\_4\_2\_14} Let $G$ be a finite group of composite order $n$ with the property that $G$ has a subgroup of order $k$ for each positive integer $k$ dividing $n$. Prove that $G$ is not simple.

\paragraph{exercise\_4\_3\_5} If the center of $G$ is of index $n$, prove that every conjugacy class has at most $n$ elements.

\paragraph{exercise\_4\_3\_26} Let $G$ be a transitive permutation group on the finite set $A$ with $|A|>1$. Show that there is some $\sigma \in G$ such that $\sigma(a) \neq a$ for all $a \in A$.

\paragraph{exercise\_4\_3\_27} Let $g_{1}, g_{2}, \ldots, g_{r}$ be representatives of the conjugacy classes of the finite group $G$ and assume these elements pairwise commute. Prove that $G$ is abelian.

\paragraph{exercise\_4\_4\_2} Prove that if $G$ is an abelian group of order $p q$, where $p$ and $q$ are distinct primes, then $G$ is cyclic.

\paragraph{exercise\_4\_4\_6a} Prove that characteristic subgroups are normal.

\paragraph{exercise\_4\_4\_6b} Prove that there exists a normal subgroup that is not characteristic.

\paragraph{exercise\_4\_4\_7} If $H$ is the unique subgroup of a given order in a group $G$ prove $H$ is characteristic in $G$.

\paragraph{exercise\_4\_4\_8a} Let $G$ be a group with subgroups $H$ and $K$ with $H \leq K$. Prove that if $H$ is characteristic in $K$ and $K$ is normal in $G$ then $H$ is normal in $G$.

\paragraph{exercise\_4\_5\_1a} Prove that if $P \in \operatorname{Syl}_{p}(G)$ and $H$ is a subgroup of $G$ containing $P$ then $P \in \operatorname{Syl}_{p}(H)$.

\paragraph{exercise\_4\_5\_13} Prove that a group of order 56 has a normal Sylow $p$-subgroup for some prime $p$ dividing its order.

\paragraph{exercise\_4\_5\_14} Prove that a group of order 312 has a normal Sylow $p$-subgroup for some prime $p$ dividing its order.

\paragraph{exercise\_4\_5\_15} Prove that a group of order 351 has a normal Sylow $p$-subgroup for some prime $p$ dividing its order.

\paragraph{exercise\_4\_5\_16} Let $|G|=p q r$, where $p, q$ and $r$ are primes with $p<q<r$. Prove that $G$ has a normal Sylow subgroup for either $p, q$ or $r$.

\paragraph{exercise\_4\_5\_17} Prove that if $|G|=105$ then $G$ has a normal Sylow 5 -subgroup and a normal Sylow 7-subgroup.

\paragraph{exercise\_4\_5\_18} Prove that a group of order 200 has a normal Sylow 5-subgroup.

\paragraph{exercise\_4\_5\_19} Prove that if $|G|=6545$ then $G$ is not simple.

\paragraph{exercise\_4\_5\_20} Prove that if $|G|=1365$ then $G$ is not simple.

\paragraph{exercise\_4\_5\_21} Prove that if $|G|=2907$ then $G$ is not simple.

\paragraph{exercise\_4\_5\_22} Prove that if $|G|=132$ then $G$ is not simple.

\paragraph{exercise\_4\_5\_23} Prove that if $|G|=462$ then $G$ is not simple.

\paragraph{exercise\_4\_5\_28} Let $G$ be a group of order 105. Prove that if a Sylow 3-subgroup of $G$ is normal then $G$ is abelian.

\paragraph{exercise\_4\_5\_33} Let $P$ be a normal Sylow $p$-subgroup of $G$ and let $H$ be any subgroup of $G$. Prove that $P \cap H$ is the unique Sylow $p$-subgroup of $H$.

\paragraph{exercise\_5\_4\_2} Prove that a subgroup $H$ of $G$ is normal if and only if $[G, H] \leq H$.

\paragraph{exercise\_7\_1\_2} Prove that if $u$ is a unit in $R$ then so is $-u$.

\paragraph{exercise\_7\_1\_11} Prove that if $R$ is an integral domain and $x^{2}=1$ for some $x \in R$ then $x=\pm 1$.

\paragraph{exercise\_7\_1\_12} Prove that any subring of a field which contains the identity is an integral domain.

\paragraph{exercise\_7\_1\_15} A ring $R$ is called a Boolean ring if $a^{2}=a$ for all $a \in R$. Prove that every Boolean ring is commutative.

\paragraph{exercise\_7\_2\_2} Let $p(x)=a_{n} x^{n}+a_{n-1} x^{n-1}+\cdots+a_{1} x+a_{0}$ be an element of the polynomial ring $R[x]$. Prove that $p(x)$ is a zero divisor in $R[x]$ if and only if there is a nonzero $b \in R$ such that $b p(x)=0$.

\paragraph{exercise\_7\_2\_4} Prove that if $R$ is an integral domain then the ring of formal power series $R[[x]]$ is also an integral domain.

\paragraph{exercise\_7\_2\_12} Let $G=\left\{g_{1}, \ldots, g_{n}\right\}$ be a finite group. Prove that the element $N=g_{1}+g_{2}+\ldots+g_{n}$ is in the center of the group ring $R G$.

\paragraph{exercise\_7\_3\_16} Let $\varphi: R \rightarrow S$ be a surjective homomorphism of rings. Prove that the image of the center of $R$ is contained in the center of $S$.

\paragraph{exercise\_7\_3\_28} Prove that an integral domain has characteristic $p$, where $p$ is either a prime or 0.

\paragraph{exercise\_7\_3\_37} An ideal $N$ is called nilpotent if $N^{n}$ is the zero ideal for some $n \geq 1$. Prove that the ideal $p \mathbb{Z} / p^{m} \mathbb{Z}$ is a nilpotent ideal in the ring $\mathbb{Z} / p^{m} \mathbb{Z}$.

\paragraph{exercise\_7\_4\_27} Let $R$ be a commutative ring with $1 \neq 0$. Prove that if $a$ is a nilpotent element of $R$ then $1-a b$ is a unit for all $b \in R$.

\paragraph{exercise\_8\_1\_12} Let $N$ be a positive integer. Let $M$ be an integer relatively prime to $N$ and let $d$ be an integer relatively prime to $\varphi(N)$, where $\varphi$ denotes Euler's $\varphi$-function. Prove that if $M_{1} \equiv M^{d} \pmod N$ then $M \equiv M_{1}^{d^{\prime}} \pmod N$ where $d^{\prime}$ is the inverse of $d \bmod \varphi(N)$: $d d^{\prime} \equiv 1 \pmod {\varphi(N)}$.

\paragraph{exercise\_8\_2\_4} Let $R$ be an integral domain. Prove that if the following two conditions hold then $R$ is a Principal Ideal Domain: (i) any two nonzero elements $a$ and $b$ in $R$ have a greatest common divisor which can be written in the form $r a+s b$ for some $r, s \in R$, and (ii) if $a_{1}, a_{2}, a_{3}, \ldots$ are nonzero elements of $R$ such that $a_{i+1} \mid a_{i}$ for all $i$, then there is a positive integer $N$ such that $a_{n}$ is a unit times $a_{N}$ for all $n \geq N$.

\paragraph{exercise\_8\_3\_4} Prove that if an integer is the sum of two rational squares, then it is the sum of two integer squares.

\paragraph{exercise\_8\_3\_5a} Let $R=\mathbb{Z}[\sqrt{-n}]$ where $n$ is a squarefree integer greater than 3. Prove that $2, \sqrt{-n}$ and $1+\sqrt{-n}$ are irreducibles in $R$.

\paragraph{exercise\_8\_3\_6a} Prove that the quotient ring $\mathbb{Z}[i] /(1+i)$ is a field of order 2.

\paragraph{exercise\_8\_3\_6b} Let $q \in \mathbb{Z}$ be a prime with $q \equiv 3 \bmod 4$. Prove that the quotient ring $\mathbb{Z}[i] /(q)$ is a field with $q^{2}$ elements.

\paragraph{exercise\_9\_1\_6} Prove that $(x, y)$ is not a principal ideal in $\mathbb{Q}[x, y]$.

\paragraph{exercise\_9\_1\_10} Prove that the ring $\mathbb{Z}\left[x_{1}, x_{2}, x_{3}, \ldots\right] /\left(x_{1} x_{2}, x_{3} x_{4}, x_{5} x_{6}, \ldots\right)$ contains infinitely many minimal prime ideals (cf. exercise\_9\_1\_36 of Section 7\_4).

\paragraph{exercise\_9\_3\_2} Prove that if $f(x)$ and $g(x)$ are polynomials with rational coefficients whose product $f(x) g(x)$ has integer coefficients, then the product of any coefficient of $g(x)$ with any coefficient of $f(x)$ is an integer.

\paragraph{exercise\_9\_4\_2a} Prove that $x^4-4x^3+6$ is irreducible in $\mathbb{Z}[x]$.

\paragraph{exercise\_9\_4\_2b} Prove that $x^6+30x^5-15x^+6x-120$ is irreducible in $\mathbb{Z}[x]$.

\paragraph{exercise\_9\_4\_2c} Prove that $x^4+4x^3+6x^2+2x+1$ is irreducible in $\mathbb{Z}[x]$.

\paragraph{exercise\_9\_4\_2d} Prove that $\frac{(x+2)^p-2^p}{x}$, where $p$ is an odd prime, is irreducible in $\mathbb{Z}[x]$.

\paragraph{exercise\_9\_4\_9} Prove that the polynomial $x^{2}-\sqrt{2}$ is irreducible over $\mathbb{Z}[\sqrt{2}]$. You may assume that $\mathbb{Z}[\sqrt{2}]$ is a U.F.D.

\paragraph{exercise\_9\_4\_11} Prove that $x^2+y^2-1$ is irreducible in $\mathbb{Q}[x,y]$.

\end{document}
