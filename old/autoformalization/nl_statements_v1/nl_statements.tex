\documentclass{article}

\title{ProofNet NL Statements}
\author{Zhangir Azerbayev}
\date{Summer 2022}

\usepackage{amsmath}
\usepackage{amssymb}
\usepackage{parskip}

\begin{document}
\maketitle
\section*{df}
\subsection*{chapter 1}
\subsubsection*{section 1}
15: Prove that $(a_1a_2\dots a_n)^{-1} = a_n^{-1}a_{n-1}^{-1}\dots a_1^{-1}$ for all $a_1, a_2, \dots, a_n\in G$. 

16: Let $x$ be an element of $G$. Prove that $x^2=1$ if and only if $|x|$ is either $1$ or $2$. 

17: Let $x$ be an element of $G$. Prove that if $|x|=n$ for some positive integer $n$ then $x^{-1}=x^{n-1}$. 

18: Let $x$ and $y$ be elements of $G$. Prove that $xy=yx$ if and only if $y^{-1}xy=x$ if and only if $x^{-1}y^{-1}xy=1$.

20: For $x$ an element in $G$ show that $x$ and $x^{-1}$ have the same order.

22a: If $x$ and $g$ are elements of the group $G$, prove that $|x|=\left|g^{-1} x g\right|$. 

22b: Deduce that $|a b|=|b a|$ for all $a, b \in G$.

25: Prove that if $x^{2}=1$ for all $x \in G$ then $G$ is abelian.

29: Prove that $A \times B$ is an abelian group if and only if both $A$ and $B$ are abelian.

34: If $x$ is an element of infinite order in $G$, prove that the elements $x^{n}, n \in \mathbb{Z}$ are all distinct.

\subsubsection*{section 3}
8: Prove that if $\Omega=\{1,2,3, \ldots\}$ then $S_{\Omega}$ is an infinite group

\subsubsection*{section 6}
4: Prove that the multiplicative groups $\mathbb{R}-\{0\}$ and $\mathbb{C}-\{0\}$ are not isomorphic.

11: Let $A$ and $B$ be groups. Prove that $A \times B \cong B \times A$.

17: Let $G$ be any group. Prove that the map from $G$ to itself defined by $g \mapsto g^{-1}$ is a homomorphism if and only if $G$ is abelian.

23: Let $G$ be a finite group which possesses an automorphism $\sigma$ such that $\sigma(g)=g$ if and only if $g=1$. If $\sigma^{2}$ is the identity map from $G$ to $G$, prove that $G$ is abelian. 

\subsubsection*{section 7}
5: Prove that the kernel of an action of the group $G$ on a set $A$ is the same as the kernel of the corresponding permutation representation $G\to S_A$. 

6: Prove that a group $G$ acts faithfully on a set $A$ if and only if the kernel of the action is the set consisting only of the identity. 

\subsection*{chapter 2}
\subsubsection*{section 1}
5: Prove that $G$ cannot have a subgroup $H$ with $|H|=n-1$, where $n=|G|>2$.

13: Let $H$ be a subgroup of the additive group of rational numbers with the property that $1 / x \in H$ for every nonzero element $x$ of $H$. Prove that $H=0$ or $\mathbb{Q}$.

\subsubsection*{section 2}
4: Prove that if $H$ is a subgroup of $G$ then $H$ is generated by the set $H-\{1\}$.

13: Prove that the multiplicative group of positive rational numbers is generated by the set $\left\{\frac{1}{p} \mid \text{$p$ is a prime} \right\}$. 

16a: A subgroup $M$ of a group $G$ is called a maximal subgroup if $M \neq G$ and the only subgroups of $G$ which contain $M$ are $M$ and $G$. Prove that if $H$ is a proper subgroup of the finite group $G$ then there is a maximal subgroup of $G$ containing $H$.

16c: Show that if $G=\langle x\rangle$ is a cyclic group of order $n \geq 1$ then a subgroup $H$ is maximal if and only $H=\left\langle x^{p}\right\rangle$ for some prime $p$ dividing $n$.

\subsection*{Chapter 3}
\subsubsection*{section 1}
3a: Let $A$ be an abelian group and let $B$ be a subgroup of $A$. Prove that $A / B$ is abelian.

22a: Prove that if $H$ and $K$ are normal subgroups of a group $G$ then their intersection $H \cap K$ is also a normal subgroup of $G$.

22b: Prove that the intersection of an arbitrary nonempty collection of normal subgroups of a group is a normal subgroup (do not assume the collection is countable).

\subsubsection*{section 2}
8: Prove that if $H$ and $K$ are finite subgroups of $G$ whose orders are relatively prime then $H \cap K=1$.

11: Let $H \leq K \leq G$. Prove that $|G: H|=|G: K| \cdot|K: H|$ (do not assume $G$ is finite).

16: Use Lagrange's Theorem in the multiplicative group $(\mathbb{Z} / p \mathbb{Z})^{\times}$to prove Fermat's Little Theorem: if $p$ is a prime then $a^{p} \equiv a(\bmod p)$ for all $a \in \mathbb{Z}$.

21a: Prove that $\mathbb{Q}$ has no proper subgroups of finite index.

\subsubsection*{section 3}
3: Prove that if $H$ is a normal subgroup of $G$ of prime index $p$ then for all $K \leq G$ either $K \leq H$ or $G=H K$ and $|K: K \cap H|=p$.

\section*{Rudin}
\subsection*{chapter 1}
1: If $r$ is rational $(r \neq 0)$ and $x$ is irrational, prove that $r+x$ and $r x$ are irrational.

2: Prove that there is no rational number whose square is $12$. 

5: Let $A$ be a nonempty set of real numbers which is bounded below. Let $-A$ be the set of all numbers $-x$, where $x \in A$. Prove that $\inf A=-\sup (-A)$

14: If $z$ is a complex number such that $|z|=1$, that is, such that $z \bar{z}=1$, compute $|1+z|^{2}+|1-z|^{2}$. 

18a: If $k \geq 2$ and $\mathbf{x} \in R^{k}$, prove that there exists $\mathbf{y} \in R^{k}$ such that $\mathbf{y} \neq 0$ but $\mathbf{x} \cdot \mathbf{y}=0$

\subsection*{chapter 3}
1a: Prove that convergence of $\left\{s_{n}\right\}$ implies convergence of $\left\{\left|s_{n}\right|\right\}$. 

3: If $s_{1}=\sqrt{2}$, and $s_{n+1}=\sqrt{2+\sqrt{s_{n}}} \quad(n=1,2,3, \ldots),$ prove that $\left\{s_{n}\right\}$ converges, and that $s_{n}<2$ for $n=1,2,3, \ldots$.

5: For any two real sequences $\left\{a_{n}\right\},\left\{b_{n}\right\}$, prove that $\limsup _{n \rightarrow \infty}\left(a_{n}+b_{n}\right) \leq \limsup _{n \rightarrow \infty} a_{n}+\limsup _{n \rightarrow \infty} b_{n},$ provided the sum on the right is not of the form $\infty-\infty$.

7: Prove that the convergence of $\Sigma a_{n}$ implies the convergence of $\sum \frac{\sqrt{a_{n}}}{n}$ if $a_n\geq 0$.

8: If $\Sigma a_{n}$ converges, and if $\left\{b_{n}\right\}$ is monotonic and bounded, prove that $\Sigma a_{n} b_{n}$ converges.

13: Prove that the Cauchy product of two absolutely convergent series converges absolutely.

20: 20. Suppose $\left\{p_{n}\right\}$ is a Cauchy sequence in a metric space $X$, and some subsequence $\left\{p_{n l}\right\}$ converges to a point $p \in X$. Prove that the full sequence $\left\{p_{n}\right\}$ converges to $p$.

21: If $\left\{E_{n}\right\}$ is a sequence of closed nonempty and bounded sets in a complete metric space $X$, if $E_{n} \supset E_{n+1}$, and if $\lim _{n \rightarrow \infty} \operatorname{diam} E_{n}=0,$ then $\bigcap_{1}^{\infty} E_{n}$ consists of exactly one point.

22: Suppose $X$ is a nonempty complete metric space, and $\left\{G_{n}\right\}$ is a sequence of dense open subsets of $X$. Prove Baire's theorem, namely, that $\bigcap_{1}^{\infty} G_{n}$ is not empty. Hint: Find a shrinking sequence of neighborhoods $E_{n}$ such that $E_{n} \subset G_{n}$. 

\section*{Munkres}
\subsection*{chapter 2}
\subsubsection*{section 13}
5a: Show that if $\mathcal{A}$ is a basis for a topology on $X$, then the topology generated by $\mathcal{A}$ equals the intersection of all topologies on $X$ that contain $\mathcal{A}$. 

\subsubsection*{section 16}
4: A map $f: X \rightarrow Y$ is said to be an open map if for every open set $U$ of $X$, the set $f(U)$ is open in $Y$. Show that $\pi_{1}: X \times Y \rightarrow X$ and $\pi_{2}: X \times Y \rightarrow Y$ are open maps.

\subsubsection*{section 17}
2: Show that if $A$ is closed in $Y$ and $Y$ is closed in $X$, then $A$ is closed in $X$.

3: Show that if $A$ is closed in $X$ and $B$ is closed in $Y$, then $A \times B$ is closed in $X \times Y$.

4: Show that if $U$ is open in $X$ and $A$ is closed in $X$, then $U-A$ is open in $X$, and $A-U$ is closed in $X$.

\subsubsection*{18}
8a: Let $Y$ be an ordered set in the order topology. Let $f, g: X \rightarrow Y$ be continuous. Show that the set $\{x \mid f(x) \leq g(x)\}$ is closed in $X$.

8b: Let $Y$ be an ordered set in the order topology. Let $f, g: X \rightarrow Y$ be continuous. Let $h: X \rightarrow Y$ be the function $h(x)=\min \{f(x), g(x)\}.$ Show that $h$ is continuous. [Hint: Use the pasting lemma.] 

13: Let $A \subset X$; let $f: A \rightarrow Y$ be continuous; let $Y$ be Hausdorff. Show that if $f$ may be extended to a continuous function $g: \bar{A} \rightarrow Y$, then $g$ is uniquely determined by $f$.

\subsubsection*{21}

6a: Define $f_{n}:[0,1] \rightarrow \mathbb{R}$ by the equation $f_{n}(x)=x^{n}$. Show that the sequence $\left(f_{n}(x)\right)$ converges for each $x \in[0,1]$. 

6b: Define $f_{n}:[0,1] \rightarrow \mathbb{R}$ by the equation $f_{n}(x)=x^{n}$. Show that the sequence $\left(f_{n}\right)$ does not converge uniformly.

8: Let $X$ be a topological space and let $Y$ be a metric space. Let $f_{n}: X \rightarrow Y$ be a sequence of continuous functions. Let $x_{n}$ be a sequence of points of $X$ converging to $x$. Show that if the sequence $\left(f_{n}\right)$ converges uniformly to $f$, then $\left(f_{n}\left(x_{n}\right)\right)$ converges to $f(x)$.

\subsubsection*{22}
1: Let $p: X \rightarrow Y$ be a continuous map. Show that if there is a continuous map $f: Y \rightarrow X$ such that $p \circ f$ equals the identity map of $Y$, then $p$ is a quotient map.

2a: Let $p: X \rightarrow Y$ be a continuous map. Show that if there is a continuous map $f: Y \rightarrow X$ such that $p \circ f$ equals the identity map of $Y$, then $p$ is a quotient map.

2b: If $A \subset X$, a retraction of $X$ onto $A$ is a continuous map $r: X \rightarrow A$ such that $r(a)=a$ for each $a \in A$. Show that a retraction is a quotient map.

3: Let $H$ be a subspace of $G$. Show that if $H$ is also a subgroup of $G$, then both $H$ and $\bar{H}$ are topological groups.
\end{document}
