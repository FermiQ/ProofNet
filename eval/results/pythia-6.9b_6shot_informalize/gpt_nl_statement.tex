\documentclass{article}

\title{\textbf{
Exercises from \\
\textit{Everything} \\
by All Authors
}}

\date{}

\usepackage{amsmath}
\usepackage{amssymb}
\usepackage{fullpage}

\begin{document}
\maketitle

\paragraph{Artin.exercise.2.3.2} If $G$ is a group, and $a, b \in G$, then there exists $g \in G$ such that $b*a = g*a*b*g^{-1}$.

\paragraph{Artin.exercise.2.8.6} If $G$ and $H$ are groups, then the center of $G \times H$ is isomorphic to the direct product of the centers of $G$ and $H$.

\paragraph{Artin.exercise.3.2.7} If $F$ is a field and $G$ is a field extension of $F$, then the mapping $φ : F →+* G$ is injective if and only if $φ$ is one-to-one.

\paragraph{Artin.exercise.3.7.2} If $V$ is an $n$-dimensional vector space over a field $K$, then the set of all subspaces of $V$ is not a singleton.

\paragraph{Artin.exercise.6.4.2} If $G$ is a finite group, then $G$ is not simple.

\paragraph{Artin.exercise.6.4.12} If $G$ is a finite group, then $G$ is not simple.

\paragraph{Artin.exercise.10.1.13} If $x$ is nilpotent, then $1 + x$ is a unit.

\paragraph{Artin.exercise.10.4.7a} If $I$ and $J$ are ideals of a commutative ring $R$, then $I + J = I \cap J$.

\paragraph{Artin.exercise.10.6.7} If $I$ is an ideal of $gaussian_int$, then there exists $z \in I$ such that $z \neq 0$ and $z$ is a Gaussian integer.

\paragraph{Artin.exercise.11.2.13} If $a$ and $b$ are integers, then $a$ divides $b$ if and only if $a$ is a Gaussian integer.

\paragraph{Artin.exercise.11.4.6a} If $F$ is a field of characteristic $0$, and $X^2 + 1$ is irreducible over $F$, then $F$ is a Galois extension of $F(X)$.

\paragraph{Artin.exercise.11.4.6c} Prove that the polynomial $X^3 - 9$ is irreducible over $\mathbb{Z}/31\mathbb{Z}$.

\paragraph{Artin.exercise.11.13.3} There exists a prime $p$ such that $p+1$ is congruent to $0$ modulo $4$. (Hint: Use the fact that $p+1$ is congruent to $0$ modulo $4$ if and only if $p$ is congruent to $1$ modulo $4$.)

\paragraph{Artin.exercise.13.6.10} If $K$ is a field, then for every $x \in K$, $x = -1$.

\paragraph{Axler.exercise.1.2} The cube root of $-1$ is $-1$.

\paragraph{Axler.exercise.1.4} If $v$ is an element of $V$ such that $a$ is a nonzero element of $F$, then $a$ is a zero element of $F$ or $v$ is a zero element of $V$.

\paragraph{Axler.exercise.1.7} There exists a set $U$ such that $U \neq \emptyset$ and for every $c \in \mathbb{R}$ and $u \in U$, $c \cdot u \in U$. Moreover, there exists a set $U'$ such that $U \neq \uparrow U'$.

\paragraph{Axler.exercise.1.9} If $U$ and $W$ are submodules of $V$, prove that there exists a submodule $U'$ of $V$ such that $U \subseteq U'$ and $U' \subseteq W$ or $U \subseteq W$ and $W \subseteq U'$.

\paragraph{Axler.exercise.3.8} If $V$ is a vector space over a field $F$, and $L: V \to W$ is a linear map, then there exists a submodule $U$ of $V$ such that $U \cap \ker L = \{0\}$ and $L$ is the restriction of a linear map from $U$ to $W$.

\paragraph{Axler.exercise.5.1} If $L$ is a linear map from $V$ into $F$, then the sum of the images of the $U_i$ is equal to the sum of the images of the $U_i$. (Hint: use the fact that $L$ is a linear map.)

\paragraph{Axler.exercise.5.11} If $V$ is a finite-dimensional vector space over a field $F$, then the eigenvalues of $S$ and $T$ are the same.

\paragraph{Axler.exercise.5.13} If $F$ is a field and $V$ is a finite-dimensional vector space over $F$, then there exists a nonzero scalar $c$ such that $T = c \cdot id$.

\paragraph{Axler.exercise.5.24} If $V$ is an $n$-dimensional vector space over $\mathbb{R}$, then the set of all eigenvectors of $T$ is a subspace of $V$ of dimension $n$.

\paragraph{Axler.exercise.6.3} If $a$ and $b$ are elements of a finite set, then $\sum_{i=1}^n a_i^2$ is less than or equal to $\sum_{i=1}^n a_i^2$ times $\sum_{i=1}^n b_i^2 / i$.

\paragraph{Axler.exercise.6.13} If $V$ is an inner product space, and $e$ is an orthonormal basis for $V$, then $e$ is a basis for the vector space $V$.

\paragraph{Axler.exercise.7.5} If $V$ is an $n$-dimensional inner product space, then there exists a subspace $U$ of $V$ such that $U$ is not the direct sum of $n$ copies of the same subspace.

\paragraph{Axler.exercise.7.9} If $V$ is an inner product space, then $T$ is self-adjoint if and only if the image of $T$ is orthogonal to its own eigenspaces.

\paragraph{Axler.exercise.7.11} If $V$ is an inner product space, then there exists a symmetric endomorphism $S$ of $V$ such that $S^2 = T$.

\paragraph{Dummit-Foote.exercise.1.1.2a} There exists an integer $a$ such that $a - b \neq b - a$ for all integers $b$.

\paragraph{Dummit-Foote.exercise.1.1.4} If $n$ is a positive integer, then $n$ is a multiple of $1$.

\paragraph{Dummit-Foote.exercise.1.1.15} If $G$ is a group, then $a_1a_2\cdots a_n = a_na_{n-1}\cdots a_1$ for all $a_1, \dots, a_n \in G$. (Hint: Use the fact that $G$ is a group and the fact that $a_1a_2\cdots a_n = a_na_{n-1}\cdots a_1$ for all $a_1, \dots, a_n \in G$ to show that $a_1a_2\cdots a_n = a_na_{n-1}\cdots a_1$ for all $a_1, \dots, a_n \in G$ and then use the fact that $a_1a_2\cdots a_n = a_na_{n-1}\cdots a_1$ for all $a_1, \dots, a_n \in G$ to show that $a_1a_2\cdots a_n = a_na_{n-1}\cdots a_1$ for all $a_1, \dots, a_n \in G$.)

\paragraph{Dummit-Foote.exercise.1.1.17} If $x$ has order $n$, then $x^{-1} = x^{n-1}$.

\paragraph{Dummit-Foote.exercise.1.1.20} If $x$ is an element of infinite order in $G$, prove that the order of $x$ is equal to the order of $x^{-1}$.

\paragraph{Dummit-Foote.exercise.1.1.22b} If $a$ and $b$ are elements of a group $G$, then the order of $a * b$ is equal to the order of $b * a$. (Hint: Use the fact that $a * b = b * a$ if and only if $a$ and $b$ commute.)

\paragraph{Dummit-Foote.exercise.1.1.29} If $A$ and $B$ are groups, and $x$ and $y$ are elements of $A$ and $B$, respectively, then $x*y = y*x$ if and only if $x*y = y*x$ for all $x$ and $y$ in $A$ and $B$, respectively.

\paragraph{Dummit-Foote.exercise.1.3.8} If $n$ is an infinite natural number, then $n$ is a permutation of $\mathbb{N}$.

\paragraph{Dummit-Foote.exercise.1.6.11} If $A$ and $B$ are groups, then $A \times B \cong B \times A$.

\paragraph{Dummit-Foote.exercise.1.6.23} If $G$ is a group, and $\sigma$ is an automorphism of $G$, then $\sigma$ is the identity map.

\paragraph{Dummit-Foote.exercise.2.1.13} If $x$ is a rational number, then $x$ is either a positive integer or a negative integer.

\paragraph{Dummit-Foote.exercise.2.4.16a} If $H$ is a subgroup of $G$ that is not the trivial subgroup, then there exists a subgroup $M$ of $G$ such that $M$ is not the trivial subgroup and $H \leq M$. Moreover, $M$ is the unique subgroup of $G$ that is not the trivial subgroup and is such that $H \leq M$.

\paragraph{Dummit-Foote.exercise.2.4.16c} If $n$ is a positive integer, and $H$ is a subgroup of $add_subgroup (zmod n)$, then $H$ is the closure of $H$ under multiplication by $n$ if and only if $H$ is the trivial subgroup of $add_subgroup (zmod n)$.

\paragraph{Dummit-Foote.exercise.3.1.22a} If $H$ and $K$ are subgroups of $G$, then $H \cap K$ is a subgroup of $H \cap K$.

\paragraph{Dummit-Foote.exercise.3.2.8} If $H$ and $K$ are finite groups, then $H \oplus K = \bot$.

\paragraph{Dummit-Foote.exercise.3.2.16} If $p$ is a prime number, then $a^p \equiv a \pmod{p}$ for all $a \in \mathbb{Z}$.

\paragraph{Dummit-Foote.exercise.3.3.3} If $H$ is a $p$-subgroup of $G$, then $H$ is normal or $H$ is a subgroup of $G$ and $H$ is a $p$-Sylow subgroup of $G$.

\paragraph{Dummit-Foote.exercise.3.4.4} If $G$ is a finite group, then there exists a subgroup $H$ of $G$ such that $H$ has order $n$ and $H$ has finite order.

\paragraph{Dummit-Foote.exercise.3.4.5b} If $H$ is a subgroup of $G$, then $G$ is solvable if and only if $G$ is solvable as a subgroup of $G \times H$.

\paragraph{Dummit-Foote.exercise.4.2.8} If $H$ is a subgroup of $G$ with index $n$, then there exists a subgroup $K$ of $G$ with index $n$ and $K \leq H$. Moreover, $K$ is normal in $G$ and $K \leq H$.

\paragraph{Dummit-Foote.exercise.4.2.9a} If $G$ is a finite group, and $p$ is a prime, then for every subgroup $H$ of $G$, the index of $H$ in $G$ is congruent to $p$ modulo $p$. 

\paragraph{Dummit-Foote.exercise.4.4.2} If $G$ is a finite group, then $G$ is cyclic if and only if $G$ is isomorphic to a subgroup of $\mathbb{Z}_p\times\mathbb{Z}_q$ for some primes $p$ and $q$.

\paragraph{Dummit-Foote.exercise.4.4.6b} If $G$ is a group, then there exists a subgroup $H$ of $G$ such that $H$ is not normal.

\paragraph{Dummit-Foote.exercise.4.4.8a} If $H$ is a subgroup of $K$, then $H$ is normal in $K$.

\paragraph{Dummit-Foote.exercise.4.5.13} There is a Sylow $p$-subgroup of $G$ for some prime $p$.

\paragraph{Dummit-Foote.exercise.4.5.15} There is a Sylow $p$-subgroup of $G$ for every prime $p$.

\paragraph{Dummit-Foote.exercise.4.5.17} If $G$ is a finite group, then there are two Sylow $5$-subgroups and two Sylow $7$-subgroups.

\paragraph{Dummit-Foote.exercise.4.5.19} If $G$ is a finite group, then $G$ is not simple.

\paragraph{Dummit-Foote.exercise.4.5.21} If $G$ is a finite group, then $G$ is not simple.

\paragraph{Dummit-Foote.exercise.4.5.23} If $G$ is a finite group, then $G$ is not simple.

\paragraph{Dummit-Foote.exercise.4.5.33} If $H$ is a $p$-Sylow subgroup of $G$, then $H$ is a subgroup of $G$ and $H$ is a subgroup of $G$ that is a $p$-Sylow subgroup of $G$. Moreover, if $H$ is nonempty, then $H$ is a subgroup of $G$ that is a $p$-Sylow subgroup of $G$.

\paragraph{Dummit-Foote.exercise.7.1.2} If $u$ is a unit in a ring $R$, then $-u$ is a unit in $R$. (Hint: Use the fact that $u$ is a unit if and only if $-u$ is a unit.)

\paragraph{Dummit-Foote.exercise.7.1.12} If $K$ is a subring of a field $F$, then $K$ is a domain if and only if $1 \in K$.

\paragraph{Dummit-Foote.exercise.7.2.2} If $p$ is a polynomial with $p \mid 0$, then there exists a $b \in R$ such that $b \neq 0$ and $b \cdot p = 0$. Conversely, if $p$ is a polynomial with $p \nmid 0$, then there is no $b \in R$ such that $b \neq 0$ and $b \cdot p = 0$.

\paragraph{Dummit-Foote.exercise.7.3.16} If $R$ is a commutative ring, and $S$ is a subring of $R$, then $R$ is a subring of $S$ if and only if $R$ is a central subring of $S$. (Hint: Use the fact that $R$ is a subring of $S$ if and only if $R$ is a subring of $R \times S$.)

\paragraph{Dummit-Foote.exercise.7.4.27} If $a$ is a nilpotent element of a commutative ring $R$, then $1-a*b$ is a unit.

\paragraph{Dummit-Foote.exercise.8.2.4} If $R$ is a commutative ring with no zero divisors, and if $a, b \in R$ are such that $a \neq 0$ and $b \neq 0$, then there exist $r, s \in R$ such that $r*a + s*b = 1$. Moreover, if $a$ is a unit, then $a$ is a generator of the principal ideal generated by $a$.

\paragraph{Dummit-Foote.exercise.8.3.5a} If $n$ is a positive integer, then $n$ is irreducible if and only if $n$ is not a square and $n$ is not a perfect square.

\paragraph{Dummit-Foote.exercise.8.3.6b} If $q$ is a prime number, then there exists a field $R$ such that $R$ is a finite extension of the Gaussian integers, and $R$ has $q^2$ elements.

\paragraph{Dummit-Foote.exercise.9.1.10} Prove that the set of primes is infinite.

\paragraph{Dummit-Foote.exercise.9.4.2a} Prove that the polynomial $X^4 - 4X^3 + 6$ is irreducible over $\mathbb{Z}$.

\paragraph{Dummit-Foote.exercise.9.4.2c} Prove that the polynomial $X^4 + 4X^3 + 6X^2 + 2X + 1$ is irreducible over $\mathbb{Z}$.

\paragraph{Dummit-Foote.exercise.9.4.9} Let $X$ be a polynomial ring over $\mathbb{Q}$ in $2$ variables, and let $C$ be a square-free polynomial in $X$ of degree $2$. Prove that $X^2 - C$ is irreducible.

\paragraph{Dummit-Foote.exercise.11.1.13} If $ι$ is an infinite cardinal, then there is a bijection between $ι$ and $\mathbb{R}$.

\paragraph{Herstein.exercise.2.1.18} If $G$ is a finite group, then there exists an element $a$ of $G$ such that $a \neq 1$ and $a = a^{-1}$.

\paragraph{Herstein.exercise.2.1.26} If $G$ is a finite group, then there exists an integer $n$ such that $a^n=1$ for all $a \in G$. (Hint: Use the fact that $G$ is a finite group and the fact that $G$ is a group.)

\paragraph{Herstein.exercise.2.2.3} If $G$ is a group, and $P$ is a property of elements of $G$, then there exists an $n$ such that $P$ holds for all $n$-tuples of elements of $G$.

\paragraph{Herstein.exercise.2.2.6c} If $G$ is a group, and $n$ is a positive integer, then for all $a, b \in G$, $(a * b) ^ n = a ^ n * b ^ n$.

\paragraph{Herstein.exercise.2.3.16} If $G$ is a group, then $G$ is cyclic if and only if $G$ is finite and $G$ has prime order.

\paragraph{Herstein.exercise.2.5.23} f $G$ is a group, and $H$ is a subgroup, then there exists a unique integer $j$ such that $b*a = a^j * b$ for all $a, b \in G$. (Hint: Use the fact that $H$ is normal.)

\paragraph{Herstein.exercise.2.5.31} If $G$ is a finite group, and $p$ is a prime dividing the order of $G$, then $G$ has a subgroup of order $p^n$ if and only if $p$ does not divide $n$.

\paragraph{Herstein.exercise.2.5.43} A group is commutative if the order of its elements is a power of $9$.

\paragraph{Herstein.exercise.2.5.52} If $G$ is a group and $φ$ is an isomorphism from $G$ to $G$, then $φ$ is an automorphism of $G$.

\paragraph{Herstein.exercise.2.7.7} If $G$ is a group and $N$ is a normal subgroup of $G$, then the map $\varphi : G \to G'$ given by $\varphi(g) = gN$ is a homomorphism.

\paragraph{Herstein.exercise.2.8.15} If $G$ and $H$ are finite groups of the same order, then $G \cong H$.

\paragraph{Herstein.exercise.2.10.1} If $b$ is an element of infinite order in $G$, then $A \cup \{b\}$ is a subgroup of $G$ that is not closed.

\paragraph{Herstein.exercise.2.11.7} If $G$ is a finite group, and $P$ is a Sylow $p$-subgroup of $G$, then the characteristic of $P$ is congruent modulo $p$ to the index of $P$ in $G$.

\paragraph{Herstein.exercise.3.2.21} If $σ$ and $τ$ are permutations of $α$ such that $σ$ is the identity and $τ$ is not, then $σ = 1$ and $τ = 1$. 

\paragraph{Herstein.exercise.4.1.34} Let $G$ be a general linear group over a field $K$. Then $G$ is isomorphic to a subgroup of the group of permutations of a finite set of $K$ with $z$-action.

\paragraph{Herstein.exercise.4.2.6} If $a$ is a nonzero element of a ring $R$, then $a^2 = 0$ if and only if $a * (a * x + x * a) = (x + x * a) * a$ for all $x \in R$.

\paragraph{Herstein.exercise.4.3.1} If $R$ is a commutative ring, then there exists an ideal $I$ such that $I*a=0$ for all $a \in R$.

\paragraph{Herstein.exercise.4.4.9} If $p$ is a prime number, then there exists a set $S$ of size $(p-1)/2$ such that $S^2 = p$ and $S$ is not a group.

\paragraph{Herstein.exercise.4.5.23} If $p$ and $q$ are irreducible polynomials over $\mathbb{Z}/7\mathbb{Z}$ such that $p$ and $q$ are not multiples of each other, then $p$ and $q$ are not multiples of each other over $\mathbb{Z}/7\mathbb{Z}$.

\paragraph{Herstein.exercise.4.6.2} Prove that the polynomial $X^3 + 3X + 2$ is irreducible over $\mathbb{Q}$.

\paragraph{Herstein.exercise.5.1.8} If $p$ is a prime number, and $a, b \in F$, then $a^m + b^m = (a + b)^m$.

\paragraph{Herstein.exercise.5.3.7} If $a$ is algebraic over $F$, then $a^2$ is algebraic over $F$.

\paragraph{Herstein.exercise.5.4.3} rove that there exists a polynomial $p$ of degree less than $80$ such that $p(a)=0$ for all $a\in\mathbb{C}$.

\paragraph{Herstein.exercise.5.6.14} If $p$ is prime, then the number of roots of $X^m-X$ in the field $F$ is $m$. (Hint: Use the Chinese remainder theorem.)

\paragraph{Ireland-Rosen.exercise.1.30} There is no natural number $a$ such that $\sum_{i=1}^{n+2} 1/i = a$ for all $n \in \mathbb{N}$.

\paragraph{Ireland-Rosen.exercise.2.4} If $a$ is nonzero, then $a^2$ is nonzero.

\paragraph{Ireland-Rosen.exercise.2.27a} If $p$ is a prime number, and $1$ is not a square modulo $p$, then the sequence $(1, 1, 1, \dots)$ is not summable.

\paragraph{Ireland-Rosen.exercise.3.4} Prove that there is no $x$ and $y$ such that $3x^2 + 2 = y^2$.

\paragraph{Ireland-Rosen.exercise.3.10} If $n$ is not a prime number, then $n!$ is divisible by $n-1$.

\paragraph{Ireland-Rosen.exercise.4.4} If $p$ is a prime, then $p$ is a primitive root modulo $p$ if and only if $p$ is a primitive root modulo $-p$.

\paragraph{Ireland-Rosen.exercise.4.6} If $p$ is a prime number, then $3$ is a primitive root modulo $p$ if and only if $p = 2^n + 1$ for some $n \in \mathbb{N}$.

\paragraph{Ireland-Rosen.exercise.4.11} If $p$ is prime, then $p^k$ is congruent to $1$ modulo $p$.

\paragraph{Ireland-Rosen.exercise.5.28} If $p$ is a prime congruent to $1$ modulo $4$, then there exist integers $A$ and $B$ such that $p = A^2 + 64B^2$. Conversely, if $p$ is a prime congruent to $1$ modulo $4$, then there exist integers $A$ and $B$ such that $p = A^2 + 64B^2$.

\paragraph{Ireland-Rosen.exercise.12.12} The algebraic number $\sin(\frac{\pi}{12})$ is irrational.

\paragraph{Munkres.exercise.13.3b} If $X$ is a set, and $s$ is a set of subsets of $X$, then the following are equivalent: (1) $s$ is infinite, (2) there is some $t \in s$ such that $t$ is infinite, (3) there is some $t \in s$ such that $t$ is not a subset of any other element of $s$, and (4) there is some $t \in s$ such that $t$ is not a subset of any other element of $s$ and $t$ is not a subset of any other element of $s$. (Hint: Use the definition of infinite set and the fact that $s$ is a set of subsets of $X$.)

\paragraph{Munkres.exercise.13.4a2} There exists a set $X$ and a family of topologies $\{T_i\}_{i\in I}$ on $X$ such that $T_i$ is the topology induced by $X$ on $X$ for each $i\in I$, but there is no topology on $X$ that is the union of the $T_i$.

\paragraph{Munkres.exercise.13.4b2} Prove that if $X$ is a topological space, and $T$ is a topology on $X$, then there is a topology $T'$ on $X$ such that $T' \subseteq T$ and $T'$ is a topology on $X$.

\paragraph{Munkres.exercise.13.5b} If $A$ is a set of subsets of $X$, then $A$ is the set of all subsets of $X$ that are generated by $A$. In other words, $A$ is the smallest set that contains $A$ and is closed under taking subsets.

\paragraph{Munkres.exercise.13.8a} If $S$ is a topological basis for $\mathbb{R}$, then $S$ is a basis for $\mathbb{R}$.

\paragraph{Munkres.exercise.16.1} If $U$ is an open subset of $A$, then $U$ is open in the subspace topology on $A$.

\paragraph{Munkres.exercise.16.6} If $S$ is a set of pairs of real numbers, then there exists a basis for the topology on $S$.

\paragraph{Munkres.exercise.18.8a} If $f$ and $g$ are continuous functions from a topological space $X$ to a topological space $Y$, then the set of points $x$ such that $f(x) \leq g(x)$ is closed.

\paragraph{Munkres.exercise.18.13} Suppose $X$ and $Y$ are topological spaces, and $A$ is a subset of $X$. Prove that if $f$ is a continuous function from $A$ to $Y$, then $f$ is a closure operator on $A$.

\paragraph{Munkres.exercise.20.2} A topological space is metrizable if it is homeomorphic to a metric space.

\paragraph{Munkres.exercise.21.6b} Prove that if $f$ is a function from $\mathbb{N}$ to $I$ such that $f(n) = n$ for all $n$, then there is a function $f_0$ such that $f_0(n) = n$ for all $n$ and $f_0$ is not uniformly continuous.

\paragraph{Munkres.exercise.22.2a} If $p$ is a continuous function from $X$ to $Y$, then there exists a continuous function $f$ from $Y$ to $X$ such that $p \circ f = id_Y$. (Hint: Use the fact that $p$ is a quotient map.)

\paragraph{Munkres.exercise.22.5} If $p$ is a continuous map from $X$ to $Y$, then $p$ is open if and only if $p ∘ subtype.val$ is open.

\paragraph{Munkres.exercise.23.3} Prove that if $A$ is a nonempty set of points in a topological space $X$, then $A$ is connected if and only if $A$ is the union of a nonempty set of connected sets.

\paragraph{Munkres.exercise.23.6} If $C$ is a connected subset of a topological space $X$, then $C$ is not a singleton.

\paragraph{Munkres.exercise.23.11} If $X$ is a connected topological space, and $p$ is a continuous surjection from $X$ onto $Y$, then $p$ is a quotient map if and only if $Y$ is connected.

\paragraph{Munkres.exercise.24.3a} If $I$ is a topological space, and $f$ is a continuous function from $I$ into $I$, then there is a point $x \in I$ such that $f(x) = x$. (Hint: Use the fact that $I$ is a topological space.)

\paragraph{Munkres.exercise.25.9} If $G$ is a topological group, and $C$ is a connected component of $G$, then $C$ is a normal subgroup of $G$.

\paragraph{Munkres.exercise.26.12} If $X$ is a topological space, and $Y$ is compact, then $X$ is compact.

\paragraph{Munkres.exercise.28.4} If $X$ is a topological space, then $X$ is countably compact if and only if every point of $X$ is a limit point of $X$.

\paragraph{Munkres.exercise.28.6} If $f$ is an isometry of $X$ into itself, then $f$ is a bijection.

\paragraph{Munkres.exercise.29.4} Prove that the space of all continuous functions from the natural numbers to the unit interval is not locally compact.

\paragraph{Munkres.exercise.30.10} Prove that there exists a set $S$ such that $S$ is countable and dense in $\prod_{i\in I}X_i$ for any family of topological spaces $\{X_i\}_{i\in I}$.

\paragraph{Munkres.exercise.31.1} If $X$ is a topological space, then there exist two open sets $U$ and $V$ such that $x \in U$ and $y \in V$ and $U \cap V = \emptyset$. 

\paragraph{Munkres.exercise.31.3} If $X$ is a topological space, then $X$ is regular if and only if $X$ is a regular space.

\paragraph{Munkres.exercise.32.2a} Prove that the product of two topological spaces is a topological space.

\paragraph{Munkres.exercise.32.2c} Prove that if $X$ is a family of topological spaces, then the normal spaces of the product $X$ are the products of the normal spaces of the $X_i$.

\paragraph{Munkres.exercise.33.7} If $X$ is a locally compact space, and $x \in X$, then there exists a continuous function $f$ such that $f(x) = 1$ and $f(A) = \{0\}$ for all closed sets $A$.

\paragraph{Munkres.exercise.34.9} Prove that the union of two compact spaces is compact.

\paragraph{Munkres.exercise.43.2} Suppose $X$ is a metric space, and $Y$ is a complete metric space. Let $f$ be a uniformly continuous function from $X$ into $Y$, and let $h$ be a uniformly continuous function from $X$ into $Y$ such that $h(x)=f(x)$ for all $x \in A$. Prove that there exists a continuous function $g$ from $X$ into $Y$ such that $g(x)=f(x)$ for all $x \in A$ and $g(x)=h(x)$ for all $x \in X \setminus A$.

\paragraph{Pugh.exercise.2.26} If $U$ is an open set in a topological space $M$, then $U$ is open if and only if for every $x \in U$, the cluster point of $x$ in $U$ is not a cluster point of $U$.

\paragraph{Pugh.exercise.2.32a} Prove that the set $A$ is clopen.

\paragraph{Pugh.exercise.2.46} If $A$ and $B$ are disjoint nonempty subsets of a metric space $M$, then there exist points $a_0 \in A$ and $b_0 \in B$ such that $a_0 \neq b_0$ and $a_0$ is closer to $b_0$ than any other point of $A$ is to any other point of $B$. (Hint: Use the fact that $A$ and $B$ are disjoint.)

\paragraph{Pugh.exercise.2.92} If $s$ is a sequence of points in a topological space $α$, then $s$ has a nonempty subsequence.

\paragraph{Pugh.exercise.3.1} If $f$ is a function from $\mathbb{R}$ to $\mathbb{R}$ such that $|f(x) - f(y)| \leq |x - y|^2$ for all $x, y \in \mathbb{R}$, then there is a constant $c$ such that $f(x) = c$ for all $x \in \mathbb{R}$.

\paragraph{Pugh.exercise.3.63a} If $f$ is a function from $\mathbb{N}$ to $\mathbb{R}$ such that $f(n) = \frac{1}{n^p}$ for all $n \in \mathbb{N}$, then there exists $l \in \mathbb{N}$ such that $f$ tends to $0$ at the top of $\mathbb{R}$.

\paragraph{Pugh.exercise.4.15a} If $F$ is a set of functions from $\mathbb{R}$ to $\mathbb{R}$ such that for all $x \in \mathbb{R}$ and all $f \in F$, $f(x)$ is a real number, and if $F$ is closed under taking limits, then there exists a real number $\mu$ such that for all $x \in \mathbb{R}$ and all $f \in F$, $f(x)$ is a real number, and for all $s, t \in \mathbb{R}$ such that $s \neq t$, $|f(s) - f(t)| \leq \mu |s - t|$.

\paragraph{Putnam.exercise.1998.b6} Prove that there exists a positive integer $n$ such that $n^3 + a*n^2 + b*n + c$ is not a perfect square.

\paragraph{Putnam.exercise.1999.b4} If $f$ is a continuous function from $\mathbb{R}$ to $\mathbb{R}$, and $f$ is differentiable at $x$, then $f$ is strictly increasing on $[0, f(x)]$ and strictly decreasing on $[f(x), \infty)$.

\paragraph{Putnam.exercise.2001.a5} There exists a positive integer $n$ such that $a^n - (a+1)^n = 2001$.

\paragraph{Putnam.exercise.2014.a5} If $P$ is a polynomial with integer coefficients, then $P$ is coprime to any other polynomial with integer coefficients.

\paragraph{Putnam.exercise.2018.a5} If $f$ is a continuous function from $\mathbb{R}$ to $\mathbb{R}$, prove that there exists an integer $n$ such that $f$ is $n$-times differentiable at $0$ and $f^{(n)}(0) = 0$. (Hint: Use the intermediate value theorem.)

\paragraph{Putnam.exercise.2018.b4} Prove that there exists a constant $c$ such that $x$ is periodic with period $c$.

\paragraph{Rudin.exercise.1.1b} If $x$ is irrational, then $x * y$ is irrational.

\paragraph{Rudin.exercise.1.4} Prove that $x$ is an upper bound of $s$ if and only if $x$ is a lower bound of $s$.

\paragraph{Rudin.exercise.1.8} There is no linear order on the reals.

\paragraph{Rudin.exercise.1.12} If $f$ is a function from $\mathbb{N}$ to $\mathbb{R}$, then $\sum_{i\in\mathbb{N}} |f(i)| \leq \sum_{i\in\mathbb{N}} |f(i)|$.

\paragraph{Rudin.exercise.1.14} If $z \in \mathbb{C}$ is a complex number, then $1 + z$ and $1 - z$ are both positive, and $1 + z$ and $1 - z$ are both negative.

\paragraph{Rudin.exercise.1.17} If $x, y \in \mathbb{R}^n$ are unit vectors, then $\langle x + y, x - y \rangle = 2\langle x, y \rangle$.

\paragraph{Rudin.exercise.1.18b} If $x$ is a real number, then there is no real number $y$ such that $x * y = 0$. 

\paragraph{Rudin.exercise.2.19a} Prove that if $A$ and $B$ are closed subsets of a metric space $X$, then $A \cap B$ is closed.

\paragraph{Rudin.exercise.2.25} There exists a countable basis for the topology of $K$.

\paragraph{Rudin.exercise.2.27b} Prove that the set of points in the euclidean space that are not contained in a countable set is not countable.

\paragraph{Rudin.exercise.2.29} Prove that there exists a function $f : \mathbb{N} \to \mathbb{R}$ such that $f(n) \in f(n)$ for all $n \in \mathbb{N}$ and $f(n) \cap f(m) = \emptyset$ for all $n \neq m$. Then show that $U = \bigcup_{n \in \mathbb{N}} f(n)$.

\paragraph{Rudin.exercise.3.2a} Prove that the sequence $(n \mapsto \sqrt{n^2 + n} - n)$ tends to $0$ at $0$.

\paragraph{Rudin.exercise.3.5} If $a, b$ are real numbers such that $a + b$ is not zero, then the limit of the sequence $a + b, a + 2b, a + 3b, \dots$ is not zero.

\paragraph{Rudin.exercise.3.7} Suppose that $a$ is a function from $\mathbb{N}$ to $\mathbb{R}$ such that $\lim_{n\to\infty} a(n) = 0$. Prove that there exists a real number $y$ such that $\lim_{n\to\infty} \sqrt[n]{a(n)}/n = y$. (Hint: Use the Bolzano-Weierstrass theorem.)

\paragraph{Rudin.exercise.3.13} Prove that if $a$ and $b$ are integers, then there exists a real number $y$ such that $a$ and $b$ are both the sums of the first $y$ integers.

\paragraph{Rudin.exercise.3.21} Prove that the set of all limit points of the sequence $E$ is a singleton.

\paragraph{Rudin.exercise.4.1a} There exists a function $f : \mathbb{R} \to \mathbb{R}$ such that for all $x \in \mathbb{R}$, $f(x + y) - f(x - y)$ tends to $0$ as $y$ tends to $0$.

\paragraph{Rudin.exercise.4.3} Prove that $f$ is closed if $f$ is continuous and $f^{-1}(0)$ is closed.

\paragraph{Rudin.exercise.4.4b} Prove that if $f$ and $g$ are continuous functions from a metric space $α$ to a metric space $β$, then $f = g$ if and only if $f$ is uniformly continuous and $g$ is dense.

\paragraph{Rudin.exercise.4.5b} There exists a function $f : \mathbb{R} \to \mathbb{R}$ such that $f$ is continuous on $E$ and $f(x) = x$ for all $x \in E$. (Hint: Use the fact that $E$ is countable and the fact that the real numbers are uncountable.)

\paragraph{Rudin.exercise.4.8a} If $f$ is uniformly continuous on $E$, then $f$ is bounded on $E$. If $E$ is bounded, then $f$ is uniformly continuous.

\paragraph{Rudin.exercise.4.11a} Prove that the sequence $(f(x_n))_{n\in\mathbb{N}}$ is Cauchy if and only if the sequence $(x_n)_{n\in\mathbb{N}}$ is Cauchy.

\paragraph{Rudin.exercise.4.15} If $f$ is a continuous function from $\mathbb{R}$ to $\mathbb{R}$, then $f$ is monotone.

\paragraph{Rudin.exercise.4.21a} If $X$ is a metric space, and $K$ and $F$ are closed subsets of $X$, then there exists a real number $\delta > 0$ such that for all $p, q \in K$ with $p \neq q$, $p$ and $q$ are at least $\delta$ apart.

\paragraph{Rudin.exercise.5.1} If $f$ is a continuous function from $\mathbb{R}$ to $\mathbb{R}$, prove that there is a constant $c$ such that $f(x) = c$ for all $x \in \mathbb{R}$.

\paragraph{Rudin.exercise.5.3} Suppose $g$ is a continuous function from $\mathbb{R}$ to $\mathbb{R}$. Prove that there exists a number $N$ such that for all $\epsilon > 0$, there exists a number $M$ such that for all $x \in \mathbb{R}$, $|g(x) - (x + \epsilon * g(x))| < M$. (Hint: Use the intermediate value theorem.)

\paragraph{Rudin.exercise.5.5} Prove that the function $f$ is differentiable at $0$ and that the derivative of $f$ at $0$ is $1$. Then show that the function $f$ is continuous at $0$ and that the derivative of $f$ at $0$ is $1$. Finally, show that the function $f$ is differentiable at $0$ and that the derivative of $f$ at $0$ is $1$.

\paragraph{Rudin.exercise.5.7} If $f$ and $g$ are differentiable at $0$, then the function $t \mapsto f(t)/g(t)$ tends to $0$ as $t$ tends to $0$. (Hint: Use the chain rule.)

\paragraph{Rudin.exercise.5.17} If $f$ is differentiable at $x$, then $f$ is strictly increasing on $(-1, 1)$.

\paragraph{Shakarchi.exercise.1.13b} If $f$ is differentiable at $a$ and $b$, then $f$ is constant on the interval $(a, b)$.

\paragraph{Shakarchi.exercise.1.19a} Prove that if $z$ is a complex number with absolute value $1$, then there is a sequence $y_n$ in the complex plane such that $y_n$ tends to $z$ at the top.

\paragraph{Shakarchi.exercise.1.19c} If $z$ is a complex number such that $z^n = 1$ for all $n$, then there is a complex number $z'$ such that $z' = z$ and $z'^n = 1$ for all $n$. (Hint: Use the fact that $z^n = 1$ for all $n$ to show that $z'$ is a root of the polynomial $x^n - 1$.)

\paragraph{Shakarchi.exercise.2.2} Prove that the function $f(x) = \int_0^x \sin t \, dt$ tends to $0$ as $x$ tends to $0$ from the right.

\paragraph{Shakarchi.exercise.2.13} Prove that $f$ is a polynomial of degree $n$.

\paragraph{Shakarchi.exercise.3.4} Prove that the function $t \mapsto \int_{-t}^t x \sin x \, dx$ tends to $0$ as $t \to \infty$.

\paragraph{Shakarchi.exercise.3.14} If $f$ is differentiable at $a$, then there is a unique $b$ such that $f(z) = (a * z + b)$.

\paragraph{Shakarchi.exercise.5.1} If $f$ is differentiable at $0$, then there is a unique $z \in \mathbb{C}$ such that $f(z) = 0$ and $z$ is a limit point of the sequence of points $z_n = (1 - \sum_{i \in finset.range n} \frac{1}{i})$.
\end{document}