\documentclass{article}

\title{\textbf{
Exercises from \\
\textit{Everything} \\
by All Authors
}}

\date{}

\usepackage{amsmath}
\usepackage{amssymb}
\usepackage{fullpage}

\begin{document}
\maketitle

\paragraph{Artin.exercise.10.1.13} If $x$ is nilpotent, then $1+x$ is a unit.

\paragraph{Artin.exercise.10.2.4} Prove that $\mathrm{span}(2) \cap \mathrm{span}(X) = \mathrm{span}(2X)$.

\paragraph{Artin.exercise.10.4.6} Let $R$ be a commutative ring with no zero divisors. Let $I$ and $J$ be ideals of $R$. Prove that the ideal $I\cap J$ is nilpotent.

\paragraph{Artin.exercise.10.4.7a} If $I$ and $J$ are ideals of a commutative ring $R$ such that $I+J=R$, then $IJ=I\cap J$.

\paragraph{Artin.exercise.10.5.16} If $F$ is a field of characteristic $2$, then the quotient of the polynomial ring $F[X]$ by the ideal generated by $X^2$ is isomorphic to the quotient of $F[X]$ by the ideal generated by $X^2-1$.

\paragraph{Artin.exercise.10.6.7} If $I$ is a nonzero ideal of $\mathbb{Z}[i]$, then there is an element $z\in I$ such that $z$ is real.

\paragraph{Artin.exercise.10.7.10} Let $M$ be an ideal of a ring $R$ such that for each $x\in R$, if $x\notin M$ then $x$ is a unit. Prove that $M$ is a maximal ideal and that every maximal ideal of $R$ is equal to $M$.

\paragraph{Artin.exercise.10.7.6} Prove that the quotient of the polynomial ring $F[X]$ by the ideal generated by $X^2+X+1$ is a field if $F$ is a field of order $5$.

\paragraph{Artin.exercise.11.12.3} If $p$ is a prime and $a$ is a square root of $-5$ modulo $p$, then $p$ is congruent to $1$ or $3$ modulo $8$.

\paragraph{Artin.exercise.11.13.3} Prove that there is a prime $p$ such that $p\geq N$ and $p+1\equiv 0\pmod{4}$.

\paragraph{Artin.exercise.11.2.13} If $a$ divides $b$ in the Gaussian integers, then $a$ divides $b$ in the integers.

\paragraph{Artin.exercise.11.3.1} If $p(x)$ is an irreducible polynomial over $F$, then $p(ax+b)$ is irreducible over $F$.

\paragraph{Artin.exercise.11.3.4} Prove that $x^3 + 6x + 12$ is irreducible over $\mathbb{Q}$.

\paragraph{Artin.exercise.11.4.1b} Prove that $x^3+6x+12$ is irreducible over $\mathbb{F}_2$.

\paragraph{Artin.exercise.11.4.6a} Prove that $X^2+1$ is irreducible over a field of order $7$.

\paragraph{Artin.exercise.11.4.6b} Prove that $x^3-9$ is irreducible over $\mathbb{F}_{31}$.

\paragraph{Artin.exercise.11.4.6c} Prove that $X^3-9$ is irreducible over $\mathbb{Z}_{31}$.

\paragraph{Artin.exercise.11.4.8} Prove that $X^n-p$ is irreducible over $\mathbb{Q}$ for any prime $p$ and any positive integer $n$.

\paragraph{Artin.exercise.13.4.10} Prove that every prime of the form $2^n+1$ is of the form $2^{2^k}+1$.

\paragraph{Artin.exercise.13.6.10} Prove that the only element of $\mathbb{F}_2^n$ is $-1$.

\paragraph{Artin.exercise.2.11.3} If $G$ is a finite group of even order, then $G$ has an element of order $2$.

\paragraph{Artin.exercise.2.2.9} Let $G$ be a group and let $a, b \in G$ be such that $ab=ba$. Prove that the closure of the set $\{a, b\}$ is a commutative subgroup of $G$.

\paragraph{Artin.exercise.2.3.1} The multiplicative group of the reals is isomorphic to the additive group of the reals.

\paragraph{Artin.exercise.2.3.2} Let $a$ and $b$ be elements of a group $G$. Prove that there exists an element $g$ of $G$ such that $b^{-1}a=gag^{-1}b$.

\paragraph{Artin.exercise.2.4.19} If $x$ is an element of order $2$ in $G$, then $x$ is in the center of $G$.

\paragraph{Artin.exercise.2.8.6} The center of the direct product of two groups is isomorphic to the direct product of the centers.

\paragraph{Artin.exercise.3.2.7} If $F$ is a field and $G$ is a field extension of $F$, then the inclusion map $F\to G$ is injective.

\paragraph{Artin.exercise.3.5.6} If $S$ is a countable set of vectors in a vector space $V$ such that $\operatorname{span}(S)=V$, then any set of vectors $\{v_i\}_{i\in I}$ that is linearly independent from $S$ is countable.

\paragraph{Artin.exercise.3.7.2} If $V$ is a vector space over a field $K$, and $\{W_i\}_{i\in I}$ is a collection of subspaces of $V$, then $\bigcap_{i\in I} W_i$ is nontrivial.

\paragraph{Artin.exercise.6.1.14} If $G$ is a group whose quotient by the center is cyclic, then the center of $G$ is the whole group.

\paragraph{Artin.exercise.6.4.12} Prove that there is no simple group of order 224.

\paragraph{Artin.exercise.6.4.2} If $G$ is a finite group of order $pq$ where $p$ and $q$ are distinct primes, then $G$ is not simple.

\paragraph{Artin.exercise.6.4.3} Prove that if $G$ is a finite group of order $p^2q$, where $p$ and $q$ are primes, then $G$ is not simple.

\paragraph{Artin.exercise.6.8.1} Let $G$ be a group and let $a, b \in G$. Prove that the closure of $\{a, b\}$ is equal to the closure of $\{bab^{-1}, bab^{-2}\}$.

\paragraph{Artin.exercise.6.8.4} The closure of the set $\{x,y,z\}$ is the free group generated by $x,y,z$.

\paragraph{Artin.exercise.6.8.6} If $G$ is cyclic and $N$ is a normal subgroup of $G$ such that $G/N$ is cyclic, then $G$ is generated by two elements.

\paragraph{Axler.exercise.1.2} Prove that $(-1/2 + i\sqrt{3}/2)^3 = -1$.

\paragraph{Axler.exercise.1.3} Prove that $-\left(-v\right) = v$.

\paragraph{Axler.exercise.1.4} If $v$ is a vector in a vector space $V$ over a field $F$, and $a$ is an element of $F$, then $av=0$ if and only if $a=0$ or $v=0$.

\paragraph{Axler.exercise.1.6} Prove that there exists a subset $U$ of $\mathbb{R}^2$ such that $U$ is not a submodule of $\mathbb{R}^2$.

\paragraph{Axler.exercise.1.7} Prove that there exists a nonempty subset $U$ of $\mathbb{R}^2$ such that $cU=U$ for all $c\in\mathbb{R}$ and $U$ is not a subspace of $\mathbb{R}^2$.

\paragraph{Axler.exercise.1.8} Let $V$ be a vector space over a field $F$, and let $\{U_i\}_{i\in I}$ be a collection of subspaces of $V$. Prove that $\bigcap_{i\in I} U_i$ is a subspace of $V$.

\paragraph{Axler.exercise.1.9} Let $U$ and $W$ be subspaces of a vector space $V$. Prove that $U$ and $W$ are comparable if and only if $U\cap W$ is a subspace of $V$.

\paragraph{Axler.exercise.3.1} If $T$ is a linear transformation of a finite-dimensional vector space $V$ over a field $F$ and $T$ has rank $1$, then $T$ is a scalar multiple of the identity.

\paragraph{Axler.exercise.3.8} Let $L:V\to W$ be a linear map. Then there is a subspace $U$ of $V$ such that $L(U)$ is a complement of $L(V)$ in $W$.

\paragraph{Axler.exercise.4.4} If $p$ is a polynomial, then the number of roots of $p$ is equal to the number of roots of $p'$.

\paragraph{Axler.exercise.5.1} Let $V$ be a vector space over a field $F$, and let $L:V\to V$ be a linear transformation. Let $U_1, \dots, U_n$ be subspaces of $V$ such that $L(U_i)=U_i$ for each $i$. Prove that $L(\sum_{i=1}^n U_i)=\sum_{i=1}^n U_i$.

\paragraph{Axler.exercise.5.11} If $S$ and $T$ are endomorphisms of a vector space $V$, then the eigenvalues of $ST$ are the same as the eigenvalues of $TS$.

\paragraph{Axler.exercise.5.12} If $S$ is a linear operator on a finite-dimensional vector space $V$ over a field $F$ such that every vector in $V$ is an eigenvector of $S$, then $S$ is a scalar multiple of the identity.

\paragraph{Axler.exercise.5.13} Let $V$ be a finite-dimensional vector space over a field $F$, and let $T:V\to V$ be a linear transformation. Suppose that for each subspace $U$ of $V$ of dimension $n-1$, $T(U)=U$. Prove that $T$ is a scalar multiple of the identity.

\paragraph{Axler.exercise.5.20} Let $S$ and $T$ be linear transformations of a finite-dimensional vector space $V$ over a field $F$. Prove that if $S$ and $T$ have the same number of distinct eigenvalues, then $ST=TS$.

\paragraph{Axler.exercise.5.24} Let $V$ be a finite-dimensional vector space over $\mathbb{R}$, and let $T$ be a linear operator on $V$ such that $T(x)=cx$ for all $x\in V$ and some $c\in\mathbb{R}$. Prove that the rank of any subspace $U$ of $V$ is even.

\paragraph{Axler.exercise.5.4} Let $S$ and $T$ be linear transformations of a vector space $V$ over a field $F$ such that $ST=TS$. Prove that $S$ maps the kernel of $T-cI$ onto the kernel of $T-cI$.

\paragraph{Axler.exercise.6.13} If $e_1, \dots, e_n$ is an orthonormal basis for $V$, then $v\in V$ is in the span of $e_1, \dots, e_n$ if and only if $|v|^2 = |\langle v, e_1\rangle|^2 + \dots + |\langle v, e_n\rangle|^2$.

\paragraph{Axler.exercise.6.16} If $U$ is a subspace of a complex inner product space $V$, then $U^{\perp} = \{0\}$ if and only if $U = V$.

\paragraph{Axler.exercise.6.2} If $u$ and $v$ are vectors in a complex inner product space, then $u$ is orthogonal to $v$ if and only if $|u|\leq |u+av|$ for all $a\in\mathbb{C}$.

\paragraph{Axler.exercise.6.3} If $a_1, \dots, a_n$ and $b_1, \dots, b_n$ are real numbers, then $(a_1b_1 + \dots + a_nb_n)^2 \leq (a_1^2 + \dots + a_n^2)(b_1^2 + \dots + b_n^2)$.

\paragraph{Axler.exercise.6.7} If $u$ and $v$ are vectors in a complex inner product space, then $\langle u, v\rangle = (|u+v|^2 - |u-v|^2 + i|u+iv|^2 - i|u-iv|^2)/4$.

\paragraph{Axler.exercise.7.10} Let $T$ be a linear operator on a finite-dimensional inner product space $V$. Prove that if $T$ is self-adjoint and $T^9=T^8$, then $T^2=T$.

\paragraph{Axler.exercise.7.11} Let $T$ be a linear operator on a finite-dimensional inner product space $V$ such that $T^*T=TT^*$. Prove that there exists a linear operator $S$ on $V$ such that $S^2=T$.

\paragraph{Axler.exercise.7.14} If $T$ is a self-adjoint linear operator on a finite-dimensional inner product space, then for any $\epsilon>0$ there is an eigenvalue $l'$ of $T$ such that $|l-l'|<\epsilon$.

\paragraph{Axler.exercise.7.5} If $V$ is a finite-dimensional complex inner product space of dimension at least 2, then the set of all operators $T$ such that $T^2=T^*T$ is a proper subspace of the space of all operators on $V$.

\paragraph{Axler.exercise.7.6} If $T$ is a linear operator on a finite-dimensional inner product space $V$ such that $T^*T=TT^*$, then $T$ is normal.

\paragraph{Axler.exercise.7.9} Let $T$ be a linear operator on a finite-dimensional inner product space $V$. Prove that $T$ is self-adjoint if and only if all of its eigenvalues are real.

\paragraph{Dummit-Foote.exercise.1.1.15} If $a_1, \dots, a_n$ are elements of a group $G$, then $(a_1\cdots a_n)^{-1} = a_n^{-1}\cdots a_1^{-1}$.

\paragraph{Dummit-Foote.exercise.1.1.16} If $x$ is an element of a group $G$ such that $x^2=1$, then the order of $x$ is either $1$ or $2$.

\paragraph{Dummit-Foote.exercise.1.1.17} If $x$ has order $n$, then $x^{-1}=x^{n-1}$.

\paragraph{Dummit-Foote.exercise.1.1.18} Prove that $x$ and $y$ commute if and only if $y^{-1}xy=x$ and $x^{-1}y^{-1}xy=1$.

\paragraph{Dummit-Foote.exercise.1.1.20} If $x$ is an element of a group $G$, then the order of $x$ is the same as the order of $x^{-1}$.

\paragraph{Dummit-Foote.exercise.1.1.22a} If $x$ is an element of a group $G$, and $g$ is an element of $G$, then the order of $x$ is equal to the order of $g^{-1}xg$.

\paragraph{Dummit-Foote.exercise.1.1.22b} If $a$ and $b$ are elements of a group $G$, prove that $|ab|=|ba|$.

\paragraph{Dummit-Foote.exercise.1.1.25} If $G$ is a group in which every element has order $2$, prove that $G$ is abelian.

\paragraph{Dummit-Foote.exercise.1.1.29} If $A$ and $B$ are groups, then $(A\times B, \cdot)$ is a group if and only if $A$ and $B$ are both commutative.

\paragraph{Dummit-Foote.exercise.1.1.2a} Prove that there exist integers $a$ and $b$ such that $a-b\neq b-a$.

\paragraph{Dummit-Foote.exercise.1.1.3} Prove that $a+b+c\equiv a+(b+c)\pmod{n}$.

\paragraph{Dummit-Foote.exercise.1.1.34} If $x$ is an element of infinite order in $G$, prove that the elements $x^n$, $n\in\mathbb{Z}$ are all distinct.

\paragraph{Dummit-Foote.exercise.1.1.4} Prove that $(a\cdot b)\cdot c \equiv a\cdot (b\cdot c)$ modulo $n$.

\paragraph{Dummit-Foote.exercise.1.1.5} Prove that the group $\mathbb{Z}_n$ is trivial if $n$ is prime.

\paragraph{Dummit-Foote.exercise.1.3.8} The group of permutations of $\mathbb{N}$ is infinite.

\paragraph{Dummit-Foote.exercise.1.6.11} Prove that the group $A\times B$ is isomorphic to the group $B\times A$.

\paragraph{Dummit-Foote.exercise.1.6.17} Let $f:G\to G$ be defined by $f(x)=x^{-1}$. Prove that $f$ is a homomorphism if and only if $G$ is abelian.

\paragraph{Dummit-Foote.exercise.1.6.23} Let $G$ be a group and let $\sigma$ be an automorphism of $G$ such that $\sigma(g)=1$ implies $g=1$ and $\sigma(\sigma(g))=g$ for all $g\in G$. Prove that $G$ is abelian.

\paragraph{Dummit-Foote.exercise.1.6.4} Prove that there is no isomorphism between the multiplicative group of real numbers and the multiplicative group of complex numbers.

\paragraph{Dummit-Foote.exercise.2.1.13} Let $H$ be a subgroup of the additive group of rational numbers. Prove that if $x\in H$ implies $1/x\in H$, then $H$ is either $\{0\}$ or the whole group.

\paragraph{Dummit-Foote.exercise.3.1.22a} If $H$ and $K$ are normal subgroups of $G$, then $H\cap K$ is normal in $G$.

\paragraph{Dummit-Foote.exercise.3.1.22b} If $H_i$ is a normal subgroup of $G$ for each $i\in I$, then $\cap_{i\in I} H_i$ is a normal subgroup of $G$.

\paragraph{Dummit-Foote.exercise.3.1.3a} If $A$ is a commutative group and $B$ is a subgroup of $A$, then the quotient group $A/B$ is abelian.

\paragraph{Dummit-Foote.exercise.3.2.16} If $p$ is prime and $a$ is coprime to $p$, then $a^p\equiv a\pmod{p}$.

\paragraph{Dummit-Foote.exercise.3.2.21a} If $H$ is a nontrivial subgroup of a finite group $G$, then $H=G$.

\paragraph{Dummit-Foote.exercise.3.2.8} If $H$ and $K$ are finite subgroups of $G$ with coprime orders, then $H\cap K = \{1\}$.

\paragraph{Dummit-Foote.exercise.3.4.1} Prove that a simple group is cyclic and has prime order.

\paragraph{Dummit-Foote.exercise.3.4.4} If $G$ is a finite commutative group, then for any divisor $n$ of $|G|$, there is a subgroup $H$ of $G$ such that $|H|=n$.

\paragraph{Dummit-Foote.exercise.3.4.5a} If $G$ is solvable, then so is any subgroup of $G$.

\paragraph{Dummit-Foote.exercise.3.4.5b} If $G$ is solvable and $H$ is a normal subgroup of $G$, then $G/H$ is solvable.

\paragraph{Dummit-Foote.exercise.4.3.26} If $X$ is a finite set with more than one element, prove that there is a permutation of $X$ that is not the identity.

\paragraph{Dummit-Foote.exercise.4.4.6a} A characteristic subgroup of a group is normal.

\paragraph{Dummit-Foote.exercise.4.5.13} If $G$ is a group of order $56$, then $G$ has a normal Sylow $p$-subgroup for some prime $p$.

\paragraph{Dummit-Foote.exercise.4.5.14} If $G$ is a group of order $312$, then $G$ has a normal Sylow $p$-subgroup for some prime $p$.

\paragraph{Dummit-Foote.exercise.4.5.1a} If $P$ is a $p$-subgroup of $G$ and $H$ is a subgroup of $G$ containing $P$, then $H$ is a $p$-subgroup of $G$.

\paragraph{Herstein.exercise.2.10.1} If $A$ is a normal subgroup of $G$ and $b$ is an element of prime order, then $A\cap \langle b\rangle = \{1\}$.

\paragraph{Herstein.exercise.2.11.22} If $G$ is a group of order $p^n$ and $K$ is a subgroup of order $p^{n-1}$, then $K$ is normal in $G$.

\paragraph{Herstein.exercise.2.11.6} If $P$ is a normal Sylow $p$-subgroup of $G$, then $P$ is the only Sylow $p$-subgroup of $G$.

\paragraph{Herstein.exercise.2.11.7} If $P$ is a normal Sylow $p$-subgroup of $G$, then $P$ is characteristic in $G$.

\paragraph{Herstein.exercise.2.1.18} If $G$ is a finite group of even order, then $G$ has an element of order $2$.

\paragraph{Herstein.exercise.2.1.21} Prove that a group of order $5$ is abelian.

\paragraph{Herstein.exercise.2.1.26} If $G$ is a finite group, then every element of $G$ has finite order.

\paragraph{Herstein.exercise.2.1.27} Prove that every finite group has an element of finite order.

\paragraph{Herstein.exercise.2.2.3} Prove that a group $G$ is commutative if and only if there exists an integer $n$ such that $a^n=b^n$ implies $a^n=b^n$ for all $a, b \in G$.

\paragraph{Herstein.exercise.2.2.5} Prove that a group $G$ is commutative if and only if $(ab)^3=a^3b^3$ and $(ab)^5=a^5b^5$ for all $a, b \in G$.

\paragraph{Herstein.exercise.2.2.6c} If $G$ is a group and $n>1$, and if $a^n=b^n$ for all $a,b\in G$, then $(abab^{-1}a^{-1})^{n(n-1)}=1$.

\paragraph{Herstein.exercise.2.3.16} Prove that a group $G$ is cyclic if and only if every subgroup of $G$ is either trivial or the whole group.

\paragraph{Herstein.exercise.2.3.17} If $a$ is an element of $G$ and $x$ is an element of $G$, then the centralizer of $x^{-1}ax$ is the image of the centralizer of $a$ under the map $g\mapsto x^{-1}gx$.

\paragraph{Herstein.exercise.2.3.19} Let $M$ be a subgroup of $G$. Prove that $M$ is normal if and only if for each $x\in G$, the set $\{x^{-1}mx: m\in M\}$ is contained in $M$.

\paragraph{Herstein.exercise.2.4.36} If $a>1$ and $n$ is a positive integer, prove that $n$ divides $\varphi(a^n-1)$.

\paragraph{Herstein.exercise.2.5.23} If $G$ is a group in which every subgroup is normal, prove that for any $a, b \in G$, there exists $j \in \mathbb{Z}$ such that $bab^{-1} = a^j$.

\paragraph{Herstein.exercise.2.5.30} If $G$ is a finite group of order $pm$ where $p$ is prime and $p$ does not divide $m$, then any normal subgroup of order $p$ is characteristic.

\paragraph{Herstein.exercise.2.5.31} If $H$ is a $p$-subgroup of $G$ and $G$ is a finite group of order $p^nm$, then $H$ is characteristic in $G$.

\paragraph{Herstein.exercise.2.5.37} If $G$ is a group of order $6$ and $G$ is not abelian, then $G$ is isomorphic to $S_3$.

\paragraph{Herstein.exercise.2.5.43} Prove that a group of order $9$ is abelian.

\paragraph{Herstein.exercise.2.5.44} If $G$ is a group of order $p^2$, where $p$ is prime, then $G$ has a normal subgroup of order $p$.

\paragraph{Herstein.exercise.2.5.52} If $G$ is a finite group and $\phi$ is an automorphism of $G$ such that $|\{x\in G: \phi(x)=x^{-1}\}|\geq 3/4|G|$, then $\phi(x)=x^{-1}$ for all $x\in G$ and $G$ is abelian.

\paragraph{Herstein.exercise.2.6.15} If $G$ is a commutative group, $m$ and $n$ are positive integers, and $m$ and $n$ are relatively prime, then there is an element of order $mn$ in $G$.

\paragraph{Herstein.exercise.2.7.7} If $N$ is a normal subgroup of $G$ and $\phi:G\to G'$ is a homomorphism, then $\phi(N)$ is a normal subgroup of $G'$.

\paragraph{Herstein.exercise.2.8.12} If $G$ and $H$ are non-abelian groups of order $21$, then $G$ is isomorphic to $H$.

\paragraph{Herstein.exercise.2.8.15} If $G$ and $H$ are groups of order $pq$ where $p>q$ are primes, then $G$ is isomorphic to $H$.

\paragraph{Herstein.exercise.2.9.2} If $G$ and $H$ are cyclic groups, then $G\times H$ is cyclic if and only if $|G|$ and $|H|$ are coprime.

\paragraph{Herstein.exercise.3.2.21} Let $\sigma$ and $\tau$ be permutations of a finite set $X$. Prove that if $\sigma(x)=x$ if and only if $\tau(x)\neq x$ for all $x\in X$, and $\tau\circ\sigma=\mathrm{id}_X$, then $\sigma=\mathrm{id}_X$ and $\tau=\mathrm{id}_X$.

\paragraph{Herstein.exercise.4.1.19} Prove that the set of all quaternions $x$ such that $x^2=-1$ is infinite.

\paragraph{Herstein.exercise.4.1.34} Prove that the group of permutations of $\{1, 2, 3\}$ is isomorphic to the group of $2\times 2$ matrices over $\mathbb{Z}/2\mathbb{Z}$.

\paragraph{Herstein.exercise.4.2.5} Prove that a ring $R$ is commutative if $x^3=x$ for all $x\in R$.

\paragraph{Herstein.exercise.4.2.6} If $a^2=0$ in a ring $R$, prove that $a(x+xa)=a(x+xa)$.

\paragraph{Herstein.exercise.4.2.9} Prove that if $p$ is an odd prime, then there exist integers $a$ and $b$ such that $a/b$ is equal to the sum of the reciprocals of the integers from $1$ to $p$, and $p$ divides $a$.

\paragraph{Herstein.exercise.4.3.1} Let $R$ be a commutative ring and $a$ an element of $R$. Prove that the set of all elements $x$ of $R$ such that $xa=0$ is an ideal of $R$.

\paragraph{Herstein.exercise.4.3.25} Prove that every ideal of the ring of $2\times 2$ real matrices is either the zero ideal or the whole ring.

\paragraph{Herstein.exercise.4.4.9} Prove that if $p$ is an odd prime, then there is a subset $S$ of $\mathbb{Z}_p$ such that $|S|=(p-1)/2$ and $S$ contains no element $x$ such that $x^2=p$.

\paragraph{Herstein.exercise.4.5.16} Let $p$ be a prime and let $q$ be an irreducible polynomial of degree $n$ over $\mathbb{Z}/p\mathbb{Z}$. Prove that the quotient ring $\mathbb{Z}/p\mathbb{Z}[x]/(q)$ is a field of order $p^n$.

\paragraph{Herstein.exercise.4.5.23} Prove that the polynomials $x^3-2$ and $x^3+2$ are irreducible over $\mathbb{Z}_7$ and that the quotient rings $\mathbb{Z}_7[x]/(x^3-2)$ and $\mathbb{Z}_7[x]/(x^3+2)$ are isomorphic.

\paragraph{Herstein.exercise.4.5.25} Prove that $X^p + X^{p-1} + \dots + X + 1$ is irreducible over $\mathbb{Q}$.

\paragraph{Herstein.exercise.4.6.2} Prove that $X^3 + 3X + 2$ is irreducible over $\mathbb{Q}$.

\paragraph{Herstein.exercise.4.6.3} Prove that the set of integers $a$ such that $X^7 + 15X^2 - 30X + a$ is irreducible is infinite.

\paragraph{Herstein.exercise.5.1.8} If $F$ is a field of characteristic $p$, then $(a+b)^p = a^p + b^p$.

\paragraph{Herstein.exercise.5.2.20} If $V$ is an infinite-dimensional vector space over a field $F$, then the union of any collection of proper subspaces of $V$ is a proper subspace of $V$.

\paragraph{Herstein.exercise.5.3.10} Prove that $\cos(\pi/180)$ is algebraic over $\mathbb{Q}$.

\paragraph{Herstein.exercise.5.3.7} If $a$ is algebraic over $F$, then $a^2$ is algebraic over $F$.

\paragraph{Herstein.exercise.5.4.3} Let $p(x) = x^5 + \sqrt{2}x^3 + \sqrt{5}x^2 + \sqrt{7}x + 11$. Prove that if $p(a)=0$, then there is a polynomial $q(x)$ of degree less than 80 such that $q(a)=0$ and the coefficients of $q(x)$ are rational numbers.

\paragraph{Herstein.exercise.5.5.2} Prove that $X^3 - 3X - 1$ is irreducible over $\mathbb{Q}$.

\paragraph{Herstein.exercise.5.6.14} If $F$ is a field of characteristic $p$, then the number of roots of $x^p-x$ is $p$.

\paragraph{Munkres.exercise.13.1} Let $X$ be a topological space and let $A$ be a subset of $X$. Prove that $A$ is open if for each $x\in A$ there is an open set $U$ containing $x$ such that $U\subset A$.

\paragraph{Munkres.exercise.13.3a} Prove that the collection of all open sets in a topological space is a topology.

\paragraph{Munkres.exercise.13.3b} Prove that the union of an infinite collection of infinite sets need not be infinite.

\paragraph{Munkres.exercise.13.4a1} If $T_i$ is a topology on $X$ for each $i\in I$, then $\bigcap_{i\in I} T_i$ is a topology on $X$.

\paragraph{Munkres.exercise.13.4a2} There exists a set $X$ and a family of topologies $\{T_i\}_{i\in I}$ on $X$ such that $\bigcap_{i\in I} T_i$ is not a topology on $X$.

\paragraph{Munkres.exercise.13.4b1} Let $T_i$, $i\in I$ be a family of topologies on $X$. Prove that there is a unique topology $T$ on $X$ such that $T_i\subset T$ for all $i\in I$ and $T$ is the smallest topology on $X$ with this property.

\paragraph{Munkres.exercise.13.4b2} Let $X$ be a set and let $\{T_i\}_{i\in I}$ be a family of topologies on $X$. Prove that there is a unique topology $T$ on $X$ such that $T_i\subset T$ for all $i\in I$ and $T$ is the smallest topology on $X$ with this property.

\paragraph{Munkres.exercise.13.5a} Let $A$ be a basis for a topology on $X$. Prove that the topology generated by $A$ is the intersection of all topologies on $X$ containing $A$.

\paragraph{Munkres.exercise.13.5b} Let $X$ be a set and let $A$ be a collection of subsets of $X$. Prove that the topology generated by $A$ is the intersection of all topologies on $X$ that contain $A$.

\paragraph{Munkres.exercise.13.6} The real line with the usual topology is not homeomorphic to the real line with the cofinite topology.

\paragraph{Munkres.exercise.13.8a} Prove that the collection of all open intervals $(a,b)$ with rational endpoints is a topological basis for the real line.

\paragraph{Munkres.exercise.13.8b} The topology generated by the intervals $[a,b]$ is not the same as the lower limit topology.

\paragraph{Munkres.exercise.16.1} Let $X$ be a topological space, $Y$ a subset of $X$, and $A$ a subset of $Y$. Prove that a subset $U$ of $A$ is open in $A$ if and only if $U$ is open in $Y$.

\paragraph{Munkres.exercise.16.4} Prove that the projections $\pi_1$ and $\pi_2$ are open maps.

\paragraph{Munkres.exercise.16.6} Let $S$ be the set of all open rectangles in the plane. Prove that $S$ is a basis for the topology of the plane.

\paragraph{Munkres.exercise.17.4} If $U$ is open and $A$ is closed, then $U\setminus A$ is open and $A\setminus U$ is closed.

\paragraph{Munkres.exercise.18.13} Let $X$ be a topological space, $Y$ a T2 space, and $A$ a subset of $X$. Let $f$ be a continuous mapping of $A$ into $Y$. Prove that if $g$ is a continuous mapping of $\overline{A}$ into $Y$ such that $g(x)=f(x)$ for all $x \in A$, then $g$ is the only continuous mapping of $\overline{A}$ into $Y$ with this property.

\paragraph{Munkres.exercise.18.8a} Let $X$ and $Y$ be topological spaces, and let $f$ and $g$ be continuous functions from $X$ into $Y$. Prove that the set $\{x\in X: f(x)\leq g(x)\}$ is closed in $X$.

\paragraph{Munkres.exercise.18.8b} If $f$ and $g$ are continuous functions from $X$ into $Y$, where $Y$ is a linearly ordered topological space, then the function $x\mapsto \min(f(x), g(x))$ is continuous.

\paragraph{Munkres.exercise.19.6a} Let $f_1, f_2, \dots$ be a sequence of functions from $X$ into $Y$. Prove that $f_n$ converges uniformly to $f$ if and only if $f_{n,x}$ converges to $f_x$ for each $x$ in $X$.

\paragraph{Munkres.exercise.20.2} Prove that the product topology on $\mathbb{R}^2$ is metrizable.

\paragraph{Munkres.exercise.21.6a} Let $f_n:I\to\mathbb{R}$ be defined by $f_n(x)=x^n$. Prove that $f_n$ converges uniformly to $f$ on $I$.

\paragraph{Munkres.exercise.21.6b} Prove that the sequence of functions $f_n(x)=x^n$ does not converge uniformly on any interval $I$.

\paragraph{Munkres.exercise.21.8} Let $X$ be a topological space, $Y$ a metric space, and $f_n:X\to Y$ a sequence of continuous functions. Let $x_n$ be a sequence of points in $X$ converging to $x_0\in X$, and let $f_0:X\to Y$ be a function such that $f_n$ converges uniformly to $f_0$. Prove that $f_n(x_n)$ converges to $f_0(x_0)$.

\paragraph{Munkres.exercise.22.2a} A continuous mapping $p:X\to Y$ is a quotient map if and only if there is a continuous mapping $f:Y\to X$ such that $p\circ f = id_Y$.

\paragraph{Munkres.exercise.22.2b} Let $X$ be a topological space, and let $A$ be a subset of $X$. Let $r$ be a continuous mapping of $X$ into $A$ such that $r(x)=x$ for all $x \in A$. Prove that $r$ is a quotient mapping.

\paragraph{Munkres.exercise.22.5} Let $X$ and $Y$ be topological spaces, and let $p:X\to Y$ be an open mapping. If $A$ is an open subset of $X$, then the restriction of $p$ to $A$ is an open mapping.

\paragraph{Munkres.exercise.23.11} If $X$ is a topological space, $Y$ is a connected space, and $p:X\to Y$ is a quotient map, then $X$ is connected.

\paragraph{Munkres.exercise.23.2} Let $X$ be a topological space and let $\{A_n\}_{n=1}^\infty$ be a sequence of connected subsets of $X$ such that $A_n\cap A_{n+1}\neq\emptyset$ for all $n$. Prove that $\bigcup_{n=1}^\infty A_n$ is connected.

\paragraph{Munkres.exercise.23.3} Let $X$ be a topological space, and let $\{A_n\}_{n=1}^\infty$ be a sequence of connected subsets of $X$. Suppose that $A_0$ is a connected subset of $X$ such that $A_0\cap A_n$ is nonempty for each $n$. Prove that $A_0\cup (\cup_{n=1}^\infty A_n)$ is connected.

\paragraph{Munkres.exercise.23.4} If $X$ is a cofinite topological space, then every infinite subset of $X$ is connected.

\paragraph{Munkres.exercise.23.6} If $C$ is a connected set and $A$ is a subset of $X$ such that $C\cap A$ and $C\cap A^c$ are nonempty, then $C\cap \partial A$ is nonempty.

\paragraph{Munkres.exercise.23.9} Let $A_1, A_2, B_1, B_2$ be connected sets in topological spaces $X, Y$ respectively, with $A_1\subset A_2$ and $B_1\subset B_2$. Prove that the set $A_2\times B_2 - A_1\times B_1$ is connected.

\paragraph{Munkres.exercise.24.2} Let $f$ be a continuous function on the unit sphere $S^n$ in $\mathbb{R}^{n+1}$. Prove that there exists $x\in S^n$ such that $f(x)=f(-x)$.

\paragraph{Munkres.exercise.24.3a} If $f$ is a continuous mapping of a compact space $I$ into itself, prove that $f$ has a fixed point.

\paragraph{Munkres.exercise.25.4} If $X$ is locally path-connected and $U$ is an open connected subset of $X$, then $U$ is path-connected.

\paragraph{Munkres.exercise.25.9} If $G$ is a topological group and $C$ is the connected component of the identity, then $C$ is a normal subgroup of $G$.

\paragraph{Munkres.exercise.26.11} Let $X$ be a compact Hausdorff space, and let $A$ be a collection of closed connected subsets of $X$ such that for any two sets $A_1, A_2 \in A$, either $A_1 \subset A_2$ or $A_2 \subset A_1$. Prove that $\bigcap_{A\in A} A$ is connected.

\paragraph{Munkres.exercise.26.12} Suppose $X$ and $Y$ are topological spaces, $p:X\to Y$ is a continuous surjection, and $p^{-1}(y)$ is compact for each $y\in Y$. If $Y$ is compact, then $X$ is compact.

\paragraph{Munkres.exercise.27.4} If $X$ is a connected metric space with at least two points, then $X$ is uncountable.

\paragraph{Munkres.exercise.28.4} A topological space $X$ is countably compact if and only if it is limit point compact.

\paragraph{Munkres.exercise.28.5} A topological space $X$ is countably compact if and only if every countable collection of closed sets with nonempty intersection has a point in common.

\paragraph{Munkres.exercise.28.6} If $X$ is a compact metric space and $f:X\to X$ is an isometry, then $f$ is a bijection.

\paragraph{Munkres.exercise.29.1} Prove that $\mathbb{Q}$ is not locally compact.

\paragraph{Munkres.exercise.29.10} Let $X$ be a $T_2$ space. If $x\in X$ and $U$ is an open set containing $x$, then there is an open set $V$ containing $x$ such that $\overline{V}$ is compact and $\overline{V}\subset U$.

\paragraph{Munkres.exercise.29.4} Prove that the space $\mathbb{N}^I$ is not locally compact.

\paragraph{Munkres.exercise.30.10} Let $X_i$ be a topological space for each $i\in\mathbb{N}$. Suppose that for each $i$, there is a countable dense subset $S_i$ of $X_i$. Prove that there is a countable dense subset of the product space $\prod_{i=1}^\infty X_i$.

\paragraph{Munkres.exercise.30.13} Let $X$ be a topological space. If $X$ has a countable dense subset, then the set of all open sets of $X$ is countable.

\paragraph{Munkres.exercise.31.1} If $X$ is a regular space, then for any two points $x, y$ in $X$ there are open sets $U$ and $V$ such that $x\in U$, $y\in V$, and $U\cap V=\emptyset$.

\paragraph{Munkres.exercise.31.2} If $A$ and $B$ are disjoint closed sets in a normal space, then there are open sets $U$ and $V$ such that $A\subset U$, $B\subset V$, and $U\cap V=\emptyset$.

\paragraph{Munkres.exercise.31.3} Prove that the order topology on a partially ordered set is regular.

\paragraph{Munkres.exercise.32.1} If $X$ is a normal space and $A$ is a closed subset of $X$, then $A$ is a normal space.

\paragraph{Munkres.exercise.32.2a} If $X_i$ is a topological space for each $i\in I$, and if $\prod_{i\in I}X_i$ is a $T_2$ space, then each $X_i$ is a $T_2$ space.

\paragraph{Munkres.exercise.32.2b} If $X_i$ is a regular space for each $i$, then $\prod_i X_i$ is regular.

\paragraph{Munkres.exercise.32.2c} If $X_i$ is a normal space for each $i$, then $\prod_i X_i$ is normal.

\paragraph{Munkres.exercise.32.3} Prove that a locally compact Hausdorff space is regular.

\paragraph{Munkres.exercise.33.7} Let $X$ be a locally compact Hausdorff space. Prove that for each closed set $A$ and each point $x$ not in $A$, there is a continuous function $f:X\to [0,1]$ such that $f(x)=1$ and $f(A)=\{0\}$.

\paragraph{Munkres.exercise.33.8} Let $X$ be a regular space. Suppose that for each $x\in X$ and each closed set $A$ not containing $x$, there is a continuous function $f:X\to [0,1]$ such that $f(x)=1$ and $f(A)=\{0\}$. Prove that if $A$ and $B$ are disjoint closed sets in $X$ and $A$ is compact, then there is a continuous function $f:X\to [0,1]$ such that $f(A)=\{0\}$ and $f(B)=\{1\}$.

\paragraph{Munkres.exercise.34.9} If $X$ is a compact space, and $X_1$ and $X_2$ are closed subsets of $X$ such that $X_1\cup X_2=X$, and $X_1$ and $X_2$ are metrizable, then $X$ is metrizable.

\paragraph{Munkres.exercise.38.4} Let $X$ be a dense subset of a compact Hausdorff space $Y$. Prove that there is a continuous surjection $g:\beta X\to Y$ such that $g(\beta X)$ is closed in $Y$ and $g(x)=x$ for all $x\in X$.

\paragraph{Munkres.exercise.38.6} Let $X$ be a regular space. Prove that $X$ is connected if and only if the Stone-Čech compactification $\beta X$ is connected.

\paragraph{Munkres.exercise.43.2} Let $X$ be a metric space, $Y$ a complete metric space, and $A$ a subset of $X$. Suppose that $f:A\to Y$ is uniformly continuous. Prove that there exists a unique continuous function $g:\overline{A}\to Y$ such that $g(x)=f(x)$ for all $x\in A$.

\paragraph{Pugh.exercise.2.109} Prove that a metric space $M$ is totally disconnected if and only if for all $x, y, z \in M$, $d(x, z) = \max\{d(x, y), d(y, z)\}$.

\paragraph{Pugh.exercise.2.126} If $E$ is an uncountable set of real numbers, then there is a point $p$ which is a cluster point of $E$.

\paragraph{Pugh.exercise.2.12a} Let $f$ be an injective function from $\mathbb{N}$ to $\mathbb{N}$, and let $p$ be a sequence of real numbers such that $p(n)\to a$ as $n\to\infty$. Prove that $p(f(n))\to a$ as $n\to\infty$.

\paragraph{Pugh.exercise.2.12b} Let $f$ be a surjective function from $\mathbb{N}$ to $\mathbb{N}$, and let $p$ be a sequence of real numbers such that $p(n)\to a$ as $n\to\infty$. Prove that $p(f(n))\to a$ as $n\to\infty$.

\paragraph{Pugh.exercise.2.137} Let $M$ be a separable metric space. If $P$ is a closed subset of $M$, then for each $x\in P$ there is a neighborhood $N$ of $x$ such that $N$ is uncountable.

\paragraph{Pugh.exercise.2.26} A set $U$ is open if and only if for each $x\in U$, there is a neighborhood of $x$ that does not contain any point of $U$.

\paragraph{Pugh.exercise.2.29} Prove that there is a bijection between the set of open sets and the set of closed sets in a metric space.

\paragraph{Pugh.exercise.2.32a} Prove that the set of all natural numbers is clopen.

\paragraph{Pugh.exercise.2.41} Prove that the closed unit ball in $\mathbb{R}^m$ is compact.

\paragraph{Pugh.exercise.2.46} Let $A$ and $B$ be compact sets in a metric space $M$ such that $A$ and $B$ are disjoint and nonempty. Prove that there exist points $a_0\in A$ and $b_0\in B$ such that $d(a_0, b_0)$ is less than or equal to $d(a, b)$ for all $a\in A$ and $b\in B$.

\paragraph{Pugh.exercise.2.57} There exists a connected set $S$ such that the interior of $S$ is not connected.

\paragraph{Pugh.exercise.2.79} Prove that a compact connected space is path-connected.

\paragraph{Pugh.exercise.2.85} Let $M$ be a compact metric space and let $U$ be a collection of open sets in $M$ such that for each $p\in M$ there are two distinct sets $U_1, U_2\in U$ containing $p$. Prove that there is a finite collection $V$ of open sets in $M$ such that for each $p\in M$ there are two distinct sets $V_1, V_2\in V$ containing $p$.

\paragraph{Pugh.exercise.2.92} Let $X$ be a topological space and let $\{A_n\}_{n=1}^\infty$ be a sequence of nonempty compact sets such that $A_n\subset A_{n+1}$ for all $n$. Prove that $\bigcap_{n=1}^\infty A_n$ is nonempty.

\paragraph{Pugh.exercise.3.1} If $f$ is a function from $\mathbb{R}$ to $\mathbb{R}$ such that $|f(x)-f(y)|\leq |x-y|^2$ for all $x,y\in\mathbb{R}$, then $f$ is constant.

\paragraph{Pugh.exercise.3.11a} Suppose that $f$ is differentiable on the open interval $(a, b)$ and that $f'$ is differentiable at $x$. Prove that $f''$ exists at $x$.

\paragraph{Pugh.exercise.3.18} If $L$ is a closed subset of $\mathbb{R}$, then there exists a continuous function $f:\mathbb{R}\to\mathbb{R}$ such that $f(x)=0$ if and only if $x\in L$.

\paragraph{Pugh.exercise.3.4} Prove that $\sqrt{n+1}-\sqrt{n}$ tends to $0$ as $n$ tends to infinity.

\paragraph{Pugh.exercise.3.63a} Prove that the function $f(x)=\frac{1}{x(\log x)^p}$ converges to $0$ as $x$ tends to infinity.

\paragraph{Pugh.exercise.3.63b} Prove that the function $f(x)=1/x(\log x)^p$ does not converge for $p\leq 1$.

\paragraph{Pugh.exercise.4.15a} Let $F$ be a set of functions from $\mathbb{R}$ to $\mathbb{R}$. Prove that $F$ is uniformly equicontinuous if and only if there exists a function $\mu:\mathbb{R}\to\mathbb{R}$ such that $\mu(x)\geq 0$ for all $x\in\mathbb{R}$, $\mu(x)\to 0$ as $x\to 0$, and $|f(s)-f(t)|\leq \mu(|s-t|)$ for all $s,t\in\mathbb{R}$ and $f\in F$.

\paragraph{Pugh.exercise.4.19} Let $M$ be a compact metric space, and let $A$ be a dense subset of $M$. Then for each $\delta > 0$ there is a finite subset $A_\delta$ of $A$ such that for each $x\in M$ there is an $i\in A_\delta$ such that $d(x, i) < \delta$.

\paragraph{Pugh.exercise.5.2} Prove that the space of continuous linear maps from $V$ to $W$ is a normed space.

\paragraph{Rudin.exercise.1.11a} Every complex number $z$ can be written in the form $z=r\omega$ where $r\in\mathbb{R}$ and $|\omega|=1$.

\paragraph{Rudin.exercise.1.12} If $z_1, \dots, z_n$ are complex, then $|z_1 + z_2 + \dots + z_n|\leq |z_1| + |z_2| + \dots + |z_n|$.

\paragraph{Rudin.exercise.1.13} If $x$ and $y$ are complex, then $|x|-|y|\leq |x-y|$.

\paragraph{Rudin.exercise.1.14} If $z$ is a complex number of modulus $1$, prove that $|1+z|^2 + |1-z|^2 = 4$.

\paragraph{Rudin.exercise.1.16a} Let $x$ and $y$ be distinct points in $\mathbb{R}^n$, $n\geq 3$. Let $d=|x-y|$ and let $r$ be a positive number such that $2r>d$. Prove that the set of points $z$ in $\mathbb{R}^n$ such that $|z-x|=r$ and $|z-y|=r$ is infinite.

\paragraph{Rudin.exercise.1.17} If $x$ and $y$ are in $\mathbb{R}^n$, then $|x+y|^2 + |x-y|^2 = 2|x|^2 + 2|y|^2$.

\paragraph{Rudin.exercise.1.18a} If $x$ is a nonzero vector in $\mathbb{R}^n$, prove that there is a nonzero vector $y$ in $\mathbb{R}^n$ such that $x\cdot y = 0$.

\paragraph{Rudin.exercise.1.18b} Prove that there is no real number $x$ such that for every real number $y$, $xy=0$.

\paragraph{Rudin.exercise.1.19} Let $a, b, c$ be three points in the plane, and let $r$ be a positive number. Prove that the following are equivalent: (1) $|x-a|=2|x-b|$; (2) $|x-c|=r$.

\paragraph{Rudin.exercise.1.1a} If $x$ is irrational and $y$ is rational, then $x+y$ is irrational.

\paragraph{Rudin.exercise.1.1b} If $x$ is irrational and $y$ is a nonzero rational number, then $xy$ is irrational.

\paragraph{Rudin.exercise.1.2} Prove that there is no rational number $x$ such that $x^2=12$.

\paragraph{Rudin.exercise.1.4} If $x$ is a lower bound of a nonempty set $S$ and $y$ is an upper bound of $S$, then $x\leq y$.

\paragraph{Rudin.exercise.1.5} Let $A$ be a nonempty set of real numbers bounded below. Prove that $\inf A = -\sup(-A)$.

\paragraph{Rudin.exercise.1.8} Prove that there is no linear order on the complex numbers.

\paragraph{Rudin.exercise.2.19a} If $A$ and $B$ are disjoint closed sets, then $A$ and $B$ are separated.

\paragraph{Rudin.exercise.2.24} Prove that every metric space $X$ in which every infinite subset has a limit point is separable.

\paragraph{Rudin.exercise.2.25} Prove that every compact metric space has a countable basis.

\paragraph{Rudin.exercise.2.27a} Let $E$ be a non-countable subset of $\mathbb{R}^k$ and let $P$ be the set of all points $x$ such that $E$ is not countable in any neighborhood of $x$. Prove that $P$ is closed and that $P$ is the set of all cluster points of $E$.

\paragraph{Rudin.exercise.2.27b} Let $E$ be a non-countable subset of $\mathbb{R}^n$ and let $P$ be the set of points $x$ such that for every neighborhood $U$ of $x$, the set $P\cap E$ is non-countable. Prove that $E\setminus P$ is countable.

\paragraph{Rudin.exercise.2.28} Let $X$ be a separable metric space and let $A$ be a closed subset of $X$. Prove that $A$ can be written as the union of two closed sets $P_1$ and $P_2$ such that $P_1$ is the set of all cluster points of $P_1$ and $P_2$ is countable.

\paragraph{Rudin.exercise.2.29} If $U$ is an open set in $\mathbb{R}$, prove that there is a sequence of open intervals $\{(a_n, b_n)\}_{n=1}^\infty$ such that $U=\bigcup_{n=1}^\infty (a_n, b_n)$.

\paragraph{Rudin.exercise.3.13} If $\sum_{i=1}^\infty |a_i|$ and $\sum_{i=1}^\infty |b_i|$ converge, then $\sum_{i=1}^\infty \sum_{j=1}^{i+1} a_jb_{i-j}$ converges.

\paragraph{Rudin.exercise.3.1a} If $f$ is a sequence of real numbers such that $f(n)$ tends to a limit $a$, prove that $|f(n)|$ tends to $|a|$.

\paragraph{Rudin.exercise.3.20} Let $p$ be a Cauchy sequence in a metric space $X$. If $p(lk)$ converges to $r$ for some $l\in\mathbb{N}$, then $p$ converges to $r$.

\paragraph{Rudin.exercise.3.21} Let $X$ be a complete metric space. Suppose that $\{E_n\}$ is a sequence of nonempty closed sets such that $E_n\supset E_{n+1}$ for all $n$ and $\lim_{n\to\infty} diam(E_n)=0$. Prove that $\cap_{n=1}^\infty E_n$ is a singleton.

\paragraph{Rudin.exercise.3.22} Let $X$ be a complete metric space, and let $\{G_n\}$ be a sequence of open dense subsets of $X$. Prove that $X$ contains a point $x$ such that $x\in G_n$ for all $n$.

\paragraph{Rudin.exercise.3.2a} Prove that $\lim_{n\to\infty} \sqrt{n^2+n} - n = 1/2$.

\paragraph{Rudin.exercise.3.3} Prove that if $f$ is a sequence of real numbers such that $f(n)<2$ for all $n$, then $f$ converges to some real number $x$.

\paragraph{Rudin.exercise.3.5} If $a_n$ and $b_n$ are bounded sequences, then $\limsup(a_n+b_n)\leq \limsup a_n + \limsup b_n$.

\paragraph{Rudin.exercise.3.6a} Prove that the sequence $\sum_{i=1}^n g_i$ converges if the sequence $g_i$ converges.

\paragraph{Rudin.exercise.3.7} If $\sum_{i=1}^n a_i$ converges, then $\sum_{i=1}^n \sqrt{a_i}/n$ converges.

\paragraph{Rudin.exercise.3.8} If $a_n$ is a sequence of real numbers such that $\sum_{n=1}^\infty a_n$ converges, and if $b_n$ is a sequence of real numbers such that $b_n\geq 0$ and $\sum_{n=1}^\infty b_n$ converges, then $\sum_{n=1}^\infty a_nb_n$ converges.

\paragraph{Rudin.exercise.4.11a} If $f:X\to Y$ is uniformly continuous and $x_n$ is a Cauchy sequence in $X$, then $f(x_n)$ is a Cauchy sequence in $Y$.

\paragraph{Rudin.exercise.4.12} If $f:X\to Y$ and $g:Y\to Z$ are uniformly continuous, then $g\circ f:X\to Z$ is uniformly continuous.

\paragraph{Rudin.exercise.4.14} Let $I$ be a linearly ordered topological space. Prove that every continuous function $f:I\to I$ has a fixed point.

\paragraph{Rudin.exercise.4.15} If $f$ is continuous and open, then $f$ is monotone.

\paragraph{Rudin.exercise.4.19} Suppose that $f$ is a function from $\mathbb{R}$ to $\mathbb{R}$ such that for each $a, b, c$ with $a < b$ and $f(a) < c < f(b)$, there is a point $x$ with $a < x < b$ and $f(x) = c$. Prove that $f$ is continuous.

\paragraph{Rudin.exercise.4.1a} There exists a function $f:\mathbb{R}\to\mathbb{R}$ such that $f$ is not continuous but for each $x\in\mathbb{R}$, the function $g_x(y)=f(x+y)-f(x-y)$ is continuous at $0$.

\paragraph{Rudin.exercise.4.21a} If $K$ and $F$ are disjoint compact and closed subsets of a metric space, then there is a positive number $\delta$ such that $d(x,y)\geq \delta$ for all $x\in K$ and $y\in F$.

\paragraph{Rudin.exercise.4.24} Suppose $f$ is continuous on $[a,b]$ and satisfies the condition $f((x+y)/2)\leq (f(x)+f(y))/2$ for all $x,y\in [a,b]$. Prove that $f$ is convex on $[a,b]$.

\paragraph{Rudin.exercise.4.2a} Let $f:X\to Y$ be a continuous mapping of a metric space $X$ into a metric space $Y$. Prove that $f(\overline{A})\subset \overline{f(A)}$ for every subset $A$ of $X$.

\paragraph{Rudin.exercise.4.3} Let $f$ be a continuous function from a metric space $X$ into $\mathbb{R}$. Prove that the set $f^{-1}(0)$ is closed.

\paragraph{Rudin.exercise.4.4a} Let $f$ be a continuous mapping of a metric space $X$ into a metric space $Y$. Prove that if $S$ is a dense subset of $X$, then $f(X)$ is a subset of the closure of $f(S)$.

\paragraph{Rudin.exercise.4.4b} Suppose $f$ and $g$ are continuous functions on a metric space $X$ and $S$ is a dense subset of $X$. Prove that if $f(x)=g(x)$ for all $x \in S$, then $f(x)=g(x)$ for all $x \in X$.

\paragraph{Rudin.exercise.4.5a} Let $f$ be a continuous function on a closed set $E$. Prove that $f$ is bounded on $E$.

\paragraph{Rudin.exercise.4.5b} There exists a set $E$ and a function $f:E\to\mathbb{R}$ such that $f$ is continuous on $E$ but there is no continuous function $g:\mathbb{R}\to\mathbb{R}$ such that $f(x)=g(x)$ for all $x\in E$.

\paragraph{Rudin.exercise.4.6} Let $E$ be a compact subset of $\mathbb{R}$ and let $f:E\to\mathbb{R}$. Prove that $f$ is continuous on $E$ if and only if the graph of $f$ is compact.

\paragraph{Rudin.exercise.4.8a} If $f$ is uniformly continuous on a bounded set $E$, then $f(E)$ is bounded.

\paragraph{Rudin.exercise.4.8b} There exists a uniformly continuous function $f:\mathbb{R}\to\mathbb{R}$ such that $f$ is unbounded on $\mathbb{R}$.

\paragraph{Rudin.exercise.5.1} Prove that if $f$ is a real-valued function on $\mathbb{R}$ such that $|f(x)-f(y)|\leq (x-y)^2$ for all $x, y \in \mathbb{R}$, then $f$ is constant.

\paragraph{Rudin.exercise.5.15} Let $f$ be a function defined on the interval $[a,b]$ and differentiable on $(a,b)$. Suppose that $f'$ and $f''$ are continuous on $[a,b]$. Prove that $|f'(x)|^2\leq 4|f(x)||f''(x)|$ for all $x\in [a,b]$.

\paragraph{Rudin.exercise.5.17} Let $f$ be a function defined on $[-1, 1]$ and differentiable on $(-1, 1)$. Suppose that $f(-1)=0$, $f(0)=0$, $f(1)=1$, and $f'(0)=0$. Prove that there exists a point $x$ in $(-1, 1)$ such that $f'''(x)\geq 3$.

\paragraph{Rudin.exercise.5.2} Let $f$ be a function defined on the interval $(a, b)$ and differentiable on $(a, b)$. Suppose that $f'(x)>0$ for all $x\in (a, b)$. Prove that the inverse function $g$ of $f$ is differentiable on $(a, b)$ and that $g'(x)=1/f'(x)$ for all $x\in (a, b)$.

\paragraph{Rudin.exercise.5.3} Suppose $g$ is a continuous function on $\mathbb{R}$ and that $g'$ is bounded. Prove that for each $N>0$ there is an $\epsilon>0$ such that the function $x\mapsto x+\epsilon g(x)$ is one-to-one on $\mathbb{R}$.

\paragraph{Rudin.exercise.5.4} Let $C_0, C_1, \dots, C_n$ be real numbers. Prove that if $\sum_{i=0}^n C_i/(i+1)=0$, then there exists $x\in [0,1]$ such that $\sum_{i=0}^n C_ix^i=0$.

\paragraph{Rudin.exercise.5.5} Suppose that $f$ is differentiable on $\mathbb{R}$ and that $f'(x)$ tends to $0$ as $x$ tends to infinity. Prove that $f(x+1)-f(x)$ tends to $0$ as $x$ tends to infinity.

\paragraph{Rudin.exercise.5.6} Suppose that $f$ is a continuous function on $[0,1]$ such that $f(0)=0$ and $f'(x)$ is monotone increasing on $[0,1]$. Prove that the function $g(x)=f(x)/x$ is monotone increasing on $(0,1]$.

\paragraph{Rudin.exercise.5.7} Suppose that $f$ and $g$ are differentiable at $x$, that $g'(x)\neq 0$, and that $f(x)=g(x)=0$. Prove that $\lim_{t\to x}\frac{f(t)}{g(t)}=\frac{f'(x)}{g'(x)}$.

\paragraph{Shakarchi.exercise.1.13a} Let $f$ be a differentiable function on an open set $\Omega$ in $\mathbb{C}$. Suppose that $f$ is constant on the real axis. Prove that $f$ is constant on $\Omega$.

\paragraph{Shakarchi.exercise.1.13b} Let $f$ be a differentiable function on an open set $\Omega$ in $\mathbb{C}$. Suppose that the imaginary part of $f$ is constant on $\Omega$. Prove that $f$ is constant on $\Omega$.

\paragraph{Shakarchi.exercise.1.13c} Let $f$ be a differentiable function on an open set $\Omega$ in $\mathbb{C}$. Suppose that $f$ is constant on $\Omega$. Prove that $f$ is constant on any connected subset of $\Omega$.

\paragraph{Shakarchi.exercise.1.19a} If $z$ is a complex number of modulus $1$, and $s_n = 1 + 2z + 3z^2 + \dots + nz^{n-1}$, prove that $s_n$ does not converge.

\paragraph{Shakarchi.exercise.1.19b} Let $z$ be a complex number of modulus $1$. Let $s_n = \sum_{i=1}^n i z/i^2$. Prove that $s_n$ converges.

\paragraph{Shakarchi.exercise.1.19c} Let $z$ be a complex number of modulus $1$ and $z\neq 1$. Let $s_n = \sum_{i=1}^n i z^i/i$. Prove that $s_n$ converges.

\end{document}
