\documentclass{article}

\title{\textbf{
Exercises from \\
\textit{Everything} \\
by All Authors
}}

\date{}

\usepackage{amsmath}
\usepackage{amssymb}
\usepackage{fullpage}

\begin{document}
\maketitle

%\paragraph{Artin.exercise.2.3.2} Let $G$ be a group, and let $a, b \in G$. Prove that there exists a unique $g \in G$ such that $b*a = g * a * b * g⁻¹$.

%\paragraph{Artin.exercise.2.8.6} Prove that the map $f : G × H → G × H$ defined by $f(g, h) = (g, h)$ is a group isomorphism.

%\paragraph{Artin.exercise.3.2.7} Suppose that $F$ is a field, and $G$ is a group. Prove that if $φ$ is a group homomorphism, then $φ$ is injective.

%\paragraph{Artin.exercise.3.7.2} Suppose that $V$ is a vector space over a field $K$ and that $Π i, v_i ∈ V$ are linearly independent. Prove that the set $\{v_i\}_{i\in I}$ is linearly independent for some $I ⊆ ι$.

\paragraph{Artin.exercise.6.4.2} If $G$ is a simple group, then $G$ is not simple.

\paragraph{Artin.exercise.6.4.12} Show that the alternating group $A_7$ is simple.

\paragraph{Artin.exercise.10.1.13} Show that if $x$ is nilpotent, then $1 + x$ is a unit.

%\paragraph{Artin.exercise.10.4.7a} Prove that $I * J = I ⊓ J$.

%\paragraph{Artin.exercise.10.6.7} Let $I$ be an ideal in a Gaussian_int $gaussian_int$. Prove that $I$ is not principal if and only if there exists a nonzero $z \in I$ such that $z$ is not a unit.

\paragraph{Artin.exercise.11.2.13} If $a$ and $b$ are integers, then $a$ divides $b$.

\paragraph{Artin.exercise.11.4.6a} Show that the polynomial $X ^ 2 + 1$ is irreducible in the ring of polynomials over $F$.

%\paragraph{Artin.exercise.11.4.6c} Show that the polynomial $X^3 - 9$ is irreducible in $\mathbb{Z}[X]$. Hint: use the fact that $X^3 - 9$ is irreducible in $\mathbb{Z}[X]$ to show that $X^3 - 9$ is irreducible in $\mathbb{Z}[X, Y]$. Deduce that $X^3 - 9$ is irreducible in $\mathbb{Z}[X, Y, Z]$. Deduce that $X^3 - 9$ is irreducible in $\mathbb{Z}[X, Y, Z, W]$. Deduce that $X^3 - 9$ is irreducible in $\mathbb{Z}[X, Y, Z, W, V]$. Deduce that $X^3 - 9$ is irreducible in $\mathbb{Z}[X, Y, Z, W, V, U]$. Deduce that $X^3 - 9$ is irreducible in $\mathbb{Z}[X, Y, Z, W, V, U, T]$. Deduce that $X^3 - 9$ is irreducible in $\mathbb{Z}[X, Y, Z, W, V, U, T, S]$. Deduce that $X^3 - 9$ is irreducible in $\mathbb{Z}[X, Y, Z, W, V, U, T, S, T', S', T'', S'', T''', S''', T''''', S''''', T''''''', S''''''', T''''''''', S''''''''', T''''''''''', S''''''''''', T''''''''''''', S''''''''''''']$. Deduce that $X^3 - 9$ is irreducible in $\mathbb{Z}[X, Y, Z, W, V, U, T, S, T', S', T'', S'', T''', S

%\paragraph{Artin.exercise.11.13.3} There exists a prime p such that p + 1 ≡ 0 [MOD 4]

\paragraph{Artin.exercise.13.6.10} Prove that the equation $x^2 + 1 = 0$ has no solutions in $K$.

\paragraph{Axler.exercise.1.2} Prove that the cube of a complex number is equal to its negative.

\paragraph{Axler.exercise.1.4} Prove that if $v$ is a nonzero vector in $V$, then $v$ is a zero vector if and only if $v$ is a zero vector.

\paragraph{Axler.exercise.1.7} Let $U$ be a submodule of $\mathbb{R}^2$ such that $U \neq \mathbb{R}^2$. Prove that there exists a vector $u \in U$ such that $u \neq 0$.

\paragraph{Axler.exercise.1.9} Let $U$ be a submodule of $V$. Prove that there exists a submodule $U'$ of $V$ such that $U'$ is a complement of $U$ and $U' \cap W = U \cap W$.

\paragraph{Axler.exercise.3.8} Let $L$ be a linear map from $V$ to $W$. Prove that there exists a subspace $U$ of $V$ such that $L$ maps $U$ isomorphically onto $L(U)$ and $L(U)$ is a direct summand of $W$.

\paragraph{Axler.exercise.5.1} Prove that the map $L$ is linear.

\paragraph{Axler.exercise.5.11} Prove that if $S$ and $T$ are commuting linear operators on a finite-dimensional vector space $V$, then the eigenvalues of $S * T$ are the same as the eigenvalues of $T * S$.

%\paragraph{Axler.exercise.5.13} Let $T$ be a linear operator on a finite-dimensional vector space $V$ over a field $F$. Prove that there exists a scalar $c$ such that $T = c • id$.

\paragraph{Axler.exercise.5.24} Prove that if $U$ is a finite-dimensional subspace of $V$, then $U$ is even.

\paragraph{Axler.exercise.6.3} Prove that if $a$ and $b$ are real numbers, then

\paragraph{Axler.exercise.6.13} Prove that the following are equivalent:

%\paragraph{Axler.exercise.7.5} Prove that if $U$ is a submodule of $End ℂ V$ such that $U.carrier ≠ {T | T * T.adjoint = T.adjoint * T}$, then $U$ is a proper submodule of $End ℂ V$.

\paragraph{Axler.exercise.7.9} Prove that if $T$ is a self-adjoint operator on a finite-dimensional inner product space, then $T$ is diagonalizable.

\paragraph{Axler.exercise.7.11} Prove that there exists a linear operator $S$ such that $S^2 = T$.

%\paragraph{Dummit-Foote.exercise.1.1.2a} There exists a function $f : ℤ → ℤ$ such that $f(a) - f(b) ≠ f(b) - f(a)$ for all $a, b ∈ ℤ$.

\paragraph{Dummit-Foote.exercise.1.1.4} Prove that the following are equivalent:

%\paragraph{Dummit-Foote.exercise.1.1.15} Prove that the map $x \mapsto x⁻¹$ is a bijection from $G$ to $G$.

\paragraph{Dummit-Foote.exercise.1.1.17} Show that $x^n = x^{n-1} x$.

\paragraph{Dummit-Foote.exercise.1.1.20} If $x$ is an element of infinite order in $G$, prove that the elements $x^n$, $n\in\mathbb{Z}$ are all distinct.

\paragraph{Dummit-Foote.exercise.1.1.22b} Prove that the order of $a * b$ is equal to the order of $b * a$.

\paragraph{Dummit-Foote.exercise.1.1.29} Prove that the following are equivalent:

\paragraph{Dummit-Foote.exercise.1.3.8} Show that the permutation group of the natural numbers is infinite.

%\paragraph{Dummit-Foote.exercise.1.6.11} Prove that the map $f : A × B → B × A$ defined by $f(a, b) = (b, a)$ is an isomorphism.

\paragraph{Dummit-Foote.exercise.1.6.23} Show that the map $x \mapsto x^{-1}$ is an automorphism of $G$.

%\paragraph{Dummit-Foote.exercise.2.1.13} Prove that H is a subgroup of ℚ if and only if H = ⊥ or H = ⊤.

%\paragraph{Dummit-Foote.exercise.2.4.16a} Prove that if $H$ is a proper subgroup of $G$, then there exists a proper subgroup $M$ of $G$ such that $H ≤ M$.

\paragraph{Dummit-Foote.exercise.2.4.16c} Prove that if $H$ is a proper subgroup of $G$, then $H$ is not a maximal subgroup of $G$.

%\paragraph{Dummit-Foote.exercise.3.1.22a} Suppose $H$ and $K$ are subgroups of $G$. Prove that $H ⊓ K$ is a subgroup of $G$.

%\paragraph{Dummit-Foote.exercise.3.2.8} Prove that $H ⊓ K = ⊥$ if $H$ and $K$ are subgroups of $G$ and $H$ and $K$ have coprime cardinalities.

\paragraph{Dummit-Foote.exercise.3.2.16} Prove that if $a$ is coprime to $p$, then $a^p \equiv a [ZMOD p]$.

\paragraph{Dummit-Foote.exercise.3.3.3} Prove that if $H$ is a $p$-subgroup of $G$, then $H$ is normal in $G$.

\paragraph{Dummit-Foote.exercise.3.4.4} Let $G$ be a group, and let $H$ be a subgroup of $G$ of index $n$. Prove that $H$ is a normal subgroup of $G$.

\paragraph{Dummit-Foote.exercise.3.4.5b} Prove that if $H$ is a normal subgroup of $G$, then $G$ is solvable.

\paragraph{Dummit-Foote.exercise.4.2.8} Let $H$ be a subgroup of $G$ of index $n$. Prove that $H$ is a normal subgroup of $G$.

\paragraph{Dummit-Foote.exercise.4.2.9a} Suppose that $H$ is a $p$-subgroup of $G$, and that $H$ has index $p$ in $G$. Prove that $H$ is normal.

\paragraph{Dummit-Foote.exercise.4.4.2} Prove that if $G$ is a finite group of order $p*q$, then $G$ is cyclic.

\paragraph{Dummit-Foote.exercise.4.4.6b} There exists a non-abelian group G such that G.characteristic = 0 and G.normal = G.

\paragraph{Dummit-Foote.exercise.4.4.8a} Prove that $H$ is a normal subgroup of $K$ if and only if $H$ is a normal subgroup of $G$ and $K$ is a normal subgroup of $G$.

\paragraph{Dummit-Foote.exercise.4.5.13} There exists a Sylow 7-subgroup of $G$.

\paragraph{Dummit-Foote.exercise.4.5.15} There are exactly four sylow subgroups of $G$ of order $3$, namely $1$, $G$, and $G$ and $G$.

\paragraph{Dummit-Foote.exercise.4.5.17} Show that the Sylow 5 and Sylow 7 subgroups of $G$ are nonempty.

\paragraph{Dummit-Foote.exercise.4.5.19} Show that the group of order $6545$ is not simple.

\paragraph{Dummit-Foote.exercise.4.5.21} Show that the group of order $2907$ is not simple.

\paragraph{Dummit-Foote.exercise.4.5.23} Show that the group of order $462$ is not simple.

\paragraph{Dummit-Foote.exercise.4.5.33} Prove that if $H$ is a $p$-subgroup of $G$, then $H$ is a Sylow $p$-subgroup of $G$.

\paragraph{Dummit-Foote.exercise.7.1.2} If $u$ is a unit, then $-u$ is a unit.

\paragraph{Dummit-Foote.exercise.7.1.12} Show that if $F$ is a field, then $F[x]$ is a domain.

%\paragraph{Dummit-Foote.exercise.7.2.2} Prove that if $p$ is a polynomial with coefficients in a ring $R$, then $p$ is a zero divisor if and only if there exists a nonzero $b$ such that $b • p = 0$.

%\paragraph{Dummit-Foote.exercise.7.3.16} Suppose $R$ is a ring, $S$ is a ring, and $φ$ is a surjective ring homomorphism. Prove that $φ '' (center R) ⊂ center S$.

\paragraph{Dummit-Foote.exercise.7.4.27} Prove that if $a$ is nilpotent, then $1-a*b$ is a unit.

\paragraph{Dummit-Foote.exercise.8.2.4} Prove that a principal ideal is generated by a single element.

\paragraph{Dummit-Foote.exercise.8.3.5a} Prove that the polynomial $x^2 - n$ is irreducible over $\mathbb{Z}$.

\paragraph{Dummit-Foote.exercise.8.3.6b} Show that the Gaussian integers are a field.

%\paragraph{Dummit-Foote.exercise.9.1.10} Show that the minimal primes of $mv_polynomial ℕ ℤ ⧸ ideal.span (range f)$ are precisely the minimal primes of $ideal.span (range f)$.

%\paragraph{Dummit-Foote.exercise.9.4.2a} Show that the polynomial $X^4 - 4X^3 + 6$ is irreducible in ℤ[X], and that the polynomial $X^4 - 4X^3 + 6$ is irreducible in ℤ[X, X^2, X^3, X^4]. Deduce that the polynomial $X^4 - 4X^3 + 6$ is irreducible in ℤ[X, X^2, X^3, X^4, X^5, X^6, X^7, X^8, X^9, X^10, X^11, X^12, X^13, X^14, X^15, X^16, X^17, X^18, X^19, X^20, X^21, X^22, X^23, X^24, X^25, X^26, X^27, X^28, X^29, X^30, X^31, X^32, X^33, X^34, X^35, X^36, X^37, X^38, X^39, X^40, X^41, X^42, X^43, X^44, X^45, X^46, X^47, X^48, X^49, X^50, X^51, X^52, X^53, X^54, X^55, X^56, X^57, X^58, X^59, X^60, X^61, X^62, X^63, X^64, X^65, X^66, X^67, X^68, X^69, X^70, X^71, X^72, X^73, X^74, X^75, X^76, X^77, X^78, X^79,

%\paragraph{Dummit-Foote.exercise.9.4.2c} Show that the polynomial $X^4 + 4*X^3 + 6*X^2 + 2*X + 1$ is irreducible in ℤ[X].

\paragraph{Dummit-Foote.exercise.9.4.9} Prove that the polynomial $X^2 - C \sqrt{d}$ is irreducible in $\mathbb{Q}[X]$ for $d \neq 0$.

%\paragraph{Dummit-Foote.exercise.11.1.13} Prove that the map $f : ι → ℝ$ defined by $f(i) = i$ is uniformly continuous.

%\paragraph{Herstein.exercise.2.1.18} Let $G$ be a group of even order. Prove that there exists an element $a$ of order $2$ in $G$ such that $a = a⁻¹$.

\paragraph{Herstein.exercise.2.1.26} Let $G$ be a group, and let $a$ be an element of $G$ of infinite order. Prove that there exists a natural number $n$ such that $a^n = 1$.

\paragraph{Herstein.exercise.2.2.3} Prove that the commutator subgroup of a group is a normal subgroup.

\paragraph{Herstein.exercise.2.2.6c} Prove that if $G$ is a group and $n$ is a natural number greater than $1$, then the following are equivalent:

(a) $G$ is abelian;
(b) $G$ is abelian and $n$ is even;
(c) $G$ is abelian and $n$ is odd;
(d) $G$ is abelian and $n$ is odd;
(e) $G$ is abelian and $n$ is even;
(f) $G$ is abelian and $n$ is even;
(g) $G$ is abelian and $n$ is odd;
(h) $G$ is abelian and $n$ is even;
(i) $G$ is abelian and $n$ is odd;
(j) $G$ is abelian and $n$ is even;
(k) $G$ is abelian and $n$ is odd;
(l) $G$ is abelian and $n$ is even;
(m) $G$ is abelian and $n$ is even;
(n) $G$ is abelian and $n$ is odd;
(o) $G$ is abelian and $n$ is even;
(p) $G$ is abelian and $n$ is odd;
(q) $G$ is abelian and $n$ is even;
(r) $G$ is abelian and $n$ is odd;
(s) $G$ is abelian and $n$ is even;
(t) $G$ is abelian and $n$ is odd;
(u) $G$ is abelian and $n$ is even;
(v) $G$ is abelian and $n$ is odd;
(w) $G$ is abelian and $n$ is even;
(x) $G$ is

\paragraph{Herstein.exercise.2.3.16} Suppose that $G$ is a group, and $H$ is a subgroup of $G$ such that $H$ is not the trivial group. Prove that $H$ is not cyclic.

\paragraph{Herstein.exercise.2.5.23} rove that if $G$ is a group, then for all $a, b \in G$, there exists $j \in \mathbb{Z}$ such that $b*a = a^j * b$.

\paragraph{Herstein.exercise.2.5.31} Show that if $H$ is a $p$-subgroup of $G$, then the index of $H$ inside its normalizer is congruent modulo $p$ to the index of $H$.

\paragraph{Herstein.exercise.2.5.43} Prove that the commutator subgroup of a group of order 9 is trivial.

%\paragraph{Herstein.exercise.2.5.52} Prove that if $G$ is a group, then $φ$ is a group automorphism.

%\paragraph{Herstein.exercise.2.7.7} Prove that if $N$ is a normal subgroup of $G$, then $φ(N)$ is a normal subgroup of $G'$.

\paragraph{Herstein.exercise.2.8.15} Prove that there is a group isomorphism between $G$ and $H$.

%\paragraph{Herstein.exercise.2.10.1} Prove that $A ⊓ (closure {b}) = ⊥$ if $A$ is a subgroup of $G$ and $b$ is an element of $G$ of infinite order.

\paragraph{Herstein.exercise.2.11.7} If $P$ is a $p$-Sylow subgroup of $G$, then the index of $P$ inside its normalizer is congruent modulo $p$ to the index of $P$.

%\paragraph{Herstein.exercise.3.2.21} Show that if σ and τ are permutations of α, then σ = 1 ∧ τ = 1 if and only if σ = 1 ∧ τ = 1.

\paragraph{Herstein.exercise.4.1.34} Show that the general linear group of degree 3 over the field of two elements is isomorphic to the symmetric group on three elements.

\paragraph{Herstein.exercise.4.2.6} Prove that the following are equivalent:
  (a) a * (a * x + x * a) = (x + x * a) * a
  (b) a * (a * x + x * a) = (x + x * a) * a
  (c) a * (a * x + x * a) = (x + x * a) * a
  (d) a * (a * x + x * a) = (x + x * a) * a
  (e) a * (a * x + x * a) = (x + x * a) * a
  (f) a * (a * x + x * a) = (x + x * a) * a
  (g) a * (a * x + x * a) = (x + x * a) * a
  (h) a * (a * x + x * a) = (x + x * a) * a
  (i) a * (a * x + x * a) = (x + x * a) * a
  (j) a * (a * x + x * a) = (x + x * a) * a
  (k) a * (a * x + x * a) = (x + x * a) * a
  (l) a * (a * x + x * a) = (x + x * a) * a
  (m) a * (a * x + x * a) = (x + x * a) * a
  (n) a * (a * x + x * a) = (x + x * a) * a
  (o) a * (a * x + x * a) = (x + x * a) * a
  

\paragraph{Herstein.exercise.4.3.1} Let $R$ be a commutative ring, and let $a$ be an element of $R$. Prove that the set of elements $x$ of $R$ such that $x*a=0$ is an ideal of $R$.

\paragraph{Herstein.exercise.4.4.9} Prove that there exists a set of $p$ elements of $\mathbb{Z}$ such that the sum of the squares of the elements is equal to $p^2$.

%\paragraph{Herstein.exercise.4.5.23} Prove that the polynomial $p$ is irreducible in $\mathbb{Z}[X]$ and that the polynomial $q$ is irreducible in $\mathbb{Z}[X]$. Deduce that the polynomial $p$ is irreducible in $\mathbb{Z}[X]$ and that the polynomial $q$ is irreducible in $\mathbb{Z}[X]$. Deduce that the polynomial $p$ is irreducible in $\mathbb{Z}[X]$ and that the polynomial $q$ is irreducible in $\mathbb{Z}[X]$. Deduce that the polynomial $p$ is irreducible in $\mathbb{Z}[X]$ and that the polynomial $q$ is irreducible in $\mathbb{Z}[X]$. Deduce that the polynomial $p$ is irreducible in $\mathbb{Z}[X]$ and that the polynomial $q$ is irreducible in $\mathbb{Z}[X]$. Deduce that the polynomial $p$ is irreducible in $\mathbb{Z}[X]$ and that the polynomial $q$ is irreducible in $\mathbb{Z}[X]$. Deduce that the polynomial $p$ is irreducible in $\mathbb{Z}[X]$ and that the polynomial $q$ is irreducible in $\mathbb{Z}[X]$. Deduce that the polynomial $p$ is irreducible in $\mathbb{Z}[X]$ and that the polynomial $q$ is irreducible in $\mathbb{Z}[X]$. Deduce that the polynomial $p$ is irreducible in $\mathbb{Z}[X]$ and that the polynomial $q$ is irreducible in $\mathbb{Z}[X]$. Deduce that the polynomial $p$ is irreducible in $\mathbb{Z}[X]$ and that the polynomial $q$ is irreducible in $\mathbb{Z}[X]$. Deduce that the polynomial $p$ is irreducible in $\mathbb{Z}[X]$ and that the polynomial $q$ is irreducible in $\mathbb{Z}[X]$. Deduce that the polynomial $p$ is irreducible in $\mathbb{Z

%\paragraph{Herstein.exercise.4.6.2} Show that the polynomial $X^3 + 3X + 2$ is irreducible in ℚ[X], and that the polynomial $X^3 + 3X + 2$ is irreducible in ℚ[X + 1/2]. Deduce that the polynomial $X^3 + 3X + 2$ is irreducible in ℚ[X + 1]. Deduce that the polynomial $X^3 + 3X + 2$ is irreducible in ℚ[X + 1/3]. Deduce that the polynomial $X^3 + 3X + 2$ is irreducible in ℚ[X + 1/4]. Deduce that the polynomial $X^3 + 3X + 2$ is irreducible in ℚ[X + 1/5]. Deduce that the polynomial $X^3 + 3X + 2$ is irreducible in ℚ[X + 1/6]. Deduce that the polynomial $X^3 + 3X + 2$ is irreducible in ℚ[X + 1/7]. Deduce that the polynomial $X^3 + 3X + 2$ is irreducible in ℚ[X + 1/8]. Deduce that the polynomial $X^3 + 3X + 2$ is irreducible in ℚ[X + 1/9]. Deduce that the polynomial $X^3 + 3X + 2$ is irreducible in ℚ[X + 1/10]. Deduce that the polynomial $X^3 + 3X + 2$ is irreducible in ℚ[X + 1/11]. Deduce that the polynomial $X^3 + 3X + 2$ is irreducible in ℚ[X + 1/12]. Deduce that the polynomial $X^3 + 3X + 2$ is irreducible in ℚ[X + 1/13]. Deduce that the polynomial $X^3 + 3X + 2$ is

%\paragraph{Herstein.exercise.5.1.8} Prove that if $p$ is prime, then the binomial coefficients $\binom{m}{k}$ are divisible by $p$ for all $k$ such that $0 ≤ k ≤ m$.

\paragraph{Herstein.exercise.5.3.7} Show that if $a$ is algebraic over $F$, then $a$ is algebraic over $F(a)$.

\paragraph{Herstein.exercise.5.4.3} rove that the polynomial $p$ of degree $80$ with coefficients in $\mathbb{Q}$ has a root in $\mathbb{Q}$.

\paragraph{Herstein.exercise.5.6.14} Show that the cardinality of the root set of $X^m - X$ is $m$.

%\paragraph{Ireland-Rosen.exercise.1.30} There is no integer $a$ such that $\sum_{i=1}^n (1 : ℚ) / (n+2) = a$.

\paragraph{Ireland-Rosen.exercise.2.4} Prove that the function $f_a$ is uniformly continuous.

\paragraph{Ireland-Rosen.exercise.2.27a} Suppose that $p$ is a prime number. Show that the sequence $(1 / p^n)$ is not summable.

\paragraph{Ireland-Rosen.exercise.3.4} There is no integer $x$ such that $3*x^2 + 2 = y^2$ for all integers $y$.

%\paragraph{Ireland-Rosen.exercise.3.10} Prove that $n! ≡ 0 [MOD n]$.

\paragraph{Ireland-Rosen.exercise.4.4} Prove that if $p$ is prime, then $a$ is a primitive root of $p$ if and only if $-a$ is a primitive root of $p$.

\paragraph{Ireland-Rosen.exercise.4.6} Show that the polynomial $x^3 - p$ is irreducible over $\mathbb{Z}$.

\paragraph{Ireland-Rosen.exercise.4.11} Prove that if $p$ is prime, then $p^k$ is prime.

%\paragraph{Ireland-Rosen.exercise.5.28} Prove that the equation $x^4 ≡ 2 [MOD p]$ has a solution if and only if $p ≡ 1 [MOD 4]$.

%\paragraph{Ireland-Rosen.exercise.12.12} Show that the polynomial $x^2 - 2$ is irreducible over ℚ.

\paragraph{Munkres.exercise.13.3b} Prove that if $X$ is a set, then $X$ is infinite if and only if $X$ is infinite and $X$ is not empty.

\paragraph{Munkres.exercise.13.4a2} There exists a set $X$ and a family of sets $\{T_i\}_{i\in I}$ such that $T_i$ is a topology on $X$ for all $i\in I$, and $T_i$ is not a topology on $X$ for all $i\in I$.

%\paragraph{Munkres.exercise.13.4b2} Prove that if $X$ is a set, $T$ is a topology on $X$, and $T'$ is a topology on $X$ such that $T' ⊆ T$, then $T' = T$.

\paragraph{Munkres.exercise.13.5b} Show that the topology generated by $A$ is the smallest topology on $X$ such that $A$ is a subset of the topology.

%\paragraph{Munkres.exercise.13.8a} Show that the set of open intervals in ℝ is a basis for the topology of ℝ.

\paragraph{Munkres.exercise.16.1} Prove that the following are equivalent:
  (1) A is open.
  (2) A = (subtype.val '' A).
  (3) A = (subtype.val '' (subtype.val '' A)).
  (4) A = (subtype.val '' (subtype.val '' (subtype.val '' A))).
  (5) A = (subtype.val '' (subtype.val '' (subtype.val '' (subtype.val '' A)))).
  (6) A = (subtype.val '' (subtype.val '' (subtype.val '' (subtype.val '' A)))).
  (7) A = (subtype.val '' (subtype.val '' (subtype.val '' (subtype.val '' A)))).
  (8) A = (subtype.val '' (subtype.val '' (subtype.val '' (subtype.val '' A)))).
  (9) A = (subtype.val '' (subtype.val '' (subtype.val '' (subtype.val '' A)))).
  (10) A = (subtype.val '' (subtype.val '' (subtype.val '' (subtype.val '' A)))).
  (11) A = (subtype.val '' (subtype.val '' (subtype.val '' (subtype.val '' A)))).
  (12) A = (subtype.val '' (subtype.val '' (subtype.val '' (subtype.val '' A)))).
  (13) A = (subtype.val '' (subtype.val '' (subtype.val '' (subtype.val '' A)))).
  (14) A = (subtype.val '' (

\paragraph{Munkres.exercise.16.6} Show that the set of all rational numbers in the open interval $(a, b)$ is a topological basis for the topology on $\mathbb{R}$.

%\paragraph{Munkres.exercise.18.8a} Prove that if $f$ and $g$ are continuous, then the set of points $x$ such that $f(x) ≤ g(x)$ is closed.

\paragraph{Munkres.exercise.18.13} Suppose $A$ is a subset of $X$, and $f$ is a continuous function from $A$ to $Y$. Prove that $f$ is continuous.

%\paragraph{Munkres.exercise.20.2} Prove that the order topology on ℝ × ℝ is not metrizable.

%\paragraph{Munkres.exercise.21.6b} There exists a function $f : ℕ → ℝ$ such that $f(n) = n$ for all $n$, and $f$ is continuous at $0$.

%\paragraph{Munkres.exercise.22.2a} Prove that the quotient map $p$ is a quotient map if and only if there exists a continuous function $f : Y → X$ such that $p ∘ f = id$.

%\paragraph{Munkres.exercise.22.5} Prove that if $p$ is continuous, then $p ∘ subtype.val$ is continuous.

%\paragraph{Munkres.exercise.23.3} Prove that if $A$ is a connected subset of $X$, then $A ∪ (⋃ n, A n)$ is connected.

%\paragraph{Munkres.exercise.23.6} Suppose $C$ is a connected subset of $X$, and $A$ is a subset of $X$ such that $C ∩ A ≠ ∅$. Prove that $C ∩ (frontier A) ≠ ∅$.

\paragraph{Munkres.exercise.23.11} Suppose $X, Y$ are topological spaces, and $Y$ is connected. Let $p$ be a quotient map from $X$ onto $Y$. Prove that $p$ is a quotient map if and only if $p$ is a quotient map.

\paragraph{Munkres.exercise.24.3a} Prove that if $f$ is continuous, then $f$ is constant.

\paragraph{Munkres.exercise.25.9} Let $G$ be a topological group. Prove that $C$ is a normal subgroup of $G$ if and only if $C$ is a connected component of $G$.

\paragraph{Munkres.exercise.26.12} Prove that $p$ is a closed map.

\paragraph{Munkres.exercise.28.4} Prove that a topological space is countably compact if and only if it is limit point compact.

\paragraph{Munkres.exercise.28.6} Prove that $f$ is bijective.

\paragraph{Munkres.exercise.29.4} There is no locally compact space that is compact and Hausdorff.

\paragraph{Munkres.exercise.30.10} Let $X$ be a topological space. Prove that there exists a countable dense subset of $X$.

%\paragraph{Munkres.exercise.31.1} Suppose $X$ is a topological space, and $x, y \in X$. Prove that there exists an open set $U$ containing $x$ and an open set $V$ containing $y$ such that $U ∩ V = ∅$.

\paragraph{Munkres.exercise.31.3} Prove that a topological space is regular if and only if it is Hausdorff, and for every point $x$ and every neighborhood $U$ of $x$, there exists a neighborhood $V$ of $x$ such that $V \subseteq U$ and $V \cap U = \emptyset$.

\paragraph{Munkres.exercise.32.2a} Suppose that $X$ is a topological space, and that $X$ is nonempty. Prove that $X$ is a $t2_space$.

\paragraph{Munkres.exercise.32.2c} Suppose $X$ is a topological space, and $Y$ is a normal space. Let $f$ map $X$ into $Y$, and let $g$ be a continuous one-to-one mapping of $Y$ into $Z$. Prove that $f$ is continuous if $g$ is continuous.

\paragraph{Munkres.exercise.33.7} Prove that if $X$ is a locally compact space, then the following are equivalent:

(1) $X$ is compact.
(2) $X$ is second-countable.
(3) $X$ is separable.
(4) $X$ is Lindelöf.
(5) $X$ is completely regular.
(6) $X$ is completely regular and second-countable.
(7) $X$ is completely regular and Lindelöf.
(8) $X$ is completely regular and second-countable.
(9) $X$ is completely regular and Lindelöf.
(10) $X$ is completely regular and second-countable.
(11) $X$ is completely regular and completely regular.
(12) $X$ is completely regular.
(13) $X$ is completely regular.
(14) $X$ is completely regular.
(15) $X$ is completely regular.
(16) $X$ is completely regular.
(17) $X$ is completely regular.
(18) $X$ is completely regular.
(19) $X$ is completely regular.
(20) $X$ is completely regular.
(21) $X$ is completely regular.
(22) $X$ is completely regular.
(23) $X$ is completely regular.
(24) $X$ is completely regular.
(25) $X$ is completely regular.
(26) $X$ is completely regular.
(27) $X$ is completely regular.
(28) $X$ is completely regular.
(29) $X$ is completely regular.
(30) $X$ is completely regular.
(31) $X$ is completely regular.
(32) $X$ is completely

\paragraph{Munkres.exercise.34.9} Prove that the union of two compact sets is compact.

%\paragraph{Munkres.exercise.43.2} Prove that if $A$ is a closed subset of $X$, then there exists a continuous function $f : X → Y$ such that $f$ is continuous on $A$ and $f$ is uniformly continuous on $A$.

\paragraph{Pugh.exercise.2.26} Prove that a set is open if and only if it contains all of its cluster points.

\paragraph{Pugh.exercise.2.32a} Show that the set of all numbers that are not in A is closed.

%\paragraph{Pugh.exercise.2.46} Suppose $M$ is a metric space, and $A, B$ are disjoint nonempty compact subsets of $M$. Prove that there exist $a₀, b₀$ such that $a₀ ∈ A$, $b₀ ∈ B$, and $dist a₀ b₀ ≤ dist a b$ for all $a ∈ A$ and $b ∈ B$.

\paragraph{Pugh.exercise.2.92} Prove that if $s$ is a sequence of nonempty compact sets, then the intersection of all the sets in the sequence is nonempty.

\paragraph{Pugh.exercise.3.1} Prove that if $f$ is a continuous function on the real line, then $f$ is constant.

\paragraph{Pugh.exercise.3.63a} Prove that the function $f$ is continuous at $1$.

%\paragraph{Pugh.exercise.4.15a} Prove that the following are equivalent: \paragraph{Putnam.exercise.1998.b6} There is no integer n such that n > 0 ∧ sqrt (n^3 + a*n^2 + b*n + c) = n.

\paragraph{Putnam.exercise.1999.b4} Show that if $f$ is differentiable at $x$, then $f'(x) < 2 * f(x)$.

\paragraph{Putnam.exercise.2001.a5} Prove that there are no solutions to the equation $a^n - (a+1)^n = 2001$ in positive integers $a$ and $n$.

\paragraph{Putnam.exercise.2014.a5} Prove that the polynomial $P$ is irreducible.

\paragraph{Putnam.exercise.2018.a5} Prove that there exists a sequence of real numbers $(x_n)_{n\in\mathbb{N}}$ such that $x_0 = 0$ and $x_{n+1} = f(x_n)$ for all $n\in\mathbb{N}$.

\paragraph{Putnam.exercise.2018.b4} Prove that there exists a periodic function $f$ such that $f(0) = a$ and $f(n) = 0$ for all $n > 0$.

\paragraph{Rudin.exercise.1.1b} Suppose that $x$ is irrational. Then $x * y$ is irrational.

%\paragraph{Rudin.exercise.1.4} If $x$ and $y$ are elements of $s$, then $x ≤ y$.

%\paragraph{Rudin.exercise.1.8} There is no linear order on ℂ that is compatible with the usual order on ℝ.

\paragraph{Rudin.exercise.1.12} Prove that if $f$ is a complex-valued function on a finite set $S$, then $|f|$ is a real-valued function on $S$.

\paragraph{Rudin.exercise.1.14} Prove that the square of the absolute value of a complex number is equal to the sum of the squares of the absolute values of its real and imaginary parts.

\paragraph{Rudin.exercise.1.17} Prove that the square of the Euclidean norm is a norm.

%\paragraph{Rudin.exercise.1.18b} There is no function $f : ℝ → ℝ$ such that $f(x) * f(y) = 0$ for all $x, y \in ℝ$.

%\paragraph{Rudin.exercise.2.19a} Suppose $A$ and $B$ are disjoint closed sets in a metric space $X$. Prove that there exists a continuous function $f : X → [0, 1]$ such that $f(A) = \{0\}$ and $f(B) = \{1\}$.

\paragraph{Rudin.exercise.2.25} Let $K$ be a compact metric space. Prove that there exists a countable basis for the topology of $K$.

\paragraph{Rudin.exercise.2.27b} Show that if $E$ is a nonempty set of real numbers, then $E$ is countable if and only if $E$ is uncountable.

\paragraph{Rudin.exercise.2.29} Prove that the set of all real numbers is the union of a countable family of open intervals.

\paragraph{Rudin.exercise.3.2a} Prove that the sequence of functions $f_n(x) = \sqrt{x^2 + n^2} - n$ converges uniformly to $f(x) = \sqrt{x^2 + 1} - 1$ on the interval $[0, 1]$.

\paragraph{Rudin.exercise.3.5} Prove that if $a$ and $b$ are two real sequences, then

\paragraph{Rudin.exercise.3.7} Prove that the sequence of functions $f_n(x) = \sqrt{x^2 + n}$ converges uniformly to $f(x) = \sqrt{x^2}$.

\paragraph{Rudin.exercise.3.13} Prove that if $f$ is a continuous function from a compact space $X$ into a metric space $Y$, then $f$ is uniformly continuous.

%\paragraph{Rudin.exercise.3.21} Prove that if $E$ is a sequence of sets in a metric space $X$ such that $E n ⊃ E (n + 1)$ for all $n$, then $E$ is a Cauchy sequence.

%\paragraph{Rudin.exercise.4.1a} There exists a continuous function $f : ℝ → ℝ$ such that $f(x) = 0$ for all $x \in ℝ$ and $f(x) = 1$ for all $x \in ℝ \setminus ℤ$.

\paragraph{Rudin.exercise.4.3} Prove that $f$ is continuous if $f$ is continuous.

\paragraph{Rudin.exercise.4.4b} Prove that $f = g$.

%\paragraph{Rudin.exercise.4.5b} There exists a continuous function $f : ℝ → ℝ$ such that $f(x) = x$ for all $x \in E$.

\paragraph{Rudin.exercise.4.8a} Suppose $E$ is a metric space, and $f$ is a continuous function from $E$ into $\mathbb{R}$. Prove that $f$ is uniformly continuous.

\paragraph{Rudin.exercise.4.11a} Suppose $X$ is a metric space, and $Y$ is a metric space. Let $f$ map $X$ into $Y$, and let $x$ be a Cauchy sequence in $X$. Prove that $f(x)$ is a Cauchy sequence in $Y$.

\paragraph{Rudin.exercise.4.15} Prove that if $f$ is monotone, then $f$ is continuous.

%\paragraph{Rudin.exercise.4.21a} Suppose $X$ is a metric space, $K$ is a compact subset of $X$, and $F$ is a closed subset of $X$ disjoint from $K$. Prove that there exists a positive real number δ such that for all p q : X, p ∈ K → q ∈ F → dist p q ≥ δ.

\paragraph{Rudin.exercise.5.1} Prove that the function $f$ defined by $f(x) = x^2$ is continuous.

\paragraph{Rudin.exercise.5.3} Prove that if $g$ is continuous and injective, then $g$ is strictly increasing.

\paragraph{Rudin.exercise.5.5} Prove that if $f$ is differentiable at $x$, then $f(x+1)-f(x)$ tends to $0$ as $x$ tends to $x$.

\paragraph{Rudin.exercise.5.7} Suppose that $f$ and $g$ are differentiable at $x$, and that $f(x) \neq 0$ and $g(x) \neq 0$. Prove that $f(x) / g(x)$ tends to $1$ as $x$ tends to $x$.

\paragraph{Rudin.exercise.5.17} Prove that there exists a point $x$ in the open interval $(-1, 1)$ such that $f$ is differentiable at $x$ and $f'(x) = 3$.

\paragraph{Shakarchi.exercise.1.13b} Prove that if $f$ is differentiable at $a$, then $f$ is differentiable at $b$.

\paragraph{Shakarchi.exercise.1.19a} Show that the sequence $s$ is not uniformly convergent.

\paragraph{Shakarchi.exercise.1.19c} Show that the sequence of partial sums of the series $\sum_{n=1}^\infty s(n) z^n$ converges to a complex number $z$.

%\paragraph{Shakarchi.exercise.2.2} Prove that the function $f : ℝ → ℝ$ defined by $f(x) = \int_0^x \sin t \, dt$ is continuous at $x = 0$.

%\paragraph{Shakarchi.exercise.2.13} Prove that the Taylor series of $f$ at $z₀$ converges to $f$.

%\paragraph{Shakarchi.exercise.3.4} Prove that the function $f : ℝ → ℝ$ defined by $f(x) = x * real.sin x / (x ^ 2 + a ^ 2)$ is continuous at $x = 0$.

\paragraph{Shakarchi.exercise.3.14} Prove that if $f$ is differentiable at $z_0$, then $f$ is linear.

\paragraph{Shakarchi.exercise.5.1} Prove that the sequence of partial sums of the series $\sum_{n=1}^\infty \frac{1}{n^2}$ converges to a limit, and that the sequence of partial sums of the series $\sum_{n=1}^\infty \frac{1}{n^2} \sin n$ converges to a limit.
\end{document}
